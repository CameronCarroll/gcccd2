\documentclass{article}

% Set up double spacing
\usepackage{setspace}
\doublespacing

\usepackage{hyperref}
\usepackage[top=1in, bottom=1.5in]{geometry}
\usepackage[ampersand]{easylist}


% Not sure if this does anything?
\raggedright

% Set up subordination
\renewcommand\thesection{\Roman{section}}
\renewcommand\thesubsection{\Alph{subsection}}
\renewcommand\thesubsubsection{\arabic{subsubsection}}
\renewcommand\theparagraph{\alph{paragraph}}
\renewcommand\thesubparagraph{\roman{subparagraph}}

\title{Another Hydraulic Fracturing 101}
\date{October 2, 2014}
\author{Cameron Carroll\\ Grossmont College, Communication-122-1947}


\begin{document}
  \maketitle
  \section*{}
    \begin{description}
      \item[General Purpose:] To inform.
      \item[Specific Purpose:] To inform my audience about hydraulic fracturing,      or `fracking.'
    \end{description}

  \section{Introduction}
    \marginpar{\footnotesize{Attn Getter}}
    \subsection{Attention Getter}
      \begin{easylist}
        \NewList
        & A weight of sand equal to the Statue of Liberty is mixed with water to fill an Olympic swimming pool (and a chemical cocktail) and forced 1-2 miles beneath the surface in order to crack a little slice of our planet.
        & A man turns on the tap water on his property and holds up a lighter: It ignites and continues to burn.
        & Hundreds of thousands of gallons of wastewater including the chemicals pumped down and radioactive material gathered on the way up sits in giant open pits -- and often overflows.
        & It's no wonder fracking is widely considered to be `dirty drilling.'
      \end{easylist}
    \marginpar{\footnotesize{Reason to Listen}}
    \subsection{Fracking is a hot topic in American politics and will influence energy, economy and the environment for many years to come.}
      \begin{easylist}
        \NewList
        & By listening to me today, I intend to get you up to speed on the implementation and technical/environmental challenges of this important technique.
        & Fracking allows us to get gas and oil we previously assumed was impossible to recover. (Stephen 2012)
        & Gas has been seen as a `balancing fuel,' compensating for the highs and lows of renewable resources. (Wood 2014)
        & Gas has also been seen as a `bridge' to a renewable-based low-carbon future. (Wood 2014)
        & It is a boon to a number of industries (rail, plastics, tools,         tires, trucks, just to name a very few.)
        & It's hugely controversial for its yet-unknown environmental
        impact.
      \end{easylist}
      \marginpar{\footnotesize{Speaker Credibility}}
    \subsection{I've read a lot of industry journals, research articles, and    government publications trying to get an unbiased survey.}
    \marginpar{\footnotesize{Thesis \& Preview}}
    \subsection{I would like to better your understanding of fracking as a    technology and its consequences for our country.}
      \begin{easylist}
        \NewList
        & First I will cover the history and purpose of fracking.
        & Next, I would like to do a survey of the technical and        logistical aspects of the practice.
        & Finally, I would like to discuss a few of the environmental impacts.
      \end{easylist}

  \section{Body}
    \subsection{History of Fracking/Shale Gas}
      \begin{easylist}
        \NewList
        \ListProperties(Style*=,Numbers=a, Numbers2=l,FinalMark={)})
        & The story of fracking is twofold: The technology, an established well stimulation technique, versus the modern boom which is married to shale gas.
        & In short, fracking is the use of high-pressure slurry in gas wells, where the liquid forces fractures open and sand (proppant) holds them open after the pressure is released.
        & Originally used to stimulate conventional-fuel wells (where gas/oil is trapped in large pockets between rocks,) fracking is now a standard part of acquiring unconventional resources. (Sjolander 2011)
        & In an unconventional reservoir, like shale, gas is tightly stored inside of the rock itself rather than between grains. (Sjolander 2011)
        & A brief history...
          && Back into the 1800's, `shooting a well' involved setting off dynamite at the bottom of a vertical well  in order to open up paths for oil and gas to escape. (Sjolander 2011)
          && In 1949, Halliburton used fracking at industrial scale for the first time but at much lower volumes and pressures than are used today. (Headwaters 2012)
          && In the 90's, fracking was being combined with horizontally-drilled wells, slickwater (gel and water combined) and new higher-pressure pumps which together allowed access to previously uneconomical gas. (Headwaters 2012)
          && In the early 2000's, extracting unconventional resources became to appear economical and companies begin widespread fracking operations.
          && And in 2005, Bush and Cheney back a national energy bill which exempts Fracking from EPA drinking water regulation. (Headwaters 2012)
        & Unconventional resource extraction has only grown since then, with 2013/2014 estimates sitting at 60,000-80,000 active `fracked' wells. (Kelso 2014; Ridlington 2013)
        & This has transformed many rural landscapes in North Dakota, Texas, Pennsylvania and others, which are now dotted with well pads and storage pits.
        & This landscape transformation is due to the monolithic logistical problems facing a drilling and fracking job...
      \end{easylist}
    \subsection{Technical Details}
      \subsubsection{Logistics}
        \begin{easylist}
          \NewList
          \ListProperties(Style*=,Numbers=l,FinalMark={)})
          & These logistics problems start and end with trucks.
          & Thousands of truck trips are necessary for each well to deliver equipment, fluids, chemicals, proppants and to remove wastewater.
          & Drilling activity ramps up so quickly that small, rural areas don't have a chance to improve infrastructure or police presence before the traffic begins to become an issue for both room and safety. (Begos, 2014)
          & Then there's the problem of water storage: The 1-9 million gallons per frack job have to be stored either in large ponds nearby or in countless tanks.
          & After roads and a cleared area (well pad) are built, a drilling rig can be erected and the process can start.
        \end{easylist}
      \subsubsection{Pad Setup \& Drilling}
        \begin{easylist}[enumerate]
          \NewList
          \ListProperties(Style*=,Numbers=l,FinalMark={)})
          & When drilling begins, the first portion of the hole is bored deeply to provide a foundation and prevent surface fluids from entering.
          & Next, the largest section of casing, `conductor casing' is assembled in the borehole and cemented or driven into place.
          & A smaller hole is drilled through the inside of the conductor casing, down below the deepest freshwater aquifer.
          & The `surface' or `freshwater' casing is suspended inside and cemented to the surface.
          & Successively smaller and deeper casings are assembled until the final casing string, known as the `production' casing, is in place.
          & Horizontal drilling, almost always used today, will curve into the target shale formation and continue horizonally to increase reservoir contact.
          & Finally, a `perforating gun,' a long pipe with holes at its ends, is filled with explosives and lowered into the bottom of the production casing which, when detonated, busts holes through the casing and into the target formation.
          & Since 2009, multi-well pads have become common, hosting up to 20 or more wellheads sticking out in different directions from a single surface location. (Sjolander 2011)
        \end{easylist}
      \subsubsection{Fracturing}
        \begin{easylist}
          \NewList
          \ListProperties(Style*=,Numbers=l,FinalMark={)})
          & After drilling is complete, the topmost pipe is capped with a `frac manifold,' which consists of a control valve and numerous piping and communication lines.
          & Also connected to the manifold are a few communication lines which allow the engineers and geolgists to change fluids, pumping rates or pressure as the job proceeds.
          & Powerful pump trucks with 1200-2500 horsepower, 500 gallon/minute and 14,000 PSI are needed to force sand or ceramic beads (called `proppant') down into the formation.
          & A blender truck comes in next, which mixes together the proppant, water and chemicals into a gel.
          & The production casing is divided into isolated zones and pumped full of the fracking fluid, which creates new fractures and widens existing ones to free gas and oil locked inside the reservoir rock. (Sjolander 2011)
        \end{easylist}
      \subsubsection{Draining and Capping}
        \begin{easylist}
          \NewList
          \ListProperties(Style*=,Numbers=l,FinalMark={)})
          & After fracking is complete, all the tanks and trucks are removed and a wellhead is all that remains.
          & Most fluids pumped into the formation are eventually forced back up immediately or over the life of the well, and must be separated from the gas and stored for reuse or deep-injection.
          & It's still unclear what percentage is unaccounted for, as it cannot be distinguished from the natural formation fluids. (Sjolander 2011)
          & During production, recovered gas is sent to `gathering pipelines' which are smaller-diameter lines feeding a larger pipeline collecting gas from a network of wells.
          & Over time, production will decline but the formation pressure still exists, so once the well is no longer economical it will be cemented shut to prevent further methane and flowback from coming to the surface. (Argonne 2013)
          & You can see that there are a lot of nontrivial, hazardous steps to the fracking process and therefore a lot of opportunity for damage.
        \end{easylist}
    \subsection{Environmental Concerns}
        \begin{easylist}
          \NewList
          \ListProperties(Style*=,Numbers=a,Numbers2=l,FinalMark={)})
          & Methane leakage estimates range from 1.4 to 5.8\% over the whole lifecycle of a well. (Argonne 2013)
          & Methane used to be vented to the atmosphere, but tighter regulations and rising gas prices led to less `flaring.' (Wood 2014)
          & Local air pollution has been identified in the forms of volatile organic compounds, hazardous air pollutants, nitrogen oxides and silica dust. (Argonne 2013)
          & Water use is staggering, especially when one considers that fracking often occurs in arid regions: A few hundred thousand gallons are required for drilling and cementing and millions more are needed for the fracking job itself. (Argonne 2013)
          & Contamination of water quality is a significant concern, and flaming tap water almost always comes up.
            && It's often a concern that the fracking fluids pumped down into a target formation will undergo `upward migration' and contaminate aquifers, but our geological knowledge suggests this is unlikely. (Argonne 2013; Flewelling, et al. 2014)
            && Further, research has shown that fugitive gases documented in drinking water wells near natural-gas wells stemmed from inadequate casing and cementing rather than upward migration. (Darrah 2014)
            && A 2011 report found no statistically significant increases in methane levels after drilling near 48 Pennsylvania water wells, and approximately 40 percent failed at least one Safe Drinking Water Act quality standard before fracking ever began. (Boyer 2011)
            && The most obvious and most easily prevented pathway for contamination is dumping or spilling of flowback water, commonly in overflows caused by heavy rain. (Argonne 2013)
            && Deep-injection of fracking fluids has been correlated with earthquakes in places like Ohio, which have never had them, and this had halted some injection operations there.
            && A recent study has concluded that the majority of disposal wells for fracking wastewater do not pose an earthquake hazard. (Argonne 2013)
          & Overall, there are a lot of hazards and pollution being handed to average Americans by those making the money, but since fracking is here to stay, we need to focus our efforts on mitigating long-term damage.
        \end{easylist}

  \section{Conclusion}
    \subsection{Summary}
      \begin{easylist}
        \NewList
        & I briefly how fracking came to be and why it's necessary for unconventional reservoirs.
        & Then I outlined the technology and logistics.
        & And finally I talked about some of the environmental consequences that are cropping up.
      \end{easylist}
    \subsection{Tie-back}
      \begin{easylist}
        \NewList
        & While the famous tap water on fire may, indeed, have been caused by drilling activites, the research has suggested that these are preventable contaminations.
        & Hydrofracking, while technically having been around for 60 years, is still in its infancy in the industrial world.
        & We (America) have a wild frack-first and clean-up-later attitude
        & Hydrofracturing still has remarkably high costs for the environments and the lives of the people in drilling areas, but our nation can weather it, even if we look back on the practice as a blight.
        & Imagine living in China (which has the largest unconventional deposits in the world) where there is zero regulation and fracking fluids are just dumped into the nearby river with no treatment.
      \end{easylist}
    \subsection{Reason to Remember}
      \begin{easylist}
        \NewList
        & If you didn't notice fracking in the news before, you will now and in the future.
        & As the world moves away from coal, we will likely be seeing much more natural gas and, therefore, fracking in the future.
      \end{easylist}

  \newpage
  \begin{center}{\Large \textbf References}
  \end{center}
  \paragraph{}  Argonne National Laboratory. (2013) Hydraulic Fracturing and Shale Gas Production: Technology, Impacts and Regulations. Retrieved from \url{http://www.afdc.energy.gov/uploads/publication/anl_hydraulic_fracturing.pdf}
  \paragraph{}  Begos, K; Fahey, J. (2014) Deadly Side Effect to Fracking Boom. Retrieved from \url{http://bigstory.ap.org/article/ap-impact-deadly-side-effect-fracking-boom-0}
  \paragraph{} Boyer, et al. (2011) The Impact of Marcellus Gas Drilling on Rural Drinking Water Supplies. The Center for Rural Pennsylvania.
  \paragraph{} Darrah, et al. (2014) Noble gases identify the mechanisms of fugitive gas contamination in drinking-water wells overlying the Marcellus and Barnett shales. Proceedings of the National Academy of Sciences of the United States of America vol 111, no 39. (September 30, 2014)
  \paragraph{} Flewelling, S; Sharma, M. (2014) Constraints on Upward Migration of Hydraulic Fracturing Fluid and Brine. Groundwater Vol 52, No.1, January-February 2014, pp 9-19.
  \paragraph{}  Headwaters Magazine. (2012) A History of Hydraulic Fracturing. Retrieved from \url{http://oberlinheadwaters.com/?p=251}
  \paragraph{}  Kelso, M. (2014) Over 1.1 Million Active Oil and Gas Wells in the US. Retrieved from \url{http://www.fractracker.org/2014/03/1-million-wells/}
  \paragraph{}  Ridlington, et al. (2013) Fracking by the Numbers: Key Impacts of Dirty Drilling at the State and National Level. Environment America Research \& Policy Center. Retrieved from \url{http://www.environmentamerica.org/sites/environment/files/reports/EA_FrackingNumbers_scrn.pdf}
  \paragraph{}  Sjolander, et al. (2011) Introduction to Hydrofracturing. Water Facts \#31. Retrieved from \url{http://extension.psu.edu/natural-resources/water/marcellus-shale/hydrofracturing/introduction-to-hydrofracturing}
  \paragraph{}  Stephen, M. (2012) Frack yeah! Canadian Plastics. Retrieved from Grossmont Gateway to Research.
  \paragraph{}  Wood, J. (2014) Pipeline Promises. Engineering \& Technology Vol 9 Issue 8 (September 2014). pp 84-86. Retrieved from Grossmont Gateway to Research.

\end{document}
