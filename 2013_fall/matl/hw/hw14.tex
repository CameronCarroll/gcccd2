% NOTE: Edited \homeworkProblemName command to stop problem number from being in header.
%\begin{homeworkProblem}[Exercise \#\arabic{homeworkProblemCounter}] % Custom section title

\documentclass{article}

\usepackage{fancyhdr} % Required for custom headers
\usepackage{lastpage} % Required to determine the last page for the footer
\usepackage{extramarks} % Required for headers and footers
\usepackage{graphicx} % Required to insert images
\usepackage{lipsum} % Used for inserting dummy 'Lorem ipsum' text into the template
 
% Margins
\topmargin=-0.45in
\evensidemargin=0in
\oddsidemargin=0in
\textwidth=6.5in
\textheight=9.0in
\headsep=0.25in 

\linespread{1.1} % Line spacing

% Set up the header and footer
\pagestyle{fancy}
\lhead{\hmwkAuthorName} % Top left header
\chead{\hmwkClass\ (\hmwkClassInstructor\ \hmwkClassTime): \hmwkTitle} % Top center header
\rhead{\firstxmark} % Top right header
\lfoot{\lastxmark} % Bottom left footer
\cfoot{} % Bottom center footer
\rfoot{Page\ \thepage\ of\ \pageref{LastPage}} % Bottom right footer
\renewcommand\headrulewidth{0.4pt} % Size of the header rule
\renewcommand\footrulewidth{0.4pt} % Size of the footer rule

\setlength\parindent{0pt} % Removes all indentation from paragraphs

%----------------------------------------------------------------------------------------
% DOCUMENT STRUCTURE COMMANDS
% Skip this unless you know what you're doing
%----------------------------------------------------------------------------------------

% Header and footer for when a page split occurs within a problem environment
\newcommand{\enterProblemHeader}[1]{
\nobreak\extramarks{#1}{#1 continued on next page\ldots}\nobreak
\nobreak\extramarks{#1 (continued)}{#1 continued on next page\ldots}\nobreak
}

% Header and footer for when a page split occurs between problem environments
\newcommand{\exitProblemHeader}[1]{
\nobreak\extramarks{#1 (continued)}{#1 continued on next page\ldots}\nobreak
\nobreak\extramarks{#1}{}\nobreak
}

\setcounter{secnumdepth}{0} % Removes default section numbers
\newcounter{homeworkProblemCounter} % Creates a counter to keep track of the number of problems

\newcommand{\homeworkProblemName}{}
\newenvironment{homeworkProblem}[1][Problem \arabic{homeworkProblemCounter}]{ % Makes a new environment called homeworkProblem which takes 1 argument (custom name) but the default is "Problem #"
\stepcounter{homeworkProblemCounter} % Increase counter for number of problems
\renewcommand{\homeworkProblemName}{#1} % Assign \homeworkProblemName the name of the problem
\section{\homeworkProblemName} % Make a section in the document with the custom problem count
% \enterProblemHeader{\homeworkProblemName} % Header and footer within the environment
}{
% \exitProblemHeader{\homeworkProblemName} % Header and footer after the environment
}

\newcommand{\problemAnswer}[1]{ % Defines the problem answer command with the content as the only argument
\noindent\framebox[\columnwidth][c]{\begin{minipage}{0.98\columnwidth}#1\end{minipage}} % Makes the box around the problem answer and puts the content inside
}

\newcommand{\homeworkSectionName}{}
\newenvironment{homeworkSection}[1]{ % New environment for sections within homework problems, takes 1 argument - the name of the section
\renewcommand{\homeworkSectionName}{#1} % Assign \homeworkSectionName to the name of the section from the environment argument
\subsection{\homeworkSectionName} % Make a subsection with the custom name of the subsection
\enterProblemHeader{\homeworkProblemName\ [\homeworkSectionName]} % Header and footer within the environment
}{
\enterProblemHeader{\homeworkProblemName} % Header and footer after the environment
}
   
%----------------------------------------------------------------------------------------
% NAME AND CLASS SECTION
%----------------------------------------------------------------------------------------

\newcommand{\hmwkTitle}{HW Set 14} % Assignment title
\newcommand{\hmwkDueDate}{December 2,\ 2013} % Due date
\newcommand{\hmwkClass}{ENGR\ 260} % Course/class
\newcommand{\hmwkClassTime}{4:00 pm} % Class/lecture time
\newcommand{\hmwkClassInstructor}{McGehee} % Teacher/lecturer
\newcommand{\hmwkAuthorName}{Cameron Carroll} % Your name

%----------------------------------------------------------------------------------------
% TITLE PAGE
%----------------------------------------------------------------------------------------

\title{
\vspace{2in}
\textmd{\textbf{Engineering Materials}}\\
\includegraphics[width=5cm]{cuyam_logo}\\
\textmd{\textbf{\hmwkTitle}}\\
\normalsize\vspace{0.1in}\small{Due\ on\ \hmwkDueDate}\\
\vspace{0.1in}\large{\textit{\hmwkClassInstructor\ \hmwkClassTime}}
\vspace{2in}
}

\author{\textbf{\hmwkAuthorName}}
\date{} % Insert date here if you want it to appear below your name

%----------------------------------------------------------------------------------------

\begin{document}

\maketitle
\newpage

%----------------------------------------------------------------------------------------
%----------------------------------------------------------------------------------------
% Question 1
%----------------------------------------------------------------------------------------

\begin{homeworkProblem}[Problem 10.28] % Custom section title
  What is the carbon concentration of an iron-carbon alloy for which the fraction of total cementite is 0.10? \\[0.2cm]
  \begin{figure}
    \caption{FundMatSciEng, 3rd Edition, Figure 10-28}*
    \centerline{\includegraphics[width=15cm]{fig10-28}}
  \end{figure}
  \problemAnswer{
    Using the lever rule to find the distance pushed towards cementite, \\
    $W_\alpha = \frac{C - C_\alpha}{C_{Fe_3 C} - C_\alpha}$ \\[0.2cm]
    $0.10 = \frac{C - 0.022}{6.70 - 0.022}$ \\[0.2cm]
    $C = 0.10(6.7-0.022) + 0.022$ \\[0.2cm]
    So we find a concentration of 0.69 wt\% carbon.
  }
\end{homeworkProblem}
\clearpage

%----------------------------------------------------------------------------------------
% Question 2
%----------------------------------------------------------------------------------------

\begin{homeworkProblem}[Problem 10.29] % Custom section title
  Using figure 10-3b, derive the lever rule. \\[0.2cm]
  \begin{figure}*
    \caption{FundMatSciEng, 3rd Edition, Figure 10-3}
    \centerline{\includegraphics[width=16cm]{fig10-3}}
  \end{figure}
  \problemAnswer{
    \begin{enumerate}
      \item We use conservation of mass... \\[0.2cm]
      First, we know that $W_\alpha + W_L$, the mass fractions, must be unity. \\[0.2cm]
      Second, we know that the overall composition must remain the same while things shift around inside a system, and so $W_\alpha C_\alpha + W_L C_L = C_0$ \\[0.2cm]
      Solve simultaneously to find $W_L = \frac{C_\alpha - C_0}{C_\alpha - C_L}$ and $W_\alpha = \frac{C_0 - C_L}{C_\alpha - C_L}$
    \end{enumerate}
  }
\end{homeworkProblem}
\clearpage

%----------------------------------------------------------------------------------------
% Question 3
%----------------------------------------------------------------------------------------

\begin{homeworkProblem}[Problem 10.5] % Custom section 
   Cite the phases present and phase compositions for the following alloys:
   \begin{enumerate}
    \item 25 wt\% Pb -- 75 wt\% Mg at $425\,^{\circ}\mathrm{C}$
    \item 55 wt\% Zn -- 45 wt\% Cu at $600\,^{\circ}\mathrm{C}$
    \item 7.6 lbm Cu and 144.4 lbm Zn at $600\,^{\circ}\mathrm{C}$
    \item 4.2 mol Cu and 1.1 mol Ag at $900\,^{\circ}\mathrm{C}$
   \end{enumerate}
   \begin{figure}*
    \caption{FundMatSciEng, 3rd Edition, Figure 10-20}
    \centerline{\includegraphics[width=13cm]{fig10-20}}
  \end{figure}
  \begin{figure}*
    \caption{FundMatSciEng, 3rd Edition, Figure 10-19}
    \centerline{\includegraphics[width=13cm]{fig10-19}}
  \end{figure}
  \begin{figure}*
    \caption{FundMatSciEng, 3rd Edition, Figure 10-7}
    \centerline{\includegraphics[width=13cm]{fig10-7}}
  \end{figure}
  \problemAnswer{
    \begin{enumerate}
      \item Looking at figure 10-20, we see just alpha phase.
      \item From figure 19-19, we find just beta phase.
      \item This is a 95 wt\% zinc alloy, which from figure 10-19 we find just the liquid phase.
      \item This is 268 grams of copper and 217 grams of gold, giving 44.7 wt\% of gold. On figure 10-7, this gives a point along the liquidus line where the alloy is in equilibrium between liquid and alpha/liquid phases. At this point, the liquid is at the 45 wt\% (Au) point while the solids are around 9 wt\% (Au.)
    \end{enumerate}
  }
\end{homeworkProblem}
\clearpage

%----------------------------------------------------------------------------------------
% Question 4
%----------------------------------------------------------------------------------------

\begin{homeworkProblem}[Problem 10-7] % Custom section title
  A 50 wt\% Ni-50, 50 wt\% Cu alloy is slowly cooled from $1400\,^{\circ}\mathrm{C}$ to $1200\,^{\circ}\mathrm{C}$.
  \begin{itemize}
    \item At what temperature does the first solid phase form?
    \item What is the composition of this solid phase?
    \item At what temperature does the liquid solidify?
    \item What is the composition of this last remaining liquid phase?
  \end{itemize}
  \begin{figure}*
    \caption{FundMatSciEng, 3rd Edition, Figure 10-3}
    \centerline{\includegraphics[width=16cm]{fig10-3}}
  \end{figure}
  \problemAnswer{
    \begin{enumerate}
      \item Referring to figure 10-3, which is both below and in a previous problem in this set, we find that the first liquid begins to form at $1250\,^{\circ}\mathrm{C}$.
      \item The composition of the solid beginning to melt is, of course, 50/50. The liquid, moving across the isotherm, is just under 40 wt\% Ni.
      \item The last solids melt away around $1325\,^{\circ}\mathrm{C}$.
      \item The composition of the liquid forming is back to the 50/50 ratio, while the solids dissolving around around 65 wt\% Ni.
    \end{enumerate}
  }
\end{homeworkProblem}
\clearpage

%----------------------------------------------------------------------------------------
% Question 5
%----------------------------------------------------------------------------------------

\begin{homeworkProblem}[Problem 10.8] % Custom section title
  Determine relative amounts, in terms of mass fractions, for the phases and termatures given in problem 10.5. \\[0.2cm]
  \problemAnswer{
    \begin{enumerate}
      \item Ranges from the Quasi War (Undeclared sea war against France) to the Boxer Rebellion (America and Friends against China) / Phillipine-American War (Filipino revolutionaries trying to secure independence from United States, who gained the area from the Spanish-American War.)
      \item American military efforts in the 1800's include numerous wars and battles against native tribes across the country, as well as the `big name' wars. \\[0.2cm]
      \textbf{(Interesting Examples)}
      \item Aegan Anti-Piracy Operations: Around 1825, the Aegean had become a haven for piracy and privateers; After a few American ships were sunk, President Monroe deployed sloops and schooners as part of the Mediterranian Squadron. The Squadron has a few skirmishes and victories; After three years the operation is considered a success. (UK, Russia and France had also deployed their own anti-piracy efforts to the Aegean area.) (There were also the West Indies Anti-Piracy Ops, and the African Anti-Slavery Ops.)
      \item Korean Expedition, or `The Shinmiyangyo:' The United States came to Korea on a diplomatic mission to establish trade and political relations; Korean Shore batteries attacked American warships, and the Americans respond with a `punitive expedition.' The expedition only results in continued Korean isolationism and American withdrawl.
    \end{enumerate}
  }
\end{homeworkProblem}
\clearpage


\end{document}

