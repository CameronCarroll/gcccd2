\documentclass[13pt]{article}


\usepackage{fourier}
\usepackage{sectsty} % Allows customizing section commands
\allsectionsfont{\centering \normalfont\scshape} % Make all sections centered, the default font and small caps
\usepackage{graphicx}
\usepackage{mhchem}
\usepackage{tabularx}
\usepackage{pbox}
\usepackage{listings}
\lstset{language=Matlab}
\usepackage{caption}
\usepackage{color}
\usepackage{xcolor}
\usepackage{fullpage}
\linespread{2}

\newcommand{\HRule}{\rule{\linewidth}{0.5mm}}


\begin{document}

\begin{titlepage}
\begin{center}

\textsc{\Large Early American History}\\[0.5cm]
\includegraphics[width=5cm]{cuyam_logo}

\HRule \\[0.4cm]
{ \LARGE \bfseries Film Review 1: Nightmare at Jamestown}\\[0.5cm]

\HRule \\[1.5cm]

\begin{minipage}{0.4\textwidth}
\begin{flushleft} \large
\emph{Author:}\\
Cameron \textsc{Carroll}
\end{flushleft}
\end{minipage}
\begin{minipage}{0.4\textwidth}
\begin{flushright} \large
\emph{Instructor \& Class:}\\
Judy \textsc{Campbell} - (HIST 108)
\end{flushright}
\end{minipage}

\vfill

{\large \today}

\end{center}
\end{titlepage}
\pagebreak

\paragraph{} `National Geographic: Nightmare in Jamestown' is a brief look into the attitudes and lifestyles of early Virginia colonists as well as their native counterparts. While not a fictionalization, it nevertheless tells the story of Jamestown's efforts to survive and how their relationship with the Indians changed.

\paragraph{(Question 1)} As a documentary, this film is very straightforward in describing the attitudes and situations of the Colonists and Natives. In the beginning of colonization, the Natives are already on edge: Their leader had been presented with a premonition in which he is overthrown, to which he responded by eliminating a rival tribe. The Jamestown Colonists were landing in very hostile land with very grumpy inhabitants. They were part of the Virginia company and consisted largely of gentlemen and artisans \& were primarily interested in obtaining riches: two years were wasted panning for gold and finding nothing. At this point the Colonists are in a foreign and hostile land, surrounded by hostiles and without proper laborers to create a functioning society. The attitudes of the Colonists surely fell into despair as 70\% of the population succumbs to hunger, disease and the other many terrifying facets of the New World. Here are the beginnings of Virginia's reputation as a `death trap.' The colony sends the message, however, that they're here to stay with their high walls around the town and the constant stream of useless reinforcements. \\
John Smith brings necessary change: His dealings with the native tribes surely saved Jamestown from complete destruction (by securing foodstuff) and brought some semblance of discipline to the Jamestown people. The tribes' help, especially during the Starving Time, brings the two factions together, at least for a time. Jamestown, under John Smith, begins to get its feet moving finally. \\
Now the Colonists have some `useful' folks in their ranks: They can farm and everything. It's becoming clear at this point that Jamestown isn't just a bunch of annoying white guys who can't feed themselves anymore: They've become an active threat to the Indians. Tobacco is discovered but requires English settlers to force Natives further inland for their land. The Colonists' attitudes shift to this being their home, while the natives' attitudes shift to realizing that they're getting kicked out and this must stop!

\newpage

\paragraph{(Question 2)} I couldn't help but draw parallels between the `colonization' and industrialization that we continue today and the Jamestown events. Thanks to our specialization economy we become experts in such fields as History or Engineering: We sit on the shoulders of giants sitting on something even larger. It's sort of remarkable that while we have these thousands of years of development behind us, we only pick a small sector of the knowledge. Almost all of us would be utterly useless in a disaster: We're all artisans. If, for whatever reason, the world needs farmers and tinkerers and craftsmen again, we're in trouble. \\
The Natives also have a contemporary equivalent: Natives. Indigenous people deep in Africa, South America and Oceania have incredible culture and societies that I feel could be crushed at any moment. Sure, there are often some protections put in place to keep these people safe from our world, but they can't stay segregated forever. Even the most agressive tribes, such as one off the coast of India where a film crew was speared (also the native who killed them proceeded to gloat,) have been increasingly accepting of the outside world. That particular tribe began accepting airdropped gifts, even if no-one has dared to go back to the island.

\paragraph{} Jamestown is an excellent example of how the English tragically underestimated the wilds: It takes them years to figure out how to make a functioning society here, just as it took years for the military to realize they can't fight in proper European style. It also is typical of Native responses to these white invaders: Initially hostility upon first meeting, eventually neutrality and trading, then finally hostility again as the two societies realize they can't coexist.

\end{document}
