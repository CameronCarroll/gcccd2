\documentclass{article}
\usepackage{textcomp}
\begin{document}

\begin{center}
  {\small{} Early American History Lecture Notes} \\[0.6cm]
  {\small{} Cameron Carroll -- November 13, 2013} \\[0.6cm]
  {\small{} Lecturer: Judy Campbell, Cuyamaca College}\\[1cm]
  {\small{} American Civil War}\\[1cm]
\end{center}

\tableofcontents
\newpage

\section{Overview}
  The Civil War totally changes the way we fight wars across the whole world.

  \subsection{Northern Strategy}
    \begin{itemize}
      \item Wanted to gain control of Missisippi river, cutting off Lousisiana and Texas from helping.
      \item Also wanted to cut the southeast part of the South in half in order to cut their supply lines.
      \item Also wanted to put in a naval blockade because the south relied on trading cotton for money. The idea was to block anything going in or out of the confederacy.
      \item And their fourth goal was to capture the southern capital, Richmond, Virginia.
    \end{itemize}
  \subsection{Southern Strategy}
    \begin{itemize}
      \item South was at a disadvantage given their lack of industry and land power.
      \item Just wanted to hold out long enough to make the north give up, make the war as costly as possible and hope they'll just let them be their own country. That wasn't a super far-fetched idea either, they came close.
      \item Wanted to capture DC, capital of the Union.
      \item They depended on cotton trade for money, so they hoped to draw in England and France's help.
    \end{itemize}
  \subsection{Main Arenas}
    \begin{itemize}
      \item There were 2400 battles spread across these three main arenas.
      \item Virginia, which hosted the vast majority of the battles. Note that DC and Richmond were very close to each other.
      \item Along the Missisippi
      \item Tennessee (Note that this and the preceeding arena are west of the Appalachians.)
    \end{itemize}

\section{Battles}
  \subsection{First Battle of Bull Run}
    \begin{itemize}
      \item Takes place only 30 miles from the capital.
      \item Known to the Union as Manassas Junction.
      \item Union has 35,000 soldiers, mostly raw recruits.
      \item The South marches from Richmond to DC intending to take the capital.
      \item Lincoln sends the first of his million generals, McDowell, to head the Union troops. (These Union troops east of the Appalachians are known as the army of the Patomek. It's this half the the Union efforts that has such trouble finding a good general.)
      \item The two armies meet at Bull Run. This is the first battle of the war, and nothing exciting like that had happened for a while, so the gentlepeople of DC decide to have a picnic. They sit and look out over the battlefield to watch.
      \item Thomas Jackson, the number two Confederate, holds off a Union assault with his brigade and gleans his nickname of `Stonewall Jackson.'
      \item The South ultimately win this battle and begin their tradition of the `rebel yell:' thousands of angry soldiers just screaming their heads off while charging. To say this unnerved the Union troops would be an understatement.
      \item The Union forces retreat and run right past all of those bourgeois folks having a picnic, who also run back to DC.
      \item Perhaps the South could have invaded DC at this point, but they decided to have the left-behind picnic intead.
      \item In this battle the two sides realized they were wearing the same uniform and shooting themselves, so the Confederacy adopts a grey uniform and also a new flag.
    \end{itemize}
  \subsection{`Unconditional Surrender' Grant}
    \begin{itemize}
      \item Lincoln isn't impressed with McDowell, so he replaces him with McClellan who is rather good at training his troops but less decisive in a battle. He raises the army up to 100,000, but really doesn't seem to want to fight.
      \item West of the Appalachians, however, Lincoln puts in Ulysses Grant who is ready to take charge and beat on the South. He wastes no time and takes forts Donelson and Henry in 4 days. Here he finds his nickname of `Unconditional Surrender' Grant.
      \item Grant ships 14,000 Confederate troops off to prisoner war camp for the duration of the war.
    \end{itemize}
  \subsection{The Monitor and the Merrimack}
    \begin{itemize}
      \item The Monitor and Merrimack were ironclad ships after which wooden ships were never seen in war again.
      \item The South developed the Merrimack, a wooden ship which had caught fire and was put into harbor. Later it was encased in iron, as was the Northern Monitor. Out of the top of the Merrimack are 10 heavy cannon, shooting 150 lb cannonballs. The Monitor, on the other hand, had a revolving turret with a 360\textdegree field of attack.
      \item The Merrimack decides to attack DC, and goes sinking Union ships one after the other as it travels. During these attacks the captain of the Merrimack kills his own brother.
      \item The Monitor eventually engages the Merrimack and the two fight for six hours at point-blank range before mutually withdrawing.
    \end{itemize}
  \subsection{West of the Appalachians}
    \subsubsection{Shiloh}
      \begin{itemize}
        \item The Union marches across most of Tennessee and decide to relax for a minute before the battles that lie ahead at Shiloh. The Confederacy, however, is much closer than they thought and stage an extremely bloody sneak attack which demonstratess a new standard in the slaughter of war. (24,000 casualties.)
      \end{itemize}
    \subsubsection{Memphis \& New Orleans}
      \begin{itemize}
        \item Union moves on Memphis to try to the the Missisippi river, still led by Grant. At this time, Admiral Farragut attacks New Orleans with his wooden boats and successully takes it, gaining control of the mouthpiece to the Missisippi.
      \end{itemize}
  \subsection{Peninsular Campaign: Invading the North}
    \begin{itemize}
      \item Part of Lee's strategy is to move up through the Virginian peninsular area to try to take DC. He's a great strategist and despite having a number disadvantage he still manages to win.
      \item Lee heads further north and nears DC. They fight a second battle at Bull Run in August of 1862. Lee and Jackson lead against Pope, who gives up and gets chased back to Washington. (And Lincoln loses another general.)
      \item Antetium/Invasion of Maryland:
        \begin{itemize}
          \item Lee has 150,000 men, about the most he ever leads, and decides to invade the North.
          \item He heads for Antetium in September but realizes that they can't transport enough food for the Confederate army. They rely on corn from Maryland farms instead, but this may have been a detriment... Maryland is a border state and if it were to side with the South they would have a great position with large population and factories. The invasion doesn't exactly go as planned, though: They see the Confederate troops without any shoes or food or anything and decide not to join with them.
          \item The Union army is waiting for them at Antetium and force them to retreat back into Virginia, with another 25,000 American casualties making this the bloodiest day of the war.
        \end{itemize}
    \end{itemize}
  \subsection{Digression: Lincoln Looks for his General}
    \begin{itemize}
      \item Lincoln is upset with his generals for being crappy despite winning at Antetium. So he sends in Burnside, the origin of the term sideburns, who isn't so sure about being a general. But Lincoln really needs someone to attack Fredricksburg.
      \item Unfortunately Lee gets there first.
      \item It's December, snowy and rainy and cold and generally shitty. Burnside needs to cross a river but finds they have to wait for pontoons. They cross the icy water, go over a stone wall, through a drainage ditch, through a large expanse of flat land only to find the Confederates on a bluff above shooting down. He gives up on being general.
      \item Lincoln picks `Fighting Joe' Hooker, who had an insatiable appetite for nocturnal female companionship, starting the term hooker's girls. After the Civil War, most soldiers went West to kill Indians and enterprising females went along, referred to as hookers.
    \end{itemize}
  \subsection{Digression: Emancipation Proclamation of January 1863}
    \begin{itemize}
      \item The cause of the war was states rights: Did the states have more authority for themselves than the federal government? The issue of the time, of course, was slavery.
      \item Lincoln wanted the support of England and France but he couldn't have it; The South had the higher moral cause even though their cause was slavery, as they just wanted to retain self-rule and way-of-life. Lincoln figured he would have more success if he convinced England and France that they're trying to eliminate slavery.
      \item Lincoln issues the Emancipation Proclamation which frees the slaves in the states of rebellion. He successfully convinces them that the intent to to eliminate slavery, so they come to help.
      \item The Proclamation was timed right after Antetium and a Northern victory, and the wording was very careful to make the Union look good.
    \end{itemize}
  Halfway through the war, the North is doing fine making ironclads and weapons and food. The only problem was that when Lincoln said he would free the slaves, northern troops said they didn't want to -- They would fight to keep the Union together but they wouldn't die for a slave's life. In the South, on the other hand, they're doing awful... inflation is rampant and the Union blockade brings shortages and high prices on things like sugar, coffee and pants.
  \subsection{Chancellorsville}
    \begin{itemize}
      \item In Chancellorsville, Lee and Hooker are battling, the latter of whom advances through the wilderness. (In the peninsular area in Virginia, you can't see anything for the solid wall of trees.)
      \item Lee was smart because he mostly stayed in the trees, making Union cavalry and cannon useless.
      \item One night, when it was super foggy and as Jackson was returning to the Confederate camp, some of his own soldiers on the side of the road shot him in his arm, which had to be amputated. But then he got pneumonia and dies. Jackson was so revered, though, that they took his arm and buried it.
      \item Despite losing the homie, Lee still win at Chancellorsville. He decides to invade the north one more time afterwards, feeling good, heading toward Geddysburg.
    \end{itemize}
\end{document}