\documentclass{article}
\begin{document}
\begin{center}
{\small{} Early American History Lecture Notes} \\[0.6cm]
{\small{} Cameron Carroll -- October 30, 2013} \\[0.6cm]
{\small{} Lecturer: Judy Campbell, Cuyamaca College}\\[1cm]
{\small{} Manifest Destiny}\\[1cm]
\end{center}

\tableofcontents
\newpage

\paragraph{} Manifest destiny declares that it's our god-given right to expand and rule from sea to shining sea. We're gonna share American liberty and freedom with everybody! (Possibly by force.)

\section{Expansionist Presidents, and People}
  \begin{itemize}
    \item Jefferson buys the Louisiana Purchase.
    \item Jackson goes into Florida and Texas.
    \item Polk goes into Mexican Cession. Was all about `54\`40 or fight' (Wanted the latitude line all the way across the top of the country.)
    \item Modern American actions are still expansionist.
    \item When the coast became crowded, the people moved North \& South to fill in the coastline. Of course, their ideas of crowded were not ours. But when the North \& South filled up, people moved West and set up trading posts \& forts all along the Appalachian mountains.
    \item Only took 90 years to go from French \& Indian war until the Gadsden Purchase, moving from 13 colonies to occupying the entire United States.
  \end{itemize}

\section{1600's Trailblazers}
  \begin{itemize}
    \item John Lederer: First guy to reach the top of the Appalachian Mountains
    \item Daniel Boone: Was a quaker, drove a wagon in French \& Indian war. When the war was over, Boone is asked to go over the mountains and see what's going on over there. He spends several months exploring, seeing if the land lived up to the legends. Starts settlement of Boonesboro, one of the first settlements along the wilderness road in Kentucky. Boone leads the first white families and the first white ladies into Kentucky. 
  \end{itemize}

\section{Northwest Territory}
  \begin{itemize}
    \item Is the next chunk of land to be settled.
    \item In 1787, the Northwest Ordinance establishes how areas become states, starting with Ohio:
      \begin{itemize}
        \item Once the population of an area reaches 5000 people it becomes a territory.
        \item The US government then appoints one governor, three judges, and a representative assembly to run that territory.
        \item When the population reaches 60,000, the territory can apply to become a state with a representative in Congress.
      \end{itemize}
    \item Once the Indians are gone, people start pouring into the West. In Ohio alone there were approximately 1000 miles of trails. In 1790, 100,000 people pour into Ohio. Ten years later, 400,000 people resided in the Ohio.
  \end{itemize}

\section{Louisiana Purchase}
  \begin{itemize}
    \item Jefferson acquires a large area of land for practically nothing from France, since Napoleon needed war funds really badly.
    \item The acquisition was in line with Jefferson's ideals, though, since he wanted an agrarian nation and it would provide ample land for growing tobacco and cotton, which deplete the soil.
    \item In 1804-1806, Lewis and Clark are sent out... the Corps of Discovery. Lewis was Jefferson's private secretary, an army officer from Virginia trained in folk medicine. They had 50,000 `thunderclappers' which were supposed to cure whatever you had. They were really just a diuertic.
  \end{itemize}
\end{document}