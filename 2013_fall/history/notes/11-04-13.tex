\documentclass{article}
\begin{document}

\begin{center}
    {\small{} Early American History Lecture Notes} \\[0.6cm]
    {\small{} Cameron Carroll -- November 4, 2013} \\[0.6cm]
    {\small{} Lecturer: Judy Campbell, Cuyamaca College}\\[1cm]
    {\small{} Settling the West I}\\[1cm]
\end{center}
  
\tableofcontents
\newpage

\section{Mountain Men}
  \begin{itemize}
    \item Mountain men move into the west; They look like Indians, mix and mingle with them, aren't a big threat.
    \item They marry Indian women, easing the transition of whites into the area. They create paths, later followed and became trails, later became roads, and later became highways.
    \item All major highways through to the west were originally paths created by mountain men.
    \item Mountain men end up corrupting the Indians; They (Indians) have a genetic disposition to alcoholism.
  \end{itemize}
\section{Settling Texas}
  \begin{itemize}
    \item By 1822, we have the Texas area, a state of Mexico.
    \item Southern Americans were looking at this area because they needed more land for their crops. They begin encroaching a bit. So the United States offers to buy the area from Mexico who refuse.
    \item But Mexico allow Moses \& Austin to bring in 300 families to settle this area, under some conditions:
      \begin{itemize}
        \item Mexico still owns the area.
        \item The settlers had to become Catholic.
        \item There was also no slavery allowed, which is a problem because Southerners wanted to settle this area.
        \item And they also had to abide by the laws of the Mexican government.
      \end{itemize}
    \item Mexico thought this was a good idea; The area was overrun by Indians and Mexico had basically no hold on it. So they wanted the settlers to move in to act as a buffer between themselves and the Indians.
    \item 300 families turned into 35,000 gringoes moving into the area, who were bringing in their slaves.
    \item Many of the kinds of people who moved here were debtors escaping the country and therefore their debts. This brings a lot of `riff-raff' into the area.
  \end{itemize}
\section{From The Alamo til San Jacinto}
  \begin{itemize}
    \item Mexico tightens up immigration and raises taxes on anyone living in the area. Austin says we're going to create the Republic of Mexico. Santa Ana says `Hey you guys you can\'t just take that.' So he marches on them and they hold out at the Alamo.
    \item A few hundred Texans hold out at the Alamo, while Santa Ana gets 4,000 Mexicans.
    \item Amazingly, the Texans managed to hold off the Mexican army for 13 days. 
    \item Davey Crocket with his Tennessean volunteers helped. This guy served in Congress, was learned, was also an adventurer and mountain man. Ends up dying at the Alamo.
    \item After 13 days of fighting, the Americans put up a flag of surrender. The Mexicans took all of the remaining soldiers and shot them, then burned their bodies.
    \item Mexican army kept on going to Goliad where almost twice as many Americans resisted the 4,000-strong Mexican army. 
    \item These contingents of Americans give the American army enough time to get down there to fight.
    \item When the American army finally arrived they cried ` remember the Alamo!.' 
    \item In 1837, Mexican army is defeated at San Jacinto. 
  \end{itemize}
\section{Formation of Texas and the Mexican-American War}
  \begin{itemize}
    \item Texas becomes independent, and a republic. Houston wanted it to become a state, and he was made president of the republic.
    \item Texas, a slave state, was an independent country for a little while.
    \item In 1846-1848, we have the Mexican war. There was a dispute over the boundary of Texas. Mexico said the boundary would be some other river that wasn't the Rio Grande. There were several families living in the disputed area; The Mexican army killed about 65 Americans living in this area.
    \item President Polk, an expansionist, was picking a fight and was given the impetus by the aforementioned murders.
    \item General Winfield Scott was sent all the way down to Mexico City to fight. He beats them.
    \item Zachary Taylor, later becomes president, attacks at Buena Vista in Northern Mexico.
    \item Even though the Americans had learned all about guerilla warfare, this war is fought in traditional European style.
    \item Mountain Men carry dispatches between the American Army, knowing the area so well.
    \item The US beats Mexico, at which point we could have taken the whole thing and opt not to.
    \item All the American generals fighting in the Mexican War later fight each other in the Civil War.
    \item Treaty of Guadalupe-Hildago ends the war, and America gets the Mexican Cession. We actually purchased the land (for \$15 million.) to kinda clear our national conscience. It cost \$100 million to fight the war. It also cost 13,000 lives and cost Mexico 4 times that many. Only 2\% of this 13,000 died from battle, the rest from dysentery.
  \end{itemize}
  \subsection{Aftermath}
    \begin{itemize}
      \item We wanted Manifest Destiny, Mexico really didn't have strong control over this land, it was kinda meant to be. 
      \item John Charles Fremont, the ` pathfinder' and Kit Carson were sent exploring. They leave from St Joseph, next to Independence, and go past SLC to Sacramento. They establish the Pony Express Route. Their accounts of the West brings tons of people into the area. 
      \item In California, which was part of Mexican Cession, we have the Bear Flag Revolt: The people in California staged a separate revolt against the Mexican government. (California was by far the most populated area.) They officially, then, said Mexico is out of here. 
      \item Wagons coming across the country weren't in sporadic groups: They left Independence one after the other. Thousands of people poured into the West, huge caravans of covered wagons. There were so many wagons that they cut grooves through solid rock.
    \end{itemize}
\end{document}
