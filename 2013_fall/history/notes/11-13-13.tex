\documentclass{article}
\begin{document}

\begin{center}
  {\small{} Early American History Lecture Notes} \\[0.6cm]
  {\small{} Cameron Carroll -- November 13, 2013} \\[0.6cm]
  {\small{} Lecturer: Judy Campbell, Cuyamaca College}\\[1cm]
  {\small{} Industrial Revolution}\\[1cm]
\end{center}

\tableofcontents
\newpage

\section{Transportation}
  \subsection{Canals}
    \begin{itemize}
      \item Allowed for significantly reduced shipping time and shipping costs.
      \item Moved goods from the eastern, industrial area into the interior of the country.
      \item Allows factory-made goods to be moved into western states. (By western we refer to anything not eastern seaboard.) This allows for access to some familiar goods and comforts.
      \item Interior of country, the `West,' move agricultural and natural goods back East. They don't have the big farms necessary in New England.
      \item In exchange, the interior states send agricultural goods and natural resources to the manufacturing Northeast.
      \item Food becomes cheaper in New England than ever before.
      \item Goods become cheaper in the West than ever before.
      \item Lots of towns sprout up along the rivers and railroad lines. They later become our big towns and cities, of course. For example: Some people stop along the Oregon trail along with a couple others and open up a trading post, or whatever.
    \end{itemize}

  \subsection{Railroads}
    \begin{itemize}
      \item Steam Engine, originally invented in England, is put on boats to move them along canals.
      \item George Stevenson (England) and Oliver Evans (America) think about putting the steam engine on a track at about the same time. So they stick a steam engine on a locomotive.
      \item Peter Cooper comes along and adds a wagon to the back of the steam engine and established the train.
      \item In 1828 there were only about 30 miles of track, from Washington to Baltimore.
      \item By the beginning of Civil War that had grown to 30,000 miles of track.
      \item The rail lines weren't standardized, however, and there were 4 or 5 different widths of track. This meant sometimes changing trains and tracks.
      \item In the North there is very heavy rail development, while in the South they don't build so much rail and instead rely on roads and long, flat rivers. In the Civil War, then, the North are able to move military supplies easily while the South had to lug things around by wagon. 
      \item Most of the track was laid with government aid, since we had a `Big' government. 
      \item Railroad is the country's first really big business.
      \item At the same time, coal is found in Pennsylvania which can be used to fuel the steam engines and factories.
      \item Coal becomes as valuable as the gold found in California because it's used to fuel everything in the industrial revolution.
      \item Railroads didn't connect us coast-to-coast until after the Civil War. We started building it, starting from each side intending to meet in the middle, but were interrupted by the Civil War.
      \item Rail companies got a few million acres very cheaply to build their railways on. They also sold small parcels of it to people wanting to settle towns along the way. These towns helped the railroads, allowed them to arbitrate land sales, and gave people a place to settle.
      \item Railroads from the East coast to the West had 25\% Irish laborers. Immigrants also went to work in the coal mines. On the West coast there's a huge influx of Chinese due to overcrowding in China. Due to language barrier and low education these immigrant were basically only good for manual labor.
      \item The Chinese built the railroad from the West coast to the East.
      \item They meet in Utah at Promotory point. They drive some stakes into the ground to commemorate the connection.
      \item Those working West were crossing flat land. On the other hand those working East were trying to get through the Rocky Mountains and many were lost trying to dynamite their way through, among other ways.
      \item Fun fact: None of the dignitaries were strong enough to drive the spikes into the ground, so they got a little chinese guy to do it real quick before sending him away. 
      \item Connection helps unify the nation, pulling a huge expanse together. It helps the economy, and also the military preparedness just in time for the civil war. 
    \end{itemize}

\section{Other Developments}
  \begin{itemize}
    \item William Harnden starts American Express, a company shipping packages, around 1830.
    \item Henry Wells starts Wells Fargo; Builds a mail delivering service.
    \item The government steps in and declares that they'll handle the postal system but they overcharge and nobody wants to pay. So they drop the fee significantly to outdo the private companies.
    \item New York Times and other newspapers established, could be purchased for a penny.
  \end{itemize}

\section{Sectionalism}
  \begin{itemize}
    \item Three sections develop throughout the country and essentially leads to the civil war.
    \item New England area had shipping, trading and manufacturing. Had harbors, mountains, coal.
    \item Southwest stick to agriculture with their flat land. Cotton grows here better than anywhere anyone knew of. Slaves were essential to their wealth. (Slavery was outlawed in New England at the time but very necessary in the South because their crops were so labor intensive.)
    \item Great Plains area, develops last, doesn't have a lot of rivers, or very good soil, or very many trees. It's much more difficult to make a living here. A couple things help people settle this area, though:
      \begin{itemize}
        \item Steel Plow: Trying to take a plow against desert granite doesn't cut it the way it did in fertile soil. They take steel, invented in England by Bessemer, and create a plow out of it which allows them to work the land.
        \item Windmill: Allows people to get water to irrigate the land.
        \item Barbed Wire: Becomes essential for the cattle ranchers.
      \end{itemize}
  \end{itemize}

\section{Fate of the Indians}
  \begin{itemize}
    \item First West was settled by Indians, then Mountain Men, then Sod Busters along Oregon Trail. Then, after the Civil War, there are cow ranches and cowboys.
    \item Indians were subjected to technology far beyond their time and so they lose and are all put onto reservations. 
    \item Those on reservations are subject to federal law but are also excused from some. They're allowed to gamble and have fireworks, for example. 
    \item When put on a reservation all Indians were given farming goods: A shovel, rake, seed and a cow. 
    \item Not knowing what you do with a cow, however, they send all of them running all at once and have one last great `buffalo' hunt.
    \item Buffalo end up getting wiped out by Buffalo Hunters, people hired by the rail companies specifically to hunt them down. They wanted to protect their rail lines, because hundreds or thousands of buffalo crossing a track would just destroy it. 
    \item The Indians had their staple taken away; They used it for food and clothing and everything. So they basically had to go to the reservations but they didn't know how to be farmers.
  \end{itemize}

\end{document}
