\documentclass{article}
\begin{document}

  \begin{center}
    {\small{} Early American History Lecture Notes} \\[0.6cm]
    {\small{} Cameron Carroll -- September 18, 2013} \\[0.6cm]
    {\small{} Lecturer: Judy Campbell, Cuyamaca College}\\[1cm]
    {\small{} Strife in the Colonies}\\[1cm]
  \end{center}
  
  \tableofcontents
  \newpage

  \section{Strife in the Colonies}
    \subsection{Repealing the Stamp Act}
      \begin{itemize}
        \item Britain tries to impress taxes upon Colonies who begin smuggling to avoid them.
        \item Embargos against Britain also put in place to boycott.
        \item Burnings in effigy happen: Colonists begin to demonstrate their displeasure.
        \item Colonies begin to organize `sons of liberty.' They were seen as violent revolutionaries, but were hardly as bad as we would suspect.
        \item Like-minds got together to complain about taxes.
        \item Colonies got together again (previously wanting to keep to themselves) to have Stamp Act Congress. They decide to boycott all English goods, which got the King to repeal the act.
      \end{itemize}

    \subsection{Virginia Resolutions}
      \begin{itemize}
        \item Were another resistance attempt, held at House of Burgesses.
        \item Patrick Henry said only colonists should be able to tax colonists\ldots No taxation without representation.
        \item These are sort of a backlash against Townshend duties.
        \item Bribery at ports is totally rampant with colonists trying to get around these restrictions.
        \item Colonists didn't have a problem with being subordinate to King, but more with being constantly taxed without representation.
      \end{itemize}

    \subsection{Boston Massacre}
      \begin{itemize}
        \item In March of 1770, some Colonists get uppity and form a mob.
        \item Somebody yelled `Fire!' and we get the Boston Massacre.
        \item In reality only 5 people died.
        \item The first Patriot (Colonist to die) was an Indian/black slave named Crispus Attucks.
        \item John Adams, lawyer, got all the British soldiers acquitted.
        \item Afterwards, Lord North (Dude in charge) repeals all acts except for tax on tea.
        \item He kept the tea tax , though, to give the message that the king has the right to tax his subjects.
        \item The moderates were appeased, but not the extremists. 
      \end{itemize}
  

    \subsection{Boston Tea Party \& Coercive / Intolerable Acts}
      \begin{itemize}
        \item Two years after Boston Massacre, Colonists form the Committees of Correspondence, which is how Adams spreads written word of the massacre.
        \item Boston Tea Party of December 1773:
          \begin{itemize}
            \item Some irate guys board three ships carrying tea in Boston Harbor and dump 300 cases, valued at \$75,000 (\$1,000,000).
            \item This was a rebellion against the forced trade of very expensive tea coming in.
            \item King gets pissed, passes more acts. Colonists call these `intolerable acts' while English call them `coercive.'
            \item These acts close the entire port of Boston.
            \item Also, King says Massachusetts may have one government meeting per year, in an area which has had democratic meetings for 100 years.
            \item Also, sends more troops into Massachusetts and said they (soldiers) are not subject to the law.
          \end{itemize}
      \end{itemize}

    \subsection{Continential Congress}
      \begin{itemize}
        \item Coercive acts result in the `Minutemen,' people who promised to be ready at a `minutes notice' to fight for the Colonies.
        \item Also results in colonies forming the first Continential Congress, meeting in Philadelphia in 1774.
        \item Patrick Henry, George Washington (Virginia)... Sam \& John Adams... John Jay... John Dickenson were there.
        \item These guys had no legal right to meet, were revolutionaries against the King.
        \item Were mostly conservative, wealthy and concerned with liberties.
        \item All states sent representatives except for Georgia, which had stronger loyalties to England and didn't participate much in the revolution.
        \item They made Declaration of Rights at the Congress, intended to be sent to the king saying colonists deserve liberty and rights.
      \end{itemize}

    \subsection{Preparing for a Fight}
      \begin{itemize}
        \item Colonists started to collecting materials for war, just in case.
        \item At this time, Ben Franklin is in London trying to get compromises but gets blown off and humiliated by Parliament. Franklin returns home and gives up on the concept of a harmonic, subordinate existence with Britain.
        \item By 1775, British send General Gage who was supposed to make an armed camp out of Boston.
        \item When they find out about Gage in the House of Burgesses, Patrick Henry stands up and gives his famous statement including `give me liberty or give me death.' The king's representative immediately disbanded House of Burgesses meeting at that.
        \item Weapons get stockpiled in Concord, Massachusetts.
        \item Two groups of people: Patriots (Planters, middle-class guys) and Loyalists [Loyal to England] (Frontiersmen, quakers, NYC merchants.)
        \item Also in Parliament, some people argued against fighting and in favor of compromise. Others of course were offended at the colonial rebellion. 
      \end{itemize}

    \subsection{Lexington \& Concord}
      \begin{itemize}
        \item Gage didn't want a battle, however, and decided to just go raid their magazine (armory.)
        \item His wife, however, was a spy for the Colonists and warns them.
        \item  Gage also knows that in Lexington, John Hancock (President of first Continental Congress) is hiding with John Adams. So gage sends his soldiers there and then to go to Concord next.
        \item Two guys Revere and Dawes found out that the soldiers were coming and rode out to warn the minutemen that they needed to go to Lexington. Dude was yelling `the regulars are out' and not `the british are coming.' Adams/Hancock get away and everything.
        \item Lexington:
          \begin{itemize}
            \item Soldiers (grumpy and tea-less after walking all morning) show up (to Lexington) and find a line of Minutemen in their way.
            \item Somebody fires (nobody knows who) which leads both sides to shoot each other.
            \item `The Shot Heard Round The World' was that stray bullet. It's called this because England (and other superpowers) had colonies all around the world. Word spreads about this revolution and begins a number of other revolutions against the mother country.
            Colonists get smushed, Lobsterbacks burn down a couple houses and go on to Concord.
          \end{itemize}
        \item Concord:
          \begin{itemize}
            \item At Concord, Minutemen gather on one side of a bridge with lobsterbacks on the other.
            \item Not quite dumb enough to stand in a line and get slaughtered, the minutement hold the bridge and prevent British soldiers from crossing and force them to walk back to Boston.
            \item Colonists continue guerilla warfare, pick off 300 soldiers while they're marching back.
          \end{itemize}
        \item Committees of Correspondence spread a call to arms, arguing that they're not skirmishes but full battles.
        \item Fun fact: Women helped in the war bringing supplies, sometimes firing cannons even!
      \end{itemize}

      \subsection{Second Continential Congress}
        \begin{itemize}
          \item John Hancock is still president.
          \item Most colonists are still down with the monarchy, but not these guys.
          \item They sent a second petition to the king: The Olive Branch petition. This says that they still don't want to fight but just to be treated like Englishmen.
          \item It doesn't work so they take George Washington and create the Continental Army. He gets 10 units of `expert riflemen.'
          \item Ethan Allen \& Green Mountain Boys attack Fort Ticonderoga to get the Continental Army arms. They succeed and get like 100 cannons, get them over the Appalachian mountains, float them across rivers, and successfully deliver all of them.
        \end{itemize}

  \section{Comparison of War Preparations}
    \begin{itemize}
      \item Britain has 12m people while Colonies have about 3m.
      \item England has a highly developed industry versus practically none.
      \item England at this time is the richest country in the world despite debt from F\&I war, versus practically none.
      \item England has a large, well-trained army and also Hessians versus mostly volunteers.
      \item England has the strongest navy in the world, versus none. 
      \item Oddly England was lacking in terms of generals and leaders, while the Colonies did have many dedicated and able officers (But less-well trained.)
      \item British are in a strange geometry and a strange land, versus colonists on familiar land and terrain.
    \end{itemize}
\end{document}
