\documentclass{article}
\begin{document}

  \begin{center}
    {\small{} Early American History Lecture Notes} \\[0.6cm]
    {\small{} Cameron Carroll -- October 7, 2013} \\[0.6cm]
    {\small{} Lecturer: Judy Campbell, Cuyamaca College}\\[1cm]
    {\small{} Establishing the United States I}\\[1cm]
  \end{center}
  
  \tableofcontents
  \newpage
  

  \section{What is a constitution?}
    \begin{itemize}
      \item  A set of customs, traditions, rules \& laws. 
      \item Could be written, or unwritten. Ours is written. England's is unwritten (a series of court cases.)
      \item Sets up how a government is organized \& operated.
      \item Grants powers to a government, and also limits its authority. People in power in our government cannot go past the constitution or they get in trouble. (Example: Gun control is a power granted by the constitution to the governed -- the government couldn't prevent governed from having guns because they couldn't get past the constitution. Also tried to raise prices on ammo, still couldn't get past constitution.)
      \item Defines structure of government. 
      \item Explains relationship between government and its governed.
      \item In comparison to a dictatorship where the rules are just decided upon by a ruler instead of documented guidelines.
      \item Constitutional government is a `limited' government. Is based on rule of law. Constitution is the `fundamental law.' On the other hand, `statuary laws' are outside of the constitution, are laws for ongoing daily use. Constitution is the basis of everything.
    \end{itemize}

  \section{Constitutional Convention}

  \begin{itemize}
    \item After Revolutionary War, we were governed by Congress of Confederation \& by Articles of Confederation. (1780's) -- Congress of Confederation is having a tough time trying to run the country which is a lot of territory, largely unsettled, to control. 
    \item 4 Million people in the territory to be governed.
    \item Currency was the `continential.' (Not worth much.) Also operated with French Francs, English Pounds, Spanish Peseta. States were deciding values for currency, however.
  \end{itemize}
  
  \subsection{Meetings \& How They Started}
    \begin{itemize}
      \item In 1785, Virginia \& Maryland were having some problems in terms of trading/shipping between themselves. So they meet at Mt Vernon (sorta neutral territory, Washington's house) and solve their problems.
      \item In 1786, five states get together in Annapolis, Maryland because the first meeting was so successful. This second meeting is successful as well.
      \item In 1787, 12 of the states get together. Note that Georgia DOES participate but Rhode Island doesn't want to go because they didn't want other states to dictate their currency.) This third meeting, 1787, is the Constitutional Convention.
    \end{itemize}
  \subsection{Ideas \& People}
    \begin{itemize}
      \item 80\% of the people at the C.C were at the 2nd consitutional congress, were rich young men, movers \& shakers. 8 of them had signed the Dec. They wanted the country to be solid, all of them believed that obedience to the bible was essential. Were giants of revolution, propertied class. No frontiersmen, small businessmen, debtors, farmers. Only educated lawyers, politicians, et cetera. `A group of demigods.'
      \item Adams says that the constitution is only adequate for a `wholly moral and religious people.'
      \item Absence of radicals: Patrick Henry said `I smell a rat' -- Refers to the small group of demigods creating another small elite.
      \item James Madison \& Edwin Randolf steered this group (C.C.) to write a new constitution.
      \item Washington is president of the C.C. at this time.
      \item Madison is known as `father of American constitution.' But he didn't write it... the took the notes at the convention. The doors were barricaded, windows covered over; We only know what was going on inside because of Madison's notes. 
      \item Poor Ben Franklin, the peacemaker, was so old he had to be carried into the meetings. All types of personalities were at the convention, and Franklin bound them together as everyone revered him.
      \item Gouverneur Morris is the `penman of the constitution' -- Actually writes it, and is author of the preamble.
      \item From the Greeks, they take the idea of a democracy: Government ruled by the people.
      \item From the Romans, they take the idea of a republic: Representative rule.
      \item from the British, idea of limited power which uses common law of rule. Power is under the law.
      \item From the Iriquois, they take the idea of tribal councils.
    \end{itemize}

  \subsubsection{Agreements}
    \begin{itemize}
      \item They agree on three things:
        \begin{enumerate}
          \item A republic with authority, run by citizens and not a king.
          \item No particular group holds power.
          \item And central power should not have absolute authority over the states; States have rights. 
        \end{enumerate}
      \item Also agree on a federal system of government (Divided into national and state governments.)  States give up some of their power to the federal government. For example, there is no Californian navy or currency. National government takes powers given to it and uses it for the good of the states. 
      \item Decided on three branches of government: Executive, legislative \& judicial. This idea comes from the bible... `The Lord is our judge, lawgiver, and king.'
    \end{itemize}

  \subsubsection{Disagreements \& Compromises}
    \begin{itemize}
      \item Most important compromise, the `Great Compromise' happens because New Jersey and Virginia have a problem:
        \begin{itemize}
          \item Virginia is large and had lots of people. They want to elect representation by population.
          \item New Jersey is very small and didn't have a ton of people, so they felt they would be punished unfairly.
          \item Congress is compromised upon by creating the bicameral system of a House and the Senate.
        \end{itemize}
      \item Slave population compromise: 
        \begin{itemize}
          \item North thinks that the South should pay taxes on their slaves, but the South thinks they should thus be represented in congress. 
          \item However slaves weren't viewed as people as much as property in the South. So they compromise and decide that a slave is 3/5ths of a person for taxation \& representation.
          \item The southerners were ready to walk out of the C.C. forever over this issue.
        \end{itemize}
      \item Trade compromise: No slave trade after 20 years. 
    \end{itemize}
\end{document}