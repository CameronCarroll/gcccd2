\documentclass{article}
\begin{document}

  \begin{center}
    {\small{} Early American History Lecture Notes} \\[0.6cm]
    {\small{} Cameron Carroll -- October 9, 2013} \\[0.6cm]
    {\small{} Lecturer: Judy Campbell, Cuyamaca College}\\[1cm]
    {\small{} Establishing the United States II}\\[1cm]
  \end{center}
  
  \tableofcontents
  \newpage

\section{What could the Constitutional Convention agree on?}
\begin{itemize}
  \item Republic government.
  \item Branches of government: Legislative, judicial, executive. This idea comes from an Enlightenment thinker, Montesquieu. System of checks and balances, where no branch has more power than another, is his idea. 
  \item Legislative creates laws, executive enforces laws, judicial interprets laws. 
  \item President is selected for 4 years per term, two possible terms.
  \item People choose the legislature.
\end{itemize}

\section{What did the Constitutional Convention compromise on?}
  \subsection{3/5ths compromise}
    \begin{itemize}
      \item South wanted slaves to count for representation.
      \item North wanted them to count for taxation.
      \item They compromised by saying that 3/5ths of the slave population counts for representation and taxation.
    \end{itemize}
  \subsection{Trade compromise}
    \begin{itemize}
      \item Slave imports were to stop after 20 years.
      \item North wanted to stop the slave trade.
      \item South figured they could just breed enough slaves for their needs after this point.
    \end{itemize}
  \subsection{Great compromise}
    \begin{itemize}
      \item Gave proportional representation in the lower house, protecting the rights of the populace.
      \item Gave equal representation in the upper house, protecting the rights of the states.
    \end{itemize}

\section{Electoral College}
\begin{itemize}
    \item President is elected by this system.
    \item Each states gets number of votes corresponding to representatives in house + senate. These representatives are (supposed) to vote to mirror the popular vote in their state. 
  \end{itemize}

\section{Bill of Rights}
  \begin{itemize}
    \item Some had a problem with the constitution: Many rights weren't being protected. So Madison writes Bill of Rights. 
    \item He proposes 12, and 10 are accepted. 
    \item Thomas Jefferson pushes them through.
    \subsection{Specific Rights}
      \begin{itemize}
        \item Freedoms of speech, religion, assembly \& press. (First amendment to constitution.)
        \item Right to bear arms. (Second amendment.)
        \item No cruel/unusual punishment. (Eighth amendment)
        \item Cannot incriminate yourself. (Fifth amendment)
        \item Trial by Jury (Sixth amendment)
      \end{itemize}
    \item Only 27 amendments in total, out of (11,000 or 19,000, depending on estimates) total proposed.
    \item Fun fact: Massachusetts was the last state to accept religious freedom amendment, despite being the first to come here in search of it.
  \end{itemize}

\section{How does constitution keep current?}
  \subsection{Amendments}
    \begin{itemize}
      \item Amendments are specific changes to the constitution that have to be pushed through a long political gauntlet.
    \end{itemize}
  \subsection{Judicial Review}
    \begin{itemize}
      \item Supreme court has the power of judicial review.
      \item What changes is how the constitution is interpreted, which changes with age.
      \item For example, in Plessy vs Ferguson, segregation was found to be okay in 1890. Then in 1954, in Brown vs Board of Education, segregation was found to be a bad thing, integration was forced.
    \end{itemize}
  \subsection{Implied Powers}
    \begin{itemize}
      \item In Article 1, Section 8, there is the `elastic clause.'
      \item This is the `what is necessary and proper' clause. These are powers that are necessary and proper for running the country but aren't specifically worded in the constitution.
      \item This clause provides wiggle room for the constitution to still be applicable much later. Gives government freedoms and powers that it wouldn't normally have.
    \end{itemize}

\section{Presidency \& George Washington}
  \subsection{Requirements}
    \begin{itemize}
      \item Older than 35.
      \item Natural born citizen of the United States.
      \item Have lived on US soil for the last 14 years.
      \item If born on a military base or an embassy, that counts as US territory and therefore you are a natural born citizen. Also anyone born in a US territory is a natural born. 
      \item President is sworn into office/inaugurated, with one hand on a bible.
    \end{itemize}
  \subsection{Washington's Establishment}
    \begin{itemize}
      \item George Washington is sworn in in NYC, the capital. Then the capital is moved to Philadelphia around 1790ish. 
      \item Virginia and Maryland both give up a few acres of land to give the government land on which it wouldn't be beholden to any state. Then they create Washington DC, named after Washington and Columbus and move the capital there.
      \item Government is fairly stable, country got off to a strong start with Washington at the helm. There wasn't much choice as to who was going to be president... of course it was George. 
      \item People were gross at this time, and George had lost all his teeth. Also he was 57 when inaugurated.
    \end{itemize}
  \subsection{Washington's Presidency}
    \begin{itemize}
      \item Instead of `your highness' or whatever, Washington decided he should be called `Mr President.'
      \item Washington establishes supreme court with Judiciary Act, with 6 justices. Along with this came lower courts, federal district court, court of appeals, circuit courts, etc.
      \item Washington establishes a `cabinet.' Asks Jefferson to be secretary of state; Hamilton to be Secretary of Treasury; Henry Knox was secretary of war. These men advised Washington. Also this is where presidential succession comes in.
      \item All the guys at the CC are called the `founding fathers.'
      \item As a result of first elections, Washington is president, but the country was to be run by the men who drew up the constitution.
      \item There were two divisions when drawing up constitutions and heading to ratification... \\
        Federalists:
        \begin{itemize}
          \item Wanted constitution passed as-is.
          \item Federalist Papers: 85 artices urging states to pass the constitution. (And so they did, years later.)
        \end{itemize}
        Antifederalists:
        \begin{itemize}
        \item Didn't want the constitution passed without a bill of rights.
        \item Very few antifederalists compared to the former.
        \item Got their way ultimately... When Washington was elected, he promised that he would help pass \& adopt Bill of Rights, the first 10 amendments. (And so he did.)
        \end{itemize}
      \end{itemize}

\end{document}