\documentclass{article}
\begin{document}
  \textbf{Copyright Note: This document contains text largely copied or paraphrased from `Give Me Liberty!' (Eric Foner)}
  \tableofcontents
  \newpage

  \section{Focus Questions}
  \begin{enumerate}
    \item \textbf{What were the achievements and problems of the Confederation government?}
    \item \textbf{What were the major debates that gave shape to the Constitution?}
    \item \textbf{What were the purpose and meaning of the Bill of Rights as it emerged from the ratification process?}
    \item \textbf{What did `We the People' mean in the new nation for Indians and African-Americans?}
  \end{enumerate}

  \section{America after its Revolution}
    \begin{enumerate}
      \item Parades testified to strong popular support for the Constitution in the nation's cities; New York's Grand Federal Procession was led by farmers, followed by artisans. The prominent role of skilled artisans reflected how the Revolution had secured their place in the American public sphere.
      \item United States exceeded Britain, Spain \& France combined in size. 
      \item Advantages as a nation: Physically isolated from Old World; youthful population destined to grow very large; And a broad distribution of property ownership and literacy. 
      \item Disadvantages as a nation: Control of the vast territory was not secure; Nearly all of the 3.9 million Americans recorded in the first national census of 1790 lived on the Atlantic coast; Britain, Spain and Indians were still threats.
      \item Local loyalties outweighed patriotism... `We have no Americans in America' (John Adams.)
    \end{enumerate}

    \subsection{Articles of Confederation, Confederation Government}
    \begin{enumerate}
      \item Was first written constitution of the United States
      \item Drafted by congress in 1777, ratified in 1781.
      \item Sought to balance the need for national coordination with the fear of centralized public power.
      \item Resembled less a blueprint for common governent than a `firm league of friendship' between states.
      \item National government consisted of a one-house congress; Each state cast a single vote; Major decisions required 9 votes, not a simple majority.
      \item Only granted powers essential to the struggle for independence -- declaring war, conducting foreign affairs, making treaties.
      \item No executive branch or judicial branch to enforce or interpret the laws.
      \item Revenue came mainly from contributions from individual states; Government had `no teeth' to levy taxes or regulate commerce.
      \item Various amendments proposed to strengthen the national government, but none recieved the approval level needed.
      \item  It established national control over the land west of the thirteen states and provided rules for its settlement. Disputes over access to western land almost prevented ratification of the Articles in the first place, as the orginal states claimed immense tracts of land all the way to the `South Sea.' (The Pacific.) Eventually these states ceded their western claims to the central government in the interest of national unity, and the articles become ratified at last.
      \item The Confederation government faced worsening econoimc problems: Congress couldn't pay soldiers or debts; The Union was barred from trading in the West Indies; Imported goods flooded the market, which drove down wages and `drained money from the country.' Individual states tried to adopt their own economic policies: Printing money, imposing tariffs, or delaying debt collection. 
    \end{enumerate}

    \subsection{Settling the West}
    \begin{enumerate}
      \item Area between existing states and the Mississippi River was spoken of as empty, but actually was occupied by 100,000 Indians. 
      \item By aiding the English, \emph{all} tribes forfeited the right to their lands. (According to Congress, anyway.)
      \item In 1784 and 1785, the Confederation received large surrenders of Indian land north of the Ohio river, and from the Cherokee, Choctaw and Chickasaw tribes in the south.
      \item While leaders believed that the farmers needed access to western land, they also saw the land sales as a source of revenue and worried about conflict with Indians.
      \item To settlers, the right to take posession of western lands [from Indians] and use them as they saw fit was an essential element of American freedom.
      \item `Ordinance of 1784' established stages of self-government for the West. Districts would initially be governed by Congress and eventually admitted to the Union.
      \item A second ordinance, in 1785, regulated land sales in the region north of the Ohio River, the Old Northwest. A township would be divided into sections, and the profits from one sector in each was to be used for public education. This system intended to control \& concentrate settlement but settlers moved in before the surveys were completed. (The \$600 for a section was beyond the reach of most people, and people really needed land.)
      \item For many years, national land policy benefited private land companies and large buyers more than individual settlers.
      \item A third ordinance, the `Northwest Ordinance of 1787,' called for eventual creation of 3-5 states north of the Ohio River, and east of the Mississippi. Jefferson called this principle the `empire of liberty' -- Rather than ruling over the area, the area's population would be equal members of the political system.
    \end{enumerate}

    \subsection{Shays' Rebellion}
    \begin{enumerate}
      \item Later 1780s, debt-ridden farmers closed the courts in western Massachusetts in order to prevent the seizure of their land.
      \item Participants believed they were acting in the spirit of the Revolution; Modeled tactics on the crowd activites of the 60's and 70's, employing liberty trees and liberty poles. (Those dang hippies.)
      \item Governor declared Americans woul descend into ` state of anarchy, confusion, and slavery' without adherence to the law.
      \item Thomas Jefferson: `A little rebellion now and then is a good thing. The tree of liberty must be refreshed from time to time with the blood of patriots and tyrants.'
      \item Was the culmination of a series of events which led to an influential group of americans believing that they must strengthen the government to ensure its economic health, and strengthens the nationalist cause greatly.
      \item James Madison: `Liberty may be endangered by the abuses of liberty as well as the abuses of power.'
    \end{enumerate}

    \subsection{Nationalism in the 80's. (...The 1780s.)}
    \begin{enumerate}
      \item A number of `nationalists' wanted to make the central government stronger. (Madison, Hamilton...)
      \item Also included army officers, influential economic interests: bondholders, artisans, merchants, and all those that feared that the states were seriously interfereing with property rights.
    \end{enumerate}

  \section{America's New Constitution}
    \begin{enumerate}
      \item In 1786, delegates from six states met at Annapolis, Maryland to consider how to regulate commerce. They propose another gathering, in Philadelphia, to amend the Articles of Constitution.
      \item When every state except Rhode Island, which went the furthest in developing its own economic policy and relief, meets in Philadelphia in 1787 they scrap the Articles of Confederation entirely and decide to draft a new constitution.
      \item New constitution was to:
        \begin{enumerate}
          \item Create a legislature, an executive, and a national judiciary. 
          \item Allow Congress to raise money by itself.
          \item Prevent states from infringing on property rights.
          \item Establish a government which represented the People. 
        \end{enumerate}
      \item Hamilton: `The rich and well-born must rule, for the masses seldom judge or determine right.'
      \item Most delegates wanted a middle ground between monarchy\\aristocracy and the excesses of popular self government.
      \item Madison proposed the Virginia Plan: Creation of a two-house legislature based on state population.
      \item Smaller states rallied behind New Jersey plan: One-house congress in which each state casts one vote, as under Articles of Confederation.
      \item Eventually a compromise was reached and our bicameral system was born:
        \begin{enumerate}
          \item A Senate where each state has two representatives, chosen for six years by state legislatures and thus insulated from sudden shifts in public opinion.
          \item A House of Representatives, where each state has representatives based on population for two year terms, and elected directly by the people
      \end{enumerate}
      \item While including popular election as at least one part of the political regime, (Madison: `essential to every plan of free government.') government was overall raher undemocratic. The delegates devised a cumbersome system of indirect election because they did not trust ordinary voters to choose the president and vice-president directly.
      \item Although some of those drafting the new constitution were adamantly against the institution of slavery, they chose national unity over insisting on that viewpoint. The constitution included:
        \begin{enumerate}
          \item The Fugitive Slave Clause, which accorded slave laws `extraterritoriality.'
          \item That the slave trade couldn't be abolished for 20 years.
          \item That 3/5ths of the slave population would be counted when determining each state's representation in the House.
        \end{enumerate}
      \item Gouverneur Morris put the finishing touches on the final draft of the Constitution, trying to make it `as clear as our language would permit.'
      \item Last session of Constitutional Convention took place on September 17, 1787. Of 45 remaining delegates, 39 signed.
    \end{enumerate}

    \subsection{Federalism \& `Checks and Balances'}
      \begin{enumerate}
        \item Constitution embodies two basic political principles:
          \begin{enumerate}

            \item Federalism or `division of powers'
              \begin{enumerate}
                \item Refers to relationship between national government and the states.
                \item Constitution greatly strengthened national authority.
                \item Congress was empowered to levy taxes, borrow money, regulate commerce, declare war, deal with foreign nations\/Indians and promote `general welfare.'
                \item National legislation declared the `supreme Law of the Land.'
                \item Included strong provisions to prevent states from infringing on property rights.
                \item States barred from issuing paper money, impairing contracts, interfering with interstate commerce, or leving import-export duties.
              \end{enumerate}

            \item System of `checks and balances' known as `separation of powers.'
              \begin{enumerate}
                \item Refers to the way the Constitution seeks to prevent any branch of national government from dominating the others.
                \item Congress enacts laws, but the President can veto them.
                \item Federal judges are nominated and approved by the other two branches, but serve for a life term to allow independence.
                \item President can be impeached by the House and removed from office by the Senate.
              \end{enumerate}
          \end{enumerate}
      \end{enumerate}

    \subsection{Ratification Debate}
      \begin{enumerate}
        \item Supporters of ratification, federalists, argued that the national government should have increased power.
          \begin{enumerate}
            \item Hamilton, Madison, Jay composed a series of essays which were gathered as a book, `The Federalist' in 1788.
            \item Hamilton: Government ws an expression of freedom, not its enemy. Any government could become opressive, but with its checks and balances and division of power, the Constitution made political tyranny almost impossible.
            \item Hamilton: The constitution had created `the perfect balance between liberty and power.'
            \item Madison: `Extend the Sphere.'
              \begin{enumerate}
                \item The nation's strength lies in its size and diversity. The multiplicity of religious denominations offered he best security for religious liberty.
                \item In a nation so large, many distinct interests would arise so that no individual one could be able to take over the governent and oppress the rest.
                \item Reinforced tradition that saw continuous westward expansion as essential to freedom.
                \item Represented a shift away from `republican' emphasis on virtuous citizenry toward `liberal' view that men are motivated by self-interest.
              \end{enumerate}
            \item Support was mostly in cities and in rural areas which were tied closely to the commercial marketplace.
            \item Most energetic supporters were men of substantial property.
            \item Also supported by urban artisans, laborers, and sailors.
          \end{enumerate}

        \item Opponents of ratification, anti-federalists, insisted that the Constitution gave too much power to the national government and too little liberty to the people or the states.
          \begin{enumerate}
            \item Lacked coherent leadership of the Federalists.
            \item Included state politicians and some revolutionary war heroes (Samuel Adams, John Hancock, Patrick Henry.) Also included small farmers.
            \item Some denounced protections for slavery, others argued it might lead to abolition.
            \item Predicted that the new government would fall under the sway of merchants, creditors, and others hostile to the interests of ordinary Americans.
            \item James Lincoln: `What is liberty? The power of governing yourselves. If you adopt this constitution, have you this power? No.'
            \item Pointed to the lack of a bill of rights. (Patrick Henry: `The most absurd thing to mankind that the world ever saw.')
            \item Support mostly from small farmers and in rural areas.
          \end{enumerate}

        \item In the end, the Federalists are better represented in the media and ultimately win ratification of the constitution.
      \end{enumerate}

    \subsection{The Bill of Rights}
      \begin{enumerate}
        \item Ten constitutional amendments approved by the first Congress, ratified by states in 1791.
        \item Madison believed a bill of rights would be redundant due to the balances already built into the Constitution. No list of rights could ever anticipate the numerous ways that Congress might operate in the future. `Parchment Barriers' to the abuse of authority would prove least effective when most needed.
        \item However, to satisfy the Constitution's critics, Madison presents a series of amendments.
        \item Offers a definition of the `unalienable rights' Jeffersion mentions in the Declaration. Not having been granted by the government in the first place, they could hardly be rescinded.
        \item The ninth amendment declares rights not specifically mentioned are `retained by the people.' This implies that the constitution was not meant to be complete, opens the door to later legal recognition of rights not originally mentioned. 
        \item The tenth amendment assures that powers not delegated to the national government or prohibited by the states continue to reside with the states.
        \item Brings constitutional recognition of religious freedom.
        \item Once an entitlement of members of Parliament and colonial assemblies, free speech becomes a basic right.
      \end{enumerate}

    \subsection{National Identity}
      \begin{enumerate}
        \item The Constitution begins with `We the People' but identifies three populations inhabiting the United States:
          \begin{enumerate}
            \item Indians, treated as members of their tribe and not the merican body politic.
              \begin{enumerate}
                \item Were deemed unfit for citizenship by most Americans, but some prominent Americans rejected this idea. (Thomas Jefferson)
                \item Tribes had no representation in the government and those `not taxed' were not counted when determining congressmen.
                \item Tribes wanted to retain autonomy and their culture while Americans wanted the to adopt western gender norms, among other assimilations.
              \end{enumerate}
            \item `Other Persons,' or slaves.
              \begin{enumerate}
                \item Majority of blacks were slaves, and slavery rendered them invisible to those imagining the American people.
                \item Constitution doesn't define who citizens are, so it's up to states to determine boundaries of liberty.
              \end{enumerate}
            \item the `People.' (Those entitled to American freedom.)
          \end{enumerate}
      \end{enumerate}
\end{document}