\documentclass{article}
\begin{document}

  \begin{center}
    {\small{} Early American History Lecture Notes} \\[0.6cm]
    {\small{} Cameron Carroll -- September 23, 2013} \\[0.6cm]
    {\small{} Lecturer: Judy Campbell, Cuyamaca College}\\[1cm]
    {\small{} Revolutionary War I}\\[1cm]
  \end{center}
  
  \tableofcontents
  \newpage
  
  \section{Breed's Hill} 
    \begin{itemize}
      \item 2nd Continential Congress sends Washington \& army to Boston from Philidelphia.
      \item Before he can get there, British troops are flooding into Boston. Prescott, in charge of Minutemen, sees buildup of British warships \& begin to fortify Breed's Hill.
      \item British general Gage decides not to wait to strike the Minutemen at Breed's Hill.
      \item Colonists are short on ammo \& supplies... Prescott says not to fire until Colonists could see the whites of the English eyes.
      \item  3 times the British soldiers tried to take Breeds hill \& were repelled every time. Eventually the British overtake the hill, but British lost 1000 and Colonists 300; The British generals thought this too costly. 
      \item Also known as the Battle of Bunker Hill, which took place on and around Breed's Hill.
    \end{itemize}
 

  \section{Common Sense} 
    \begin{itemize}
      \item Published in 1776, Thomas Paine wrote Common Sense. (A tiny pamphlet.)
      \item This guy was an Englishman, a corsetmaker. He becomes called the `pen of the American Revolution.'
      \item It sold 100,000 copies, spreads like wildfire. Was geared to `common man' -- frontiersman as well as businessman.
      \item Basic premise is that it only makes common sense to separate from England. Before this pamphlet it isn't really addressed.
      \item Also ridicules the king, pushes idea that people should control their own government.
      \item Argues that constitutions should be written down, governments should have checks \& balances, and that one guy shouldn't be in charge. (Basic ideas we sorta take for granted today.)
    \end{itemize}

  \section{Declaration of Independence} 
    \begin{itemize}
      \item May 1776, 2nd Continental Congress says to all Colonies to kick out the royal governors. (They did.) Then they go to each colony and ask each one to come up with their own constitution.
      \item Franklin, Livingston, Adams, Sherman \& Jefferson are elected to write the Declaration of Independence.
      \item Thomas Jefferson is given the task of actually writing it.
      \item Franklin \& Adams were to proofread it.
      \item Ideas involved came from the Enlightenment:
        \begin{itemize}
          \item Life, Liberty \& Property.
          \item Natural rights, not political rights.
          \item All men are created equal.
          \item Humanism, reasoning, rationale.
          \item A secular approach to life.
        \end{itemize}
      \item Declaration is written to the King of England. It was easier to focus on an individual (the King) rather than a country (all of Britain.)
      \item Copies also sent to Spain, France, and all enemies of England.
      \item Adopted by 2nd Continental Congress, whose president is John Hancock.
      \item 3 topics covered in the Declaration:
        \begin{enumerate}
          \item Why we have a problem with England.
          \item What England did wrong. ( A list of grievances.)
          \item Declaration of cutting ties from England
        \end{enumerate}
      \item By siging the Declaration, all involved were committing treason against the mother country.
      \item Fun fact: John Hancock signs `big enough for the King to read without his glasses.'
    \end{itemize}

  \section{Three Periods of Revolutionary War}
    \subsection{1: Lexington until Declaration} 
      \begin{itemize}
        \item There was sort of a 9 month standoff after Lexington/Concord. British troops aren't tryin to fight in the snow, and also the American soldiers had to go take care of their families and lives and houses.
        \item King is determined to end the war at this point (Decl. of Independence)
      \end{itemize}
        
    \subsection{2: Declaration until French Entry} 
      \begin{itemize}
        \item In 1776, English have 32,000 soldiers in the colonies and an additional 15,000 Hessian mercenaries.
        \item Washington has about 18,000 soldiers at best but usually more like half. American soldiers have no uniforms: A ribbon differeniates an officer from enlistedmen; They don't take orders well, aren't really soldiers but untrained guys. Plus, people from some northern states don't like the people from southern states. Washington's army is a rag-tag bunch, to say the least.
        \item In 1778, after 2-3 years, the French start to help Americans.
        \item General Howe notes that New York has a lot of British loyalists, and sends some soldiers there as well as to the Carolinas in a two-prong attack to end the war.
        \item Washington needs a spy, finds Nathan Hale who volunteers to be a spy (a schoolteacher.) Buuuut he gets drunk and spills his plan to help Washington, and is taken out the next day and hung as a traitor without a trial. Before death, he says `I only regret having one life to give for my country.'
        \item As British armies begin pushing in from Long Island toward New York, Washington consistently loses battles and gets pushed back.
        \item Finally Washington is pushed back against the water \& escapes via the river with help of the Massachusetts boaters. Also helped by a chance fog, which Washington calls `providence.'
        \item Howe is pissed though, takes New York, heads for Philidelphia passing Trenton \& Princeton.
      \end{itemize}
     
    \subsection{3: ???} On the next episode of Early American History.
\end{document}
