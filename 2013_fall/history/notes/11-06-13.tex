\documentclass{article}
\begin{document}

\begin{center}
    {\small{} Early American History Lecture Notes} \\[0.6cm]
    {\small{} Cameron Carroll -- November 6, 2013} \\[0.6cm]
    {\small{} Lecturer: Judy Campbell, Cuyamaca College}\\[1cm]
    {\small{} Settling the West II}\\[1cm]
\end{center}
  
\tableofcontents
\newpage

\section{Mormons}
  \begin{itemize}
    \item Mormons started under Joseph Smith in New York. He got a revelation from god! 
    \item The Mormon church practice[d] polygamy. This practice was [is] frowned upon by other Americans, but hey it's their thing. So they get run out of New York, and then from Ohio, and then out of Missouri and out of Illinois. They were okay people but nobody could stand the polygamy.
    \item Smith is murdered, so Brigham Young takes over and moves everybody out West along the Mormon route to a piece of desert which becomes Salt Lake City.
    \item When the Mormons wanted to join the United States they were forced to abandon polygamy.
    \item A few thousand of them head out into the desert and turn it into ` a land of milk and honey.' They proper and turn the desert into a beautiful area. 
    \item They establish the state of Deseret.
    \item Unfortunately for the Mormons, however, when they head out into the middle of nowhere, we get the Mexican Cession and so they're not isolated for very long. This leads to more conflict.
    \item Utah has had a lot of resident Mormons because of this initial exodus.
  \end{itemize}
\section{Oregon Trail}
  \begin{itemize}
    \item Oregon was bought peacefully from England around 1850. We settled on bringing the latitude line down a few degrees in exchange for Washington, Oregon and Idaho. 
    \item The first missionaries in Oregon were the Whitmans and Spauldings. They went down the Columbia river and settled a mission in Oregon, at the end of the Oregon trail. They were encouraged by Sacagaweia's trip. The Whitman lady was the first white woman to cross.
    \item They open their home which turns into a mission for thousands of people. The people would then go down into California.
    \item People died by the hundreds on the trail out to Oregon, ie smallpox, chicken pox, injury, attacked by a bear, etc etc. They would leave behind their children. So the children got dragged along by other families and if not wanted were left at the mission where they could live until they could leave on their own.
    \item The Indians in the area were somewhat tolerant, however smallpox broke out at the Missionary and wiped out the Indians. The Indians in return murdered the Whitmans and all the children.
    \item Mountain men would often open up stores, around which town would spring up.
  \end{itemize}
\section{Mining Frontier}
  \begin{itemize}
    \item James Marshall found gold at Sutter's Fort. It was supposed to be a secret, they thought they had a great deal, but of course it got out.
    \item There was a huge rush of prospectors coming to Northern California in search of riches.
    \item Vast majority would come by boat around South America, but many came overland as well.
    \item A few people struck it rich but the richest were the shopkeepers and lawyers.
    \item Places settled mostly known as `Boom Towns.' Extremely rudimentary buildings with a facade to look respectable. Filled with the men coming in to ` strike it rich,' created a town full of vices. Rough, to say the least.
    \item In the Rocky Mountains people also found copper and silver. Phoenix was started as a colony of tents. People had found copper and were mining into it. 
  \end{itemize}
\section{Pony Express}
  \begin{itemize}
    \item Only lasted 1.5 years.
    \item Went from St. Joseph, Missouri to Salt Lake City. 
    \item Was a mountain man path, first traveled by Carson and Fremont.
    \item 14-18 y/o males rode on the Pony Express, carrying a waterproof leather pouch.
    \item There were about 90 ` way-stations' along the way with 2-3 people manning each one. There were, then, a few hundred people running the way-stations and 90 riders.
    \item The riders rode about 75 miles per day, alone, across the Great Plains.
    \item Paid about \$100 per month.
    \item Stopped by the industrial revolution and the invention of the telegraph as well as progress of the railroads.
  \end{itemize}

\section{Industrial Revolution}
  \begin{itemize}
    \item Characterized by big changes in manufacturing, transportation, primarily in the first half of the 19th century. 
    \item Mountain men are exploring the West while the East undergoes the industrial revolution.
    \item Leads to corporations.
    \item Factories grow.
    \item Goods produced en masse.
    \item Yankee Ingenuity: Of course we're not smarter than anyone else, but we have things that help the nation's industry that others were lacking:
      \begin{itemize}
        \item All the natural resources we needed were available in one area.
        \item We had overseas trade and mercantilism.
        \item We have people who want our manufactured goods and natural resources. (People overseas.)
        \item We could trade amongst ourselves and be quite prosperous.
        \item People are pouring into the US, so we have tons of cheap labor.
        \item Government gave help to build roads and canals. National road is built running from Washington to St. Louis. Other roads are built all over the place. The government has the money and stability to help.
        \item We also have technical ingeniuity, or new inventions created. Electric telegraph: A copper wire run between DC and Baltimore marks the very first time people could communicate with each other without seeing them. First words broadcast: ` what hath god wrought?'
      \end{itemize}       
    \item By 1861, coasts were linked by the telegraph. 
    \item Interchangable parts were used in revolutionary war in guns. However Eli Whitney invents the cotton gin. Cotton gin removed the little black seeds from cotton balls before processing. 
    \item Elias Howe revolutionizes the New England area by discovering (stealing from England) a machine that could sew two pieces of cloth together in a straight line. 
    \item Then Singer takes Howe's machine, uses his own ingenuity, and creates a machine which can sew curves. 
    \item These efforts mean slaves no longer had to pick out the seeds, and factories could churn out huge quantities of goods.
    \item Textile mills in New England go crazy.
    \item The Erie Canal goes between Albany and Buffalo in New York; Revolutionizes transportation. So a ship in the Atlantic could go up the Hudson to Albany, cross the Appalachian mountains, then go down the Great Lakes into the interior of the country without ever landing. This made huge differences in shipping times and costs. 
  \end{itemize}

\end{document}
