\documentclass{article}
\begin{document}

  \begin{center}
    {\small{} Early American History Lecture Notes} \\[0.6cm]
    {\small{} Cameron Carroll -- September 9, 2013} \\[0.6cm]
    {\small{} Lecturer: Judy Campbell, Cuyamaca College}\\[1cm]
    {\small{} Early English Colonization III}\\[1cm]
  \end{center}
  
  \tableofcontents
  \newpage
  
\section{English Colonies: Georgia}
  \begin{itemize}
    \item Established ~120 years after Jamestown.
    \item Two humanitarians, Percival \& Oglethorpe felt sorry for debtors in prison.. so they took them to Georgia.
    \item They were really sent to act as a buffer between Spanish Florida and everything else. 
    \item Two rules: No slaves and no alcohol (rum.) 
    \item Total religious freedom was granted. 
    \item But this doesn't really work out: most ended up back in debt in Georgia instead. 
    \item Eventually king sends his royal governor to oversee things.
  \end{itemize}

  \section{Great Awakening}
  \begin{itemize}
    \item Was the first `mass movement' in America -- A movement looking for change. 
    \item At the time, for nine of the thirteen colonies you had to belong to a Church in order to vote.
    \item John Edwards:
      \begin{itemize}
        \item Born in Connecticut.
        \item Was in college at 13.
        \item Was President of Princeton.
        \item Preached in Puritan Massachusetts, but was made to leave.
        \item Went around preaching all over, and people would come from all over. Helped to unify areas.
      \end{itemize}
    \item John Wesley: An Englishman working down in Georgia.
    \item George Whitefield: 
      \begin{itemize}
        \item Another Englishman.
        \item Preached to Indians.
        \item Got huge gatherings that they couldn't fit in a church, so he would stand in a field sometimes talking to thousands of people.
        \item Ends up influecing ~150 colleges.
      \end{itemize}
    \item Circuit Riding Preachers:
      \begin{itemize}
        \item Rode in large circuits going to small towns, preaching from one to the next.
        \item Often serviced backwoods, but would share news as well as their revival message.
      \end{itemize}
  \end{itemize}

\section{French \& Indian War}
  \begin{itemize}
    \item Thanks to French and Spanish trade, Indians are pretty well armed with guns. 
    \item English begin pushing out and the Indians and French fight this expansion. Mostly Indian raids against English colonies. 
    \item In Queen Anne war, Indians kill a full 1/4 of the Colonists in New England. 
    \item Fun fact: Some people survived being scalped! 
    \item The French at this time were paying the Indians per scalp, no matter whether woman or baby or man.
    \item King George war: Largely skirmishes, more Indian raids.
  \end{itemize}
\end{document}