\documentclass{article}
\begin{document}

  \begin{center}
    {\small{} Early American History Lecture Notes} \\[0.6cm]
    {\small{} Cameron Carroll -- September 16, 2013} \\[0.6cm]
    {\small{} Lecturer: Judy Campbell, Cuyamaca College}\\[1cm]
    {\small{} Early English Colonization IV}\\[1cm]
  \end{center}
  
  \tableofcontents
  \newpage

\section{Proclamation of 1763}
  \begin{itemize}
    \item England tells colonies not to go over the mountains because it can't protect them from the Indians. But they send money and supplies to maintain an army of 10,000 British soldiers along the Proclamation line to protect from Indians. They were also stationed to provide feeling of subordinance to England. 
    \item England starts to impose taxes because mainlanders `paid for the French and Indian war.' It is deeply in debt from 80 years of war.
    \item Colonists see this as a `grab of power.' They saw themselves as Englishmen, but they saw their basic rights being violated. 
    \item They argued that Englishmen had a right to a trial by jury of peers -- So soldiers commiting crimes in Colonies should be tried in the Colonies, not England where they would be let off the hook.
    \item Parliament passed taxes to help pay for war and upkeep of soldiers left in Colonies. These taxes were passed `without representation.' 
    \item British had 10,000 troops in the colonies. Quartering Act allowed soldiers to be force colonists to feed and house them. Colonists decide that is an affront on personal property. Have some years of self-government behind them at this point already. 
  \end{itemize}

\section{`Basic Problem' / Mercantilism}
  \paragraph{} By mercantilism, colonies suffered. Mercantilism is the idea that colonies exist to benefit mother country economically. Writs of Assistance were used to enforce the acts (search warrants basically.)

  \subsection{Navigation Acts}
    \begin{itemize}
      \item These taxes were in effect while colonies were being settled but were largely skirted. They were actually enforced after the French \& Indian war.
      \item Decreed that all trade should be on British ships, which hurts New England shipbuilding.
      \item Also decreed that all sailors had to be British, which helps England's unemployment but hurts the colonies.
      \item Finally, certain items were itemized and could only be sold to England. (Cotton, tobacco.)
    \end{itemize}
  \subsection{Sugar Act}
    \begin{itemize}
      \item Taxed sugar, rum, molasses, coffee, or anything else that had to do with sugar.
    \end{itemize}
  \subsection{Wool Act}
    \begin{itemize}
      \item Settlers were raising sheep, spinning wool, weaving on loom to make clothes for family.
      \item Wool act says colonists can't manufacture anything with their own wool!
      \item This forced them to shear the sheep and send the raw material to England, which will manufacture goods and sell them back. (Plus taxes.)
    \end{itemize}
  \subsection{Quartering Act}
    \begin{itemize}
      \item Soldiers come to your door and announce that you get to feed and house them.
      \item Mostly dispersed across cities.
      \item Hessians (German mercenaries) also were expected to be quartered.
    \end{itemize}
  \subsection{Stamp Act}
    \begin{itemize}
      \item Tax on all papers or legal documents. (Playing cards, marriage notices, everything.)
      \item Colonists thought that inter-colonial trade shouldn't be taxed by England.
      \item This act raised lots of ire\ldots inspired burnings in effigy.
    \end{itemize}
  \subsection{Currency Act}
    \begin{itemize}
      \item No printing of money allowed by colonists.
      \item Before this, they had their own monetary system in addition to British pound.
      \item Opted to barter over using British currency when possible.
    \end{itemize}
  \subsection{Townshend Duties}
    \begin{itemize}
      \item King doesn't like Colonial reaction... so he passes even more taxes!
      \item Townshend Duties were meant to hit every colonist: on lead, paint, glass, paper, and tea.
    \end{itemize}
\end{document}