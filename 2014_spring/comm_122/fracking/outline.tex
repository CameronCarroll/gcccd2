\documentclass{article}
\raggedright
\renewcommand\thesection{\Roman{section}}
\renewcommand\thesubsection{\thesection.\Alph{subsection}}

\renewcommand{\theenumi}{\arabic{enumi}}
\renewcommand{\theenumii}{\alph{enumii}}
\renewcommand{\theenumiii}{\roman{enumiii}}

\usepackage{hyperref}

\begin{document}
\section*{}
  \begin{description}
    \item[Title:] Truths about Fracking
    \item[General Purpose:] To Inform
    \item[Specific Purpose:] To inform the audience about fracking's role in tomorrow's economy and environment.
  \end{description}

\section{Introduction}
  \subsection{I'd like everyone to imagine that they are playing civilization, and you are the ruler of a modern-age city-state.}
    \marginpar{\footnotesize{Attn Getter}}
    \begin{enumerate}
      \item Your capital city, Tenochtitlan is threatened by a Spanish cultural expansion from the north!
        \begin{enumerate}
          \item You're pumping out Aztec Warriors as fast as you can in Tenochtitlan.                                                     
          \item But suddenly Pablo Picasso is born over in Madrid!
          \item We will lose the capital in 15 turns unless we take action.
        \end{enumerate}
      \item If we can get fuel for our tanks, we can go take out Picasso before his art wins over our people.
        \begin{enumerate}
          \item We used up all our oil.
          \item Now we need natural gas to fuel our tank.
          \item We need to employ hydraulic fracturing to get natural gas.
          \item But fracking could possibly cause the population of nearby Chalco to become sick and will certainly make them dissident.
        \end{enumerate}
    \end{enumerate}
  \subsection{Reason to Listen}
    \marginpar{\footnotesize{Reason to Listen}}
    \begin{enumerate}
      \item Natural gas is important:
        \begin{enumerate}
          \item It's projected to displace coal as leading fuel in the power generation market with 35\% share by 2040. (Energy Information Administration, 2014)
          \item It produces fewer air pollutants than coal or oil at the power plant. (Environmental Protection Agency, 2013)
          \item It's widely considered to be a `bridge fuel' to close the gap between conventional oil and new renewable energy sources.
        \end{enumerate}
      \item Fracking brings us natural gas!
      \item America is poised to displace Saudi Arabia as the worlds top producer. (Usborne, 2014)
      \item The environmental impacts are being explored currently.
        \begin{enumerate}
          \item In 2011, the EIA estimated that there are huge quantities of recoverable oil in the Monterey Shale formation in California. (Cart, 2014)
          \item We're just going to have to wait and see what environmental studies find.
          \item In the mean time, oil and gas companies are going to go ahead and do it anyway.
        \end{enumerate}
    \end{enumerate}
  \subsection{Speaker Credibility}
    \marginpar{\footnotesize{Speaker Credibility}}
    \begin{enumerate}
      \item I took a lot of time looking through publications from a variety of sources:
        \begin{enumerate}
          \item Trade journals
          \item Academic journals
          \item Think-tanks
          \item Government agencies
        \end{enumerate}
      \item I believe this has given a fair perspective on the issue
      \item I also have an interest and am considering pursuing chemical engineering.
    \end{enumerate}
  \subsection{Thesis \& Preview}
    \marginpar{\footnotesize{Thesis \& Preview}}
    \begin{enumerate}
      \item I wish to convey a brief overview of the role and consequences of fracking.
      \item \begin{enumerate}
        \item First, I'd like to explain the process and purpose.
        \item Next, I will enumerate environmental concerns.
        \item Finally, I would like to show fracking's economic role today and in the future.
      \end{enumerate}
    \end{enumerate}
  
\section{Body}
  \subsection{Fracking process and purpose}
    \begin{enumerate}
      \item Shales are ancient layers of rock which were formed with large amounts of organic material. (Curtis, 2011why)
        \begin{enumerate}
          \item This organic matter turned into methane and seeped out of tiny spaces between shale particles and into small fractures in the rock. (Curtis, 2011why)
          \item Unlike traditional oil wells, the gas doesn't permeate easily through the shale and requires extra stimulation for production.
        \end{enumerate}
      \item Hydrofracking is a well-stimulation technique:
        \begin{enumerate}
          \item But it's not a new one:
            \begin{enumerate}
              \item Hydrofracturing was first used in Kansas in 1947 by the Stanolind Oil company, which injected acid and oil into a well to stimulate production. (Curtis, 2011wha)
              \item They obtained a patent and gave the Halliburton Oil Well Cementing Company exclusive rights. (Curtis, 2011wha)
            \end{enumerate}
          \item Large amounts of water are mixed with similarly large amounts of sand and a minute quantity of chemicals. (Curtis, 2011wha)
          \item This slurry is injected into a pre-drilled well at very high pressure which fractures the rock. (Curtis, 2011wha)
            \begin{enumerate}
              \item Gas is trapped in the gaps between shale particles and also in small natural fractures. (Institute for Energy \& Environental Research for Northeastern Pennsylvania)
              \item When the pressurized slurry is forced into the rock, it expands existing fractures and connects them to the well bore. (Institute for Energy \& Environental Research for Northeastern Pennsylvania)
            \end{enumerate}
          \item The sand (proppant) holds fractures open after pressure is released. (Curtis, 2011wha)
          \item Water either flows back out the well or disperses into the rocks. (Schramm, 2011)
          \item With the fractures held open, the natural gas can flow into and up the well. (Curtis, 2011wha)
          \item The water returned is called flowback, while the natural water released is called produced water. (Schramm, 2011)
          \item Both types of wastewater can have high levels of dissolved solids and produced water can have TENORM. (Schramm, 2011)
          \item Wastewaters are stored in open-air pits and then either treated and reused or deep-injected. (Brown, 2014)
          \item The fruits of production are trucked away to be distributed either by train or pipeline.
        \end{enumerate}
    \end{enumerate}
    
  \subsection{Environmental impact}
    \begin{enumerate}
      \item Water Use:
        \begin{enumerate}
          \item A well in the US was fractured 30 times with a total slurry volume of 2.25 million gal, while two wells in Canada were fracked 25 and 5 times with 425,000 and 160,000 gallons respectively. (Daneshy, 2013) 
        \end{enumerate}
        
      \item Water Supplies:
        \begin{enumerate}
          \item Elevated bromide levels were found over a mile downstream from a commercial treatment facility handling wastewater. (Brown, 2014)
          \item Bromide can combine with organic matter and chlrone disinfectant to form trihalomethanes, associated with liver, kidney and nervous system problems. (Environmental Protection Agency, 2013)
          \item Out of 48 wells in a study, only one was significantly affected by nearby fracking operations, suggesting possible mixing of well water with wastewater or naturally occuring brine.
          \item 20\% of wells in same study had methane before drilling began, although generally far below advisory levels. (Boyer (et al.), 2012)
        \end{enumerate}
      \item Surface Water:
        \begin{enumerate}
          \item Wastewaters are often held in large open-air pits at the fracking site until they can be disposed of.
          \item There are around 25 centralized impoundments in Pa., which can be as large as a football field and hold at least 10 million gallons. (Brown, 2014)
          \item Fracking wastes may be disposed of by: (Brown, 2014)
             \begin{enumerate}
                \item Applying produced water as a road de-icer or dust suppressant
                \item Using drilling cuttings in road maintenance.
                \item Spreading liquids and sludge on fields
                \item (This is in addition to the conventional disposal methods, treatment and injection.)
            \end{enumerate}
          \item \url{http://www.youtube.com/watch?v=mxb671gbmkY} (Unlined pit in Kern County) \marginpar{\footnotesize{Visual Aid: Illegal Fracking Site}}
        \end{enumerate}
      \item Radioactivity:
        \begin{enumerate}
          \item Formations which have oil and gas also contain naturally occuring radionuclides, `NORM.' (Uranium, Thorium, Radium, Lead-210) (Environmental Protection Agency, 2012)
          \item These formations also contain brine which may dissolve radium (the decay product of uranium/thorium) into the flowback which can accumulate in scales or sluges in tanks and equipment. (Environmental Protection Agency, 2012) 
          \item Between 20 and 100 percent of facilities in every state reported some TENORM in their wastes. (Environmental Protection Agency, 2012) 
          \item Production process generate an estimated 100 tons of scale (per well) with high radioactivity 230,000 megatons (all together) of TERNOM sludge with low-to-moderate radioactivity and high levels of lead. (Environmental Protection Agency, 2012) 
          \item In order to try to control the TENORM, metals are cleaned before smelting, filters are placed in smeling stacks, sludges and scales are stored and disposed of properly, and waters are inejcted or coastally discharged. (Environmental Protection Agency, 2012) 
        \end{enumerate}
      \item Upward Migration:
        \begin{enumerate}
          \item It's a common concern that fluids from fracking will flow up through the bedrock into aquifers.
          \item However, the basins in which black shales are located do not allow for rapid upward migration of fluid or brine over short ($< 10^6$) years. (Flewelling \& Sharma, 2014)
          \item The rapid upward migration recently suggested in 2012 are then not physically plausible. (Flewelling \& Sharma, 2014)
        \end{enumerate}
      \item Soil Erosion:
        \begin{enumerate}
          \item Significant impacts on surface water resources were measured when a gas well pad was constructed with little attention given to surface drainage patterns. (McBroom (et al.), 2012)
          \item Stormwater generated by even relatively small rain events washed pollutants directly off of the pad. (McBroom (et al.), 2012) 
          \item Since construction of gas well pads in Texas isn't regulated like other construction sites, the responsibility lies on industry to self-regulate. (McBroom (et al.), 2012) 
        \end{enumerate} 
    \end{enumerate}
    
  \subsection{Economic impact}
    \begin{enumerate}
      \item In 2000, shale gas represented 2\% of US gas production; By 2013 that has grown to 37\%. (Houlton, 2013)
      \item Tight oil and shale gas support 1.7m jobs, predicted to rise to 3m by the end of the decade. (Houlton, 2013)
      \item Economic impact is stretching country-wide. (Houlton, 2013)
      \item Before the US figured out to fracture its shales, the price of natural gas was closely linked to that of oil since they came from the same fields. (Hughes, 2013)
      \item Now unlinked, gas prices fluctuated but were generally high until around 2009 when they settled to 4 USD/MMBtu, and have since come down to 3.30 USD/MMBtu. (Hughes, 2013)
      \item Low gas prices have been a boon to many industries:
        \begin{enumerate}
          \item Polyethylene producers have become some of the most cost-competitive in the world. (Stephen, 2012)
          \item Tire companies have tons of industrial sales to the fracking companies. (McCarron, 2013)
          \item Rail companies have seen huge growth: They transported around 10,000 carloads of crude in each quarter of 2010 but 97,000 in the first quarter of 2013. (Smith, 2013)
        \end{enumerate}
      \item Fracking has spurred development of boom-towns in rural areas of the country: `If you bring people in, there is no local housing, certainly no affordable housing... the rates are double or triple what they might be elsewhere.' (McNally, 2013)
    \end{enumerate}
    

\section{Conclusion}
  \subsection{}
  \marginpar{\footnotesize{Summary}}
    \begin{enumerate}
      \item Hydrofracturing is a well-stimulation technique and the only way we really know how to get the precious resources out from beneath us.
      \item Some of the dangers are clear and undeniable while others are fear, uncertainty, and doubt: There's a lot of rhetoric on both sides of the issue.
      \item The economic benefits are unchallenged, however -- Natural gas and the other chemical byproducts trickle down into a number of industries, with energy foremost.
    \end{enumerate}
    
  \subsection{}
  \marginpar{\footnotesize{Tie-Back}}
    \begin{enumerate}
      \item It's certainly not nice to use the analogy of a game strategy to talk about putting entire regions full of people in danger.
      \item But I think it's a fair analogy: The people are just happy or angry (or sickly) faces in a vast scheme.
      \item I argue that Chalco (and our own shale regions) will simply have to suffer the yet-unknown consequences because we absolutely cannot stand to let Picasso's tripe take over our treasured capital city full of sweet sculptures.
    \end{enumerate}
    
  \subsection{}
  \marginpar{\footnotesize{Reason to Remember}}
    \begin{enumerate}
      \item Fracking is here to stay and we simply don't know what we're even doing to ourselves yet.
      \item We're only just getting some real research into the safety of the industry in the last few years, so we're going to have to just watch and see what comes of it.
      \item There are a ton of jobs out there and more coming directly in the industry.
    \end{enumerate}
    
% References
\newpage
\begin{center}{\Large \textbf References}
\end{center}

  \paragraph{} \hangindent=.7cm Boyer, E., Swistock, B., Clark, J., Madden, M., Rizzo, D. (2012) The Impact of Marcellus Gas Drilling on Rural Drinking Water Supplies. The Center for Rural Pennsylvania. [\url{http://www.rural.palegislature.us/documents/reports/Marcellus_and_drinking_water_2012.pdf}]

  \paragraph{} \hangindent=.7cm Brown, V. (2014) Radionuclides in Fracking Wastewater. Environmental Health Perspectives. [GCCCD DB]

  \paragraph{} \hangindent=.7cm Cart, J. (2014) Vast oil trove trapped in Monterey Shale formation. Los Angeles Times. [\url{http://articles.latimes.com/2014/apr/06/local/la-me-monterey-shale-20140407}]

  \paragraph{} \hangindent=.7cm Curtis, R. (2011) What is hydrofracking? The Institute for Energy \& Environmental Research for Northeastern Pennsylvania.  [\url{http://energy.wilkes.edu/pages/156.asp}]

  \paragraph{} \hangindent=.7cm Curtis, R. (2011) Why does Marcellus contain gas? The Institute for Energy \& Environmental Research for Northeastern Pennsylvania. [\url{http://energy.wilkes.edu/pages/149.asp}]

  \paragraph{} \hangindent=.7cm Daneshy, A. (2013) Are we over-fracturing horizontal shale wells? Business Source Elite. [GCCCD DB]

  \paragraph{} \hangindent=.7cm Energy Information Administration (2013) Annual Energy Outlook 2014 Early Release Overview [\url{http://www.eia.gov/forecasts/aeo/er/executive_summary.cfm}]

  \paragraph{} \hangindent=.7cm Environmental Protection Agency. (2012) Oil and Gas Production Wastes [\url{http://www.epa.gov/radiation/tenorm/oilandgas.html}]

  \paragraph{} \hangindent=.7cm Environmental Protection Agency. (2013) Basic Information about Disinfection Byproducts in Drinking Water: Total Trihalomethanes, Haloacetic Acids, Bromate, and Chlorite. [\url{http://water.epa.gov/drink/contaminants/basicinformation/disinfectionbyproducts.cfm}]

  \paragraph{} \hangindent=.7cm Environmental Protection Agency (2013) Natural Gas [\url{http://www.epa.gov/cleanenergy/energy-and-you/affect/natural-gas.html}]

  \paragraph{} \hangindent=.7cm Flewelling, S., Sharma, M. (2014) Constraints on upward migration of hydraulic fracturing fluid and brine. Groundwater. [\url{http://onlinelibrary.wiley.com/doi/10.1111/gwat.12095/abstract;jsessionid=D8EB484C21D1FD47FE4270BB98D38BCD.f03t01}]

  \paragraph{} \hangindent=.7cm Houlton, S. (2013) Fracking Boost. Chemistry \& Industry. [GCCCD DB]

  \paragraph{} \hangindent=.7cm Hughes, E. (2013) Is fracking affecting natural gas prices? Industrial Materials. [GCCCD DB]

  \paragraph{} \hangindent=.7cm McBroom, M., Thomas, T., Zhang, Y. (2012) Soil Erosion and Surface Water Quality Impacts of Natural Gas Development in East Texas, USA. Water.  [\url{http://www.mdpi.com/2073-4441/4/4/944}]

  \paragraph{} \hangindent=.7cm McCarron, K. (2014) Drilling for Dollars Tire Business. [GCCCD DB]

  \paragraph{} \hangindent=.7cm McNally, J. (2013) Boom goes the Bakken. Supply House Times. [GCCCD DB]

  \paragraph{} \hangindent=.7cm Schramm, E. (2011) What is flowback, and how does it differ from produced water? The Institute for Energy \& Environmental Research for Northeastern Pennsylvania. [\url{http://energy.wilkes.edu/pages/205.asp}]

  \paragraph{} \hangindent=.7cm Smith, K. (2013) Risk and Reward from the US Fracking Boom. International Railway Journal. [GCCCD DB]

  \paragraph{} \hangindent=.7cm Stephen, M. (2012) Frack yeah! Canadian Plastics. [GCCCD DB]

  \paragraph{} \hangindent=.7cm The Institute for Energy \& Environmental Research for Northeastern Pennsylvania. Why is it necessary to fracture the rock? [\url{http://energy.wilkes.edu/pages/186.asp}]
  
  \paragraph{} \hangindent=.7cm Usborne, D. (2014) Fracking is turning the US into a bigger oil producer than Saudi Arabia The Independent. [\url{http://www.independent.co.uk/news/world/americas/fracking-is-turning-the-us-into-a-bigger-oil-producer-than-saudi-arabia-9185133.html}]
\end{document}