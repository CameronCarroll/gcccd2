  \documentclass[11pt,letterpaper]{report}
\usepackage[pdftex]{graphicx}
\usepackage[version=3]{mhchem}
\usepackage{tabularx}
\usepackage{latexsym}
\usepackage{multirow}

\newcommand{\HRule}{\rule{\linewidth}{0.5mm}}
\setlength{\topmargin}{-.7in}
\setlength{\leftmargin}{-.7in}
\setlength{\textheight}{9in}
\setlength{\oddsidemargin}{0in}
\setlength{\textwidth}{6.25in}



%\mbox{}\\
%_{()}

%\multicolumn{4}{|l|}{\ce{}} \\

\begin{document} 

\begin{titlepage}
\begin{center}

\textsc{\Large Lab 8}\\[1.5cm]
\textsc{\Large Grossmont College Chemistry}\\[0.5cm]
\includegraphics[width=0.15\textwidth]{./logo.jpg}

\HRule \\[0.4cm]
{ \LARGE \bfseries Report: Periodicity of Chemical Properties}\\[0.5cm]

\HRule \\[1.5cm]

\begin{minipage}{0.4\textwidth}
\begin{flushleft} \large
\emph{Author:}\\
Cameron \textsc{Carroll}\\[0.2cm]

\end{flushleft}
\end{minipage}
\begin{minipage}{0.4\textwidth}
\begin{flushright} \large
\emph{Instructor \& Class:}\\
Cary \textsc{Willard} - Chem 141 (6657)
\end{flushright}
\end{minipage}

\vfill

{\large \today}

\end{center}
\end{titlepage}

\section*{Procedure:}
Lehman, J; Olmstead, T; Vance, D, et alia. (2011)
Experiment 11: Periodicity of Chemical Properties
Chemistry 141 Lab Manual (Edition 5.2) (Pages 87-94)

\section*{Trend Analysis:}
\begin{enumerate}
  \item \textbf{Trend:} Atomic Radius vs Electronegativity \\
  \item \textbf{Definition (Electronegativity):} A chemical property describing the extent to which an atom attracts electrons toward itself. A higher value corresponds to a stronger attraction for electrons and results from a higher number of protons in the nucleus. The values for electronegativity can be calculated in a variety of ways, but this analysis used the popular Pauling Scale, which ranges from 0.7 to 3.98, roughly. Some elements have no electronegativity value at all, implying that they just don't attract electrons. \\
  \item \textbf{Summary:} As atomic radius increases, electronegativity tends to increase within a period. Also, electronegativity values tend to decrease from top to bottom except for the transition metals from group 6 onward, whose values decrease from bottom to top. \\
  \item \textbf{Graph:} \\
  \begin{center}
    \includegraphics[width=0.75\textwidth]{./electroneg_group.png}\\[0.7cm]
  \end{center}
  \item \textbf{Discussion:} The period-based trend can be explained thanks to increased effective nuclear charge: The amount of shielding from core electrons stays the same as you travel across a period, but the nuclear charge increases with more protons. The greater nuclear charge means that more energy is required to coerce an electron away. The first group-based trend, where values decrease from top to bottom in groups can be explained by the increased primary energy level. A higher energy level means further orbitals from the nucleus and therefore nuclear charge thanks to the $\frac{1}{r^2}$ relationship from Coulomb's law. It's much easier to snatch an electron away from these outside orbitals with weaker magnetic anchors to the atom.  For the transition metals that demonstrate an opposite tendency (that is, to increase from top to bottom) consider that the primary energy level 
  \item \textbf{References:} \\
    \begin{enumerate}
      \item https://en.wikipedia.org/wiki/Electronegativity \\
      \item http://www.chemguide.co.uk/atoms/bonding/electroneg.html \\
    \end{enumerate}
\end{enumerate}

\section*{Element Biography: Antimony}

While antimony was used by the egyptians as a component in the eye cosmentic ``kohl,'' it was not recognized or `discovered' as a pure element until thousands of years later. There's some controversy regarding whether Jabir Ibn Hayyan was aware of antimony or not: It's claimed that was familiar with it, but this isn't evidenced by any of his translated works. \\

In the 1500's, Biringuccio detailed a procedure for isolating pure antimony. Pure antimony found in the Earth's crust was detailed by von Swab in 1783. \\

Antimony is a silver colored lustious metal with a relatively low hardness. As with most metals, it has a very high boiling and melting point.\\

The element is currently used mostly in flame-retardant compounds because the antimony reacts with oxygen, starving the fire. It's also used in alloys for a number of different applications, as a silicon dopant, as a catalyst, a lining agent, and as a pigment. \\

Most of the antimony is mined from China, producing 89\% of the annual value with 120,000 tonnes. It is naturally found as stibnite, and can be separated either by reaction with iron, carbon, or by heating the mineral until the element splits off. 



\end{document}