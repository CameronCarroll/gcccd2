\documentclass[11pt,letterpaper]{report}
\usepackage[pdftex]{graphicx}
\usepackage[version=3]{mhchem}
\usepackage{tabularx}
\usepackage{latexsym}
\usepackage{multirow}

\newcommand{\HRule}{\rule{\linewidth}{0.5mm}}
\setlength{\topmargin}{-.7in}
\setlength{\leftmargin}{-.7in}
\setlength{\textheight}{9in}
\setlength{\oddsidemargin}{0in}
\setlength{\textwidth}{6.25in}



%_{()}

%\multicolumn{4}{|l|}{\ce{}} \\

\begin{document} 

\input{./copper_coversheet}

\section*{Objective} 
This experiment is a quick tour of and introduction to different reaction types. In the process of transmuting copper to and from other substances, double displacement, redox, acid-base and decomposition reactions occur. This experiment teaches the student about different signs and stages of reactions. In addition, the student explores the law of conservation of mass as copper is changed into various other substances and back into copper.

\section*{Procedure:}
Lehman, J; Olmstead, T; Vance, D, et alia. (2011)
Experiment 7: Copper Reactions
Chemistry 141 Lab Manual (Edition 5.2) (Pages 57-64)

\section*{Hazardous Material and Safety Notes:}
\begin{itemize}
\item Closed-toe shoes and safety goggles are required at all times. \\[-0.60cm]
\item Hoods vents should be on and perform reaction slowly when adding \ce{HNO3} to \ce{Cu} \\[-0.60cm]
\end{itemize}

\section*{Reaction List}
\begin{enumerate}
\item \ce{Cu_{(s)} + 4HNO3_{(aq)} -> Cu(NO3)2_{(aq)} + 2NO2_{(g)} + 2H2O_{(l)}}
\begin{itemize}
\item \textsc{Reaction Type:} Reduction - Oxidation
\item \textsc{Expected Yield:} 0.715 g \ce{Cu(NO3)2}
\item \textsc{Stoichiometry:} $0.00348 \text{ mol \ce{Cu} } ( \frac{\text{1 mol \ce{Cu(NO3)2}}}{\text{1 mol \ce{Cu}}} ) ( \frac{\text{205.55 \text{ grams} \ce{Cu(NO3)2}}}{\text{1 mol \ce{Cu(NO3)2}}} ) = 0.715 \text{ g \ce{Cu(NO3)2}}$
\end{itemize}
\item \ce{Cu^2+_{(aq)} + 2(OH)^{-}_{(aq)} -> Cu(OH)2_{(s)}}
\begin{itemize}
\item \textsc{Reaction Type:} Precipitation
\item \textsc{Expected Yield:} 0.340 grams
\item \textsc{Stoichiometry:} $0.00348 \text{ mol \ce{Cu(NO3)2} } ( \frac{\text{1 mol \ce{Cu(OH)2}}}{\text{1 mol \ce{Cu(NO3)2}}} ) ( \frac{\text{97.57 \text{ grams} \ce{Cu(OH)2}}}{\text{1 mol \ce{Cu(OH)2}}} ) = 0.340 \text{ g \ce{Cu(OH)2}}$
\end{itemize}
\item \ce{Cu(OH)2_{(s)} -> CuO_{(s)} + H2O_{(l)}}
\begin{itemize}
\item \textsc{Reaction Type:} Decomposition
\item \textsc{Expected Yield:} 0.277 grams
\item \textsc{Stoichiometry:} $0.00348 \text{ mol \ce{Cu(OH)2} } ( \frac{\text{1 mol \ce{CuO}}}{\text{1 mol \ce{Cu(OH)2}}} ) ( \frac{\text{79.55 \text{ grams} \ce{CuO}}}{\text{1 mol \ce{CuO}}} ) = 0.277 \text{ g \ce{CuO}}$
\end{itemize}
\item \ce{CuO_{(s)} + H2SO4_{(aq)} -> CuSO4_{(aq)} + H2O_{(l)}}
\begin{itemize}
\item \textsc{Reaction Type:} Acid/Base Neutralization
\item \textsc{Expected Yield:} 0.555 grams
\item \textsc{Stoichiometry:} $0.00348 \text{ mol \ce{CuO} } ( \frac{\text{1 mol \ce{CuSO4}}}{\text{1 mol \ce{CuO}}} ) ( \frac{\text{159.6 \text{ grams} \ce{CuSO4}}}{\text{1 mol \ce{CuSO4}}} ) = 0.555 \text{ g \ce{CuSO4}}$
\end{itemize}
\item \ce{3Cu^2+_{(aq)} + 2(PO4)^{3-}_{(aq)} -> Cu3(PO4)2_{(s)}}
\begin{itemize}
\item \textsc{Reaction Type:} Double Displacement
\item \textsc{Expected Yield:} 0.441 grams
\item \textsc{Stoichiometry:} $0.00348 \text{ mol \ce{CuSO4} } ( \frac{\text{1 mol \ce{Cu3(PO4)2}}}{\text{3 mol \ce{CuSO4}}} ) ( \frac{\text{380.6 \text{ grams} \ce{Cu3(PO4)2}}}{\text{1 mol \ce{Cu3(PO4)2}}} ) = 0.441 \text{ g \ce{Cu3(PO4)2}}$
\end{itemize}
\item \ce{Cu3(PO4)2_{(s)} -> 3Cu^{2+}_{(aq)} + 2(PO4)^{3-}_{(aq)}}
\begin{itemize}
\item \textsc{Reaction Type:} Double Displacement
\item \textsc{Expected Yield:} 0.468 grams
\item \textsc{Stoichiometry:}  $0.00116  \text{ mol \ce{Cu3(PO4)2} } ( \frac{\text{3 mol \ce{CuCl2}}}{\text{1 mol \ce{Cu3(PO4)2}}} ) ( \frac{\text{134.5 \text{ grams} \ce{CuCl2}}}{\text{1 mol \ce{CuCl2}}} ) = 0.468 \text{ g \ce{CuCl2}}$
\end{itemize}
\item \ce{2Al_{(s)} + Cu^2+_{(aq)} -> Cu_{(s)} + 2Al+_{(aq)}}\\
\begin{itemize}
\item \textsc{Reaction Type: Reduction - Oxidation} 
\item \textsc{Expected Yield:} 0.221 grams
\item \textsc{Stoichiometry:} $0.00348 \text{ mol \ce{CuCl2} } ( \frac{\text{1 mol \ce{Cu}}}{\text{1 mol \ce{CuCl2}}} ) ( \frac{\text{63.55 \text{ grams} \ce{Cu}}}{\text{1 mol \ce{Cu}}} ) = 0.221 \text{ g \ce{Cu}}$
\end{itemize}
\end{enumerate}


\section*{Results \& Calculations}
\begin{itemize}
\item \textsc{Expected Yield:} 0.221 grams
\item \textsc{Actual Yield:} 0.0535 grams
\item \textsc{Percent Yield:} 24.2\%
\item \textsc{Reclaimed copper was black rather than dull brown and significantly less abundant than peer samples.}
\item \textsc{Watch glass was cold when weighing copper, indicating evaporation was still occuring.}
\end{itemize}

\paragraph{Sample Calculations}\mbox{}\\
\textsc{Expected Yield:} \\
$0.00348 \text{ mol \ce{CuCl2} } ( \frac{\text{1 mol \ce{Cu}}}{\text{1 mol \ce{CuCl2}}} ) ( \frac{\text{63.55 \text{ grams} \ce{Cu}}}{\text{1 mol \ce{Cu}}} ) = 0.221 \text{ g \ce{Cu}}$\\
\textsc{Percent Yield:} \\
$Yield \% = \frac{\text{observed value}}{\text{expected value}} * 100\% = \frac{0.0535\ g}{0.221\ g} * 100\% = 24.2\%$

\section*{Discussion}
\paragraph{Copper Yield Percent}
I was expecting a yield far above 100\%, only because the lab literature keeps asking if it's reasonable or why it happened. Instead I got only one fifth  (24.2\%) of the starting copper back and even then it wasn't entirely pure copper. (The sample was black rather than the expected dull brown.) There were a couple accidents where ammonia was introduced to the copper-containing solution either too early in the experiment (losing some copper to the copper ammonium complex) and where ammonia was introduced to insufficiently rinsed copper, creating more ammonium complex. This sample was lost into the wash solution. Another contributing factor may be that after aluminum was added and dissolved, the experiment was shelved for a week which allowed all precipitate to dissolve back into the solution. The low percent yield suggests that not all of the copper was precipitated out of the solution. Finally, the watch glass was cold when presented to the instructor which indicated that there was still evaporation occuring. The weight of the unevaporated liquid and the ammonia which was added to catalyze evaporation both may have added to the sample, meaning the true yield is even lower than 24.2\%.

\section*{Conclusion}
Out of 0.2209 grams of copper at the beginning of the experiment we expect all of it to exist at the end. Only 24.2\% of the expected sample was reclaimed, giving 0.535 grams of copper.

\section*{Post-lab Questions}
\begin{enumerate}
\item $3.50 \text{kg dirty Cu } (\frac{68.3 \text{kg clean Cu}}{100 \text{kg dirty Cu}}) (7 \text{ reactions}) (\frac{90 \text{kg succeeded}}{100 \text{kg attempted}}) = 15.1 \text{kg copper wiring}$ \\
\item \ce{CuCl2 + NaNO3 -> Cu(NO3)2 + NaCl} -- add sodium nitrate (\ce{NaNO3}) \\
\item \ce{CuSO4 + 2H2O -> Cu(OH)2_{s} + H2SO4} -- add water (\ce{H2O} and separate by filtration or decanting.)
\end{enumerate}


\end{document}