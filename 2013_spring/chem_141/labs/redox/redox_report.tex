\documentclass[11pt,letterpaper]{report}
\usepackage[pdftex]{graphicx}
\usepackage[version=3]{mhchem}
\usepackage{tabularx}
\usepackage{latexsym}
\usepackage{multirow}

\newcommand{\HRule}{\rule{\linewidth}{0.5mm}}
\setlength{\topmargin}{-.7in}
\setlength{\leftmargin}{-.7in}
\setlength{\textheight}{9in}
\setlength{\oddsidemargin}{0in}
\setlength{\textwidth}{6.25in}



%_{()}

%\multicolumn{4}{|l|}{\ce{}} \\

\begin{document} 

\begin{titlepage}
\begin{center}

\textsc{\Large Lab 4}\\[1.5cm]
\textsc{\Large Grossmont College Chemistry}\\[0.5cm]
\includegraphics[width=0.15\textwidth]{./logo.jpg}

\HRule \\[0.4cm]
{ \LARGE \bfseries Report: Redox Reactions}\\[0.5cm]

\HRule \\[1.5cm]

\begin{minipage}{0.4\textwidth}
\begin{flushleft} \large
\emph{Author:}\\
Cameron \textsc{Carroll}\\[0.2cm]

\end{flushleft}
\end{minipage}
\begin{minipage}{0.4\textwidth}
\begin{flushright} \large
\emph{Instructor \& Class:}\\
Cary \textsc{Willard} - Chem 141 (6657)
\end{flushright}
\end{minipage}

\begin{center}
  \includegraphics[width=0.50\textwidth]{./redox_rubric.png}\\[0.7cm]
\end{center}

\vfill

{\large \today}

\end{center}
\end{titlepage}

\section*{Procedure:}
Lehman, J; Olmstead, T; Vance, D, et alia. (2011)
Experiment 6: Redox Reactions
Chemistry 141 Lab Manual (Edition 5.2) (Pages 49-55)


\section*{Discussion}
The only part of the activity series which I had some difficulty with was determining the place of the iron (iii) ion. There didn't appear to be enough information to place it in the series, as I could only say that it was more active than iodine. In order to determine its location I would propose doing a reaction between iron (iii) and hydrogen, perhaps by mixing an iron (iii) solution with an acid. I would also do an upper-bound check by reacting the iron (iii) ion with zinc dust and seeing whether zinc became oxidized.

\section*{Conclusion}
\paragraph{Activity Series:}
\begin{enumerate}
\item Zn \\[-0.5cm]
\item Fe \\[-0.5cm]
\item \ce{Fe^{3+}} ? \\[-0.5cm]
\item H \\[-0.5cm]
\item Cu \\[-0.5cm]
\item I \\[-0.5cm]
\item Br \\[-0.5cm]
\end{enumerate}



\end{document}