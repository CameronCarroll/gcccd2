\documentclass[11pt,letterpaper]{report}
\usepackage[pdftex]{graphicx}
\usepackage[version=3]{mhchem}
\usepackage{tabularx}
\usepackage{latexsym}
\usepackage{multirow}

\newcommand{\HRule}{\rule{\linewidth}{0.5mm}}
\setlength{\topmargin}{-.7in}
\setlength{\leftmargin}{-.7in}
\setlength{\textheight}{9in}
\setlength{\oddsidemargin}{0in}
\setlength{\textwidth}{6.25in}



%_{()}

%\multicolumn{4}{|l|}{\ce{}} \\

\begin{document}

\begin{titlepage}
\begin{center}

\textsc{\Large Lab 1}\\[1.5cm]
\textsc{\Large Grossmont College Chemistry}\\[0.5cm]
\includegraphics[width=0.15\textwidth]{./logo.jpg}

\HRule \\[0.4cm]
{ \LARGE \bfseries Report: Measuring Density with Different Types of Glassware}\\[0.5cm]
{ \large \bfseries (Standard Deviation Lab)}\\[0.5cm]

\HRule \\[1.5cm]

\begin{minipage}{0.4\textwidth}
\begin{flushleft} \large
\emph{Author:}\\
Cameron \textsc{Carroll}\\[0.2cm]

\end{flushleft}
\end{minipage}
\begin{minipage}{0.4\textwidth}
\begin{flushright} \large
\emph{Instructor \& Class:}\\
Cary \textsc{Willard} - Chem 141 (6657)
\end{flushright}
\end{minipage}

\vfill

{\large \today}

\end{center}
\end{titlepage}

\section*{Objective} 
This experiment is intended to introduce precision and accuracy in different types of glassware, as well as to introduce basic error analysis including standard deviation. The goal is to use different tools to measure the density of a liquid, and then to compare their precision and accuracy. Finally, the density of coke or diet coke is determined using what is to be determined as the `best' option of glassware.

\section*{Introduction}
\paragraph*{}
Different types of glassware are used for different purposes, and each have their own level of precision. By knowing how precise each option is, one can make an informed decision on which one to use, based on how the precision needed. A rough approximation is often enough for many purposes in the lab, but just as often the whole experiment is riding on the accuracy of the numbers: Bad data will lead to a incorrect conclusions.

\paragraph*{}
By measuring the change in mass after a specified amount of liquid is delivered from various types of glassware (beaker, graduated cylinder, and pipet,)  we can calculate the density of that liquid. Then, by calculating the standard deviation of the density values, we can compare the precision of the different glass containers. This allows for exploration of accuracy and precision and refamiliarization with the lab and glassware after a period away.


\section*{Procedure:}
Lehman, J; Olmstead, T; Vance, D, et alia. (2011)
Experiment 1: Measuring Density with Different Types of Glassware
Chemistry 141 Lab Manual (Edition 5.2) (Pages 1.1 - 1.12)

\section*{Hazardous Material and Safety Notes:}
\begin{itemize}
\item Closed-toe shoes and safety goggles are required at all times. \\[-0.5cm]
\item Cola stored for long periods of time tends to grow bacteria; In this case, all glassware should be cleaned with bleach. \\[-0.5cm]
\end{itemize}


\section*{Results \& Calculations}

\paragraph{Part I: Beaker Calibration}\mbox{}\\
\begin{tabularx}{\textwidth}{ | X| X| X|}
\hline
\textsc{Trial:} & \textsc{Water Delivered:} & \textsc{Deviation from Average:} \\
\hline
1 & 60.372 g & 0.371 g\\
2 & 59.776 g & 0.967 g\\
3 & 62.082 g & 1.34 g\\
\hline
\multicolumn{2}{|l|}{\textsc{Average Mass Delivered:}} & 60.743 g \\
\hline
\multicolumn{2}{|l|}{\textsc{Standard Deviation:}} & 1.2 g \\
\hline
\multicolumn{2}{|l|}{\textsc{Density of Water: ($18.8\,^{\circ}{\rm C}$)}} & 0.998444 $\frac{g}{mL}$ \\
\hline
\multicolumn{2}{|l|}{\textsc{Volume of Water Delivered:}} & 60.838 mL \\
\hline
\multicolumn{2}{|l|}{\textsc{Observed Volume of Water:}} & 60 mL \\
\hline
\multicolumn{2}{|l|}{\textsc{Percent Difference (Real vs Observed):}} & 1.377\% \\
\hline
\end{tabularx}

\paragraph{Part II: Graduated Cylinder Calibration}\mbox{}\\
\begin{tabularx}{\textwidth}{ | X| X| X|}
\hline
\textsc{Trial:} & \textsc{Water Delivered:} & \textsc{Deviation from Average:} \\
\hline
1 & 9.827 g & 0.271 g\\
2 & 9.950 g & 0.148 g\\
3 & 9.911 g & 0.187 g\\
4 & 9.834 g & 0.264 g\\
5 & 9.766 g & 0.332 g\\
6 & 11.3 g & 1.202 g\\
\hline
\multicolumn{2}{|l|}{\textsc{Average Mass Delivered:}} & 10.098 g \\
\hline
\multicolumn{2}{|l|}{\textsc{Standard Deviation:}} & 0.5925 g \\
\hline
\multicolumn{2}{|l|}{\textsc{Density of Water: ($18.8\,^{\circ}{\rm C}$)}} & 0.998444 $\frac{g}{mL}$ \\
\hline
\multicolumn{2}{|l|}{\textsc{Volume of Water Delivered:}} & 10.114 mL \\
\hline
\multicolumn{2}{|l|}{\textsc{Observed Volume of Water:}} & 10 mL \\
\hline
\multicolumn{2}{|l|}{\textsc{Percent Difference (Real vs Observed):}} & 1.127\% \\
\hline
\end{tabularx}

\paragraph{Part III: Pipet Calibration}\mbox{}\\
\begin{tabularx}{\textwidth}{ | X| X| X|}
\hline
\textsc{Trial:} & \textsc{Water Delivered:} & \textsc{Deviation from Average:} \\
\hline
1 & 9.881 g & 0.07525 g\\
2 & 9.922 g & 0.3425 g\\
3 & 10.046 g & 0.8975 g\\
4 & 9.976 g & 0.1975 g\\
\hline
\multicolumn{2}{|l|}{\textsc{Average Mass Delivered:}} & 9.956 g \\
\hline
\multicolumn{2}{|l|}{\textsc{Standard Deviation:}} & 0.07137 g \\
\hline
\multicolumn{2}{|l|}{\textsc{Density of Water: ($18.8\,^{\circ}{\rm C}$)}} & 0.998444 $\frac{g}{mL}$ \\
\hline
\multicolumn{2}{|l|}{\textsc{Volume of Water Delivered:}} & 9.972 mL\\
\hline
\multicolumn{2}{|l|}{\textsc{Observed Volume of Water:}} & 10 mL \\
\hline
\multicolumn{2}{|l|}{\textsc{Volume Percent Difference (Real vs Observed):}} & 0.28\% \\
\hline
\end{tabularx}

\paragraph{Part IV: Measuring Density of Coke}\mbox{}\\
\begin{tabularx}{\textwidth}{ | X| X| X|}
\hline
\textsc{Trial:} & \textsc{Calculated Density:} & \textsc{Deviation from Average:} \\
\hline
1 & 3.592 $\frac{g}{mL}$ & 1.247 $\frac{g}{mL}$\\
2 & 2.291 $\frac{g}{mL}$ & 0.0538 $\frac{g}{mL}$\\
3 & 1.856 $\frac{g}{mL}$ & 0.4888 $\frac{g}{mL}$\\
4 & 1.640 $\frac{g}{mL}$ & 0.7048 $\frac{g}{mL}$\\
\hline
\multicolumn{2}{|l|}{\textsc{Average Calculated Density:}} & 2.345 $\frac{g}{mL}$ \\
\hline
\multicolumn{2}{|l|}{\textsc{Standard Deviation:}} & 0.8745 $\frac{g}{mL}$ \\
\hline
\end{tabularx}

\paragraph{Sample Calculations}\mbox{}\\
(All samples are for part I, or the first trial of part I. Except for the cola bit, which is part IV trial 1.) \\[0.5cm]
\textsc{Deviation from Average:} \\
$|observed\ value - average\ value| = |mass\ delivered\ (\ce{H2O}) - average\ mass\ delivered| $ \\
$= |60.372 - 60.743| = 0.371$ \\[0.3cm]
\textsc{Standard Deviation:} \\
$\sigma = \sqrt{\frac{\Sigma d^2}{n-1}} = \sqrt{\frac{2.86\ g^2}{2}} = 1.20\ g$ \\[0.3cm]
\textsc{Mean:} \\
$\bar{x} = \frac{\Sigma M_n}{n} = \frac{(60.372 + 59.776 + 62.082)\ g}{3} = 60.74\ g$ \\[0.3cm]
\textsc{Density (Cola):} \\
$\rho_{cola} = \frac{mass\ cola}{volume\ cola} = \frac{35.92\ g}{10\ mL} = 3.592 \frac{g}{mL}$ \\[0.3cm]
\textsc{Volume Delivered:} \\
$V_{water} = \frac{average\ mass\ delivered\ (water)}{density\ (water)} = \frac{60.743\ g}{0.998444 \frac{g}{mL}} = 60.838\ mL$



\section*{Discussion}
\paragraph{Precision and Accuracy in the Glassware}
Looking at standard deviation from average, the pipet is considerably more precise than the other two options: 0.0714 g for the pipet compared to 0.593 g and 1.2 g for the graduated cylinder and breaker respectively. In terms of accuracy, the pipet is by far the best again at 0.28\% difference between observed and delivered volumes of water. The cylinder and beaker were close to oneanother in accuracy at 1.13\% and 1.38\% respectively. 

\paragraph{Choice of Glassware}
When I chose a device to determine the density of coke, I elected to use the graduated cylinder based on its relatively close precision to the pipet and feeling like the pipet was in constant danger of breaking in the cramped hood. It turns out that while their precision was fairly close, the accuracy of the pipet was far higher than the accuracy of the graduated cylinder, and I should have used it after all.

\paragraph{Density of Coke}
My one-sigma calculated value for the density of coke is 1.47 - 3.22 $\frac{g}{mL}$. This is a long way away from the `actual densities' from searching the web: 1.042 - 1.1 $\frac{g}{mL}$. In addition to being quite inaccurate, my results are quite inprecise as well. This could possibly be attributed to the bacteria that was found growing on top of the cola being unevenly distributed throughout the sample. 


\section*{Conclusion}
The pipet is both accurate and precise; The graduated cylinder is quite precise but a little bit less accurate; The graduated cylinder lacks in both qualities. 


\end{document}