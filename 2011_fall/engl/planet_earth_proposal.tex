\documentclass[11pt]{article}
\usepackage[left=30mm,width=150mm,top=25mm,bottom=25mm]{geometry}
\usepackage{titlesec}
\usepackage{setspace}
\usepackage{csquotes}
\usepackage[american]{babel}
\usepackage{graphicx}
\usepackage{hyperref}


\begin{document}

\thispagestyle{empty}
\begin{spacing}{1.5}
\begin{center}
{\Large \bfseries
Grossmont College}

\vspace*{10mm}

\includegraphics[width=5cm]{gc_crest}

\vspace*{15mm}
{\large \bfseries Research Proposal --- \today
}

\vspace*{10mm}



\LARGE Utterly Broken: Corruption and Environmental Mismanagement in the Energy Sector


\vspace*{20mm}

{\large \bfseries Cameron Carroll
}

\vspace*{20mm}

Instructor: \textsc{Stephanie Mood}

\end{center}
\end{spacing}

\newpage
\setcounter{page}{1}

\section*{Introduction: Research Problem}
\paragraph*{} Energy is arguably our most precious commodity, given that almost anything consists of and relies upon energy for its manufacture, transport and consumption. As a result of its position as a national resource and its modern deregulation/privatization, energy sources and technology are subject to greed, mismanagement and corruption. The identification and publication of such corruption and inefficiencies is essential to the efforts of anti-corruption and government transparency. What conditions encourage energy-sector corruption, and what ideas are there for remedy? What social and environmental impacts have been observed and correlated to this corruption? The current state of knowledge on the subject is concerned with political and social stability; While legitimate issues, there is a lack of material regarding corruption with regard to the environment.
\paragraph*{} In this project, I intend to survey the effects dishonesty and bribery have had upon our civilization, as well as the consequences of this activity upon the ecology \& natural environment. I will seek to glean the trends and attitudes of the past using a search through older books and periodicals. Contrary to this approach, I intend to use a technological basis for the modern aspect of this project; Specifically, diplomatic cables and searchable databases of publications are to be used in order to establish a scope of literature and to establish the frontiers of environmentally-based anti-corruption efforts. Finally, I intend to fuse past and present scopes together and identify similar trends, attitudes and outcomes.

\section*{Methodology}
\paragraph*{Research Methods}
 My research methods will be primarily analysis of other documents. Foremost, articles on the subject are to be sought and their argument deconstructed, which will provide a rough estimate of the existing literature. I intend to use historical materials obtained through libraries to provide the 'control group' for trends and outlook. Modern material, including political writings, diplomatic cables, and online news articles will be accounted for 'in scope,' compared against historical counterparts, and analyzed with a contemporary frame of reference. 
\paragraph*{Research Limitations}
 The nature of corruption means that it is almost always a secret, and so often such dishonesty comes to light with much scandal. The documentation of corruption is not high-quality data and should be considered as such. Similarly, articles and arguments regarding corruption and environment are scarce, and some extrapolation (or induction) must be done to reach a conclusion. Finally, the time frame of this project means that it cannot possibly be a complete survey; Rather, I hope to get a grasp on the prevailing paradigm regarding greed \& environment, as well as an idea of how that outlook has changed.

\newpage

\section*{Reading List \& Initial Review of Literature}
\paragraph{\cite{msi_ruth_02} Corruption and the Energy Sector:} Ruth argues that change comes slowly or not at all, so money greases the wheels until it does occur. Matthias Ruth is a professor of economics at the University of Maryland, and has focused his research on economic thermodynamics and the modeling of social/ecologic processes using physical laws.

\paragraph{\cite{lovei_alastair_00} Costs of Corruption for the Poor -- The Energy Sector:}
 Laszlo Lovei is the director of the Sustainable Development Department at the World Bank, and has an academic history teaching at the Budapest University of Economics. Alastair McKechnie has had a menagerie of titles including Energy Sector Director (South Asia region) and Division Chief for Energy, Infrastructure and Private-Sector Development in the North Africa/Middle East region. he is currently director for the Fragile and Conflict-Affected Countries Group at the World Bank, where he continues to provide assistance.

\paragraph{\cite{ieeespec_zorpette_11} Reengineering Afghanistan:} Glenn Zorpette is executive editor of \emph{IEEE Spectrum} and went overseas to Iraq for a year, and then later Afghanistan in order to research his articles on infrastructure reconstruction corruption. 

\paragraph{\cite{egi_wood_10} EGI at Anti-Corruption Conference:} Davida Wood, a South African native, is the project manager of the Electricity Governance Institute. She has publicly campaigned for increased transparency and accountability regarding the energy sector.

\paragraph{\cite{wri_mock_04} Corruption's Corrosive Effect on the Environment:}

\paragraph{\cite{eas_morse_06} Is Corruption Bad for Environmental Sustainability? A Cross-Nationl Analysis:}

\bibliographystyle{plain}
\bibliography{planet_earth_ref}



\end{document}