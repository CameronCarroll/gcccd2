\documentclass[12pt,letterpaper]{article}

\usepackage{ifpdf}
\usepackage{mla}
\usepackage{longtable}

\begin{document}
\begin{mla}{Cameron}{Carroll}{Prof. Mood}{English 124}{\today}{Argument Articulation: Essay Charting}

\begin{longtable}
\P 1 & Introduces the concept of war and sets the tone of the essay: That the institution of government numbs its warriors to the horror and futility of war. \\
\P 2 & Suggests that the illusion provided by institutions is not sufficient to explain the violence of war; Notes that Freud suggested that there is a "perverse desire to destroy, countering Eros and the will to live." \\
\P 3 & Concedes that institutions certainly influence and push for war, but strongly suggests that there is also a psychological, irrational push for war/violence in the individual. \\
\P 4 & An aside, noting the demarcation of the general 'introduction' (Remainder discusses individuals) \\
\P 5 & Discusses the historical values found in the warrior: Previous to the invention and adoption of ranged weaponry, ferocity and viciousness was ideal: The biggest, scariest guy with a huge axe is ideal in a fight. Due to the precision and dexterity required for ranged weapons, the ideal warrior became cool, collected \& completely unemotional. \\
\P 6 & Notes that warriors join an institution designed to make them into the ideal fighter; It is no longer enough to be able to wield a weapon. War has a connection to the institutions of government and religion in that their purposes require trained experts. The individual's requirements, which are suggested as a need for violence, extend not far beyond immediate, emotional reaction \\
\P 7 & Suggests that the notion of a 'killer instinct' is irrelevant: The institutions will take emotional, irrational human beings (Which we are naturally) and turn them into, essentially, machines. The institutions know that the brain is a pattern machine, and they know exactly what to feed into it. \\
\P 8 & Introduces the concept of the fearful warrior: Those that do not wish to fulfill whatever 'violent instinct' they may have lurking inside; Those that "melt away into the trees" or maim themselves to avoid conscription are all evidence against a violent tendency. \\
\P 9 & Points out that human beings, by their own will, usually do not wish to participate in war. The institutions know this, and force their human militia to transcend their civilian state, rewire their pattern machines and feed in only a steady stream of 'propaganda.' A negative connotation almost always, propaganda here simply means anything said or done rhetorically (or with the intent of control.) \\
\P 10 & Again, discusses the transformation or transcendence required to become a 'warrior.' Discipline, mind control, rewiring the brain; These all mean the same thing, and they are hidden under the guise of narratives and reverse-antropomorphations: Humans as ferocious animals. \\
\P 11 & Discusses the use of substance and ritual in order to assist the transformation: It is well known that a drunk is prone to violence; Hoplites and Chinese warriors would drink wine before battle. Notes that the destructive instinct must be helped along by a great deal of intoxication, lowering ones inhibitions. \\
\P 12 & Introduces the 'brute force discipline' method usually used today: By months of training and drilling, men are transformed into part of the "fighting machine." This comradery and dissociation from self helps to distance the fighter from the human. \\
\P 13 & Suggests that while humans are certainly capable of being horribly violent, war and its institutions push humans to abandon reason and morality. \\
\P 14 & Points out that some institutions do not just create warriors and leave them that way: Painful rituals have been used in Africa and South America to reintroduce warriors to society, letting them know that their efforts are not pointless, but also that they are not so important as to forsake ones own life and community. \\
\end{longtable}

\end{mla}
\end{document}