\documentclass[12pt,letterpaper]{article}

\usepackage{ifpdf}
\usepackage{mla}
\usepackage{setspace}


\begin{document}
\begin{mla}{Cameron}{Carroll}{Prof. Mood}{English 124}{\today}{Midterm Reflection}
\singlespacing

\begin{itemize}

\item Critical thinking is the practice of going deeper into an argument or claim. When someone presents information, they always have a motive. True enough, this motive is often good: Teachers intend to inform fully, truthfully and to convery understanding. Teachers less than others, but when one converys information they filter it through their own ideals and beliefs. Tone and terminology, pace and imagery are all subject to this personal filter. Critical thinking considers this bias, although that is a harsh word for the truth. Critical thinkers take into account the motives, background and syntactic choices of the speaker or writer. In this way, all arguments have some degree of invalidity or unfairness about them, but in the same way they have often very valid points. The critical thinker sorts this out and weights how valid this datum is versus that. \\

\item \begin {enumerate}

 \item I've learned that while language is almost always used for a motive, judging that language based upon that criterion is unreasonable. If everyone has a motive, surely SOMETHING must be valid. Thus, everything is valid and ridiculous, to a judgable extent. Words have long and glorious histories, throughout which their meaning has changed without mercy to previous incarnations. Entire sets of vernacular are lost to the ages, with all of their nuances and meaning. So while in some ways, the control that words exert is to be resented, the beauty of the rise and fall of their meaning as opposed to definition is to be revered. \\

\item Language is much more powerful than I previously considered it to be. My opinion still is that words and just words, and for the most part they should be allowed to flow freely. But through some phenomena, a few words have such obvious disvalue that they come public taboo. Through other phenomena, phrases carry meaning that they literally no longer associate with. My point is that language affects us a lot more deeply than I had a respect for. \\

\item I've learned that region and culture make a huge difference in language, which makes a difference in the person. For example, asian cultures have much shorter 'counting numbers,' which allows them to memorize more arbitrary digits than an equivalent westerner, on average. This partially explains the excellence at math, but also applies to language. Certain aspects of Japanese culture give them reason to create a word for which we can only construct a rough phrase. \\

\end{enumerate}

\item I hope to learn how better to stay focused on my topic, and how to better articulate my own arguments. I find that often while writing, I flock to other ideas and stray easily from the course. I also want to become more concise with my wording, striving to something akin to Orwellian English. Finally, I feel like my rhetorical skills need some work: I would prefer to be able to write a good convincing article that doesn't give away its position. Not necessarily because I like influencing others, but just because I feel like to be fully cultured one must be able to truly wield the language.


\end{itemize}


\end{mla}
\end{document}