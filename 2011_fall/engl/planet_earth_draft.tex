\documentclass[12pt,letterpaper]{article}

\usepackage{ifpdf}
\usepackage{mla}
\usepackage[american]{babel}
\usepackage{csquotes}
\usepackage{setspace}
\usepackage{hyperref}



\begin{document}
\begin{mla}{Cameron}{Carroll}{Prof. Mood}{English 124}{\today}{Planet Earth Paper: Notes \& Workspace}
\vspace{0.5cm}
\hrule
\vspace{0.5cm}
\singlespacing

\begin{abstract}
Energy is arguably our most precious commodity, given that almost anything consists of and relies upon energy for its manufacture, transport and consumption. As a result of its position as a national resource and its modern deregulation/privatization, energy sources and technology are subject to greed, mismanagement and corruption. The identification and publication of such corruption and inefficiencies is essential to repairing the system; What conditions encourage energy-sector corruption? What social and environmental impacts does unchecked corruption have? The current state of knowledge on the subject is concerned with political and social stability; While legitimate issues, there is a lack of material regarding corruption with regard to the environment. This proposal embodies a survey of contemporary issues in energy-sector corruption, as well as extrapolation of those issues into environmental concerns.
\end{abstract}

\section*{References \& Quotations}
``The key role that the energy sector plays in society and economy make it vulnerable to corruption. Individuals in the public sector may find opportunities for personal gain due to the powers they hold over access to, and transformation and distribution of energy, including procurement for the development, operation and maintenance of energy system components.'' (Ruth 2002)

``Extraction of fossil fuels and radioactive materials for nuclear power generation and transformation of these extracted materals into usable forms of energy are typically carried out in large scale at a few locations. Similarly, locations with ample river flow for hydroelectricity, sufficient solar influx for photovoltaic conversion into electricity, geothermal vents for district heating or electricity generation, and significant amounts of biomass for energy conversion are limited. '' (Ruth 2002)

``Because of the growing need for energy and limited sources of supply, governments have both natural monopolies in the energy sector and significant interests in protecting their energy supply from disruption... Within this energy scenario, ample opportunities exist for high profits and resource rents, and for individuals to engage in corrupt practices to gain access to, or use, the power associated with access to energy.'' (Ruth 2002)

``Countries with strong private interests and political and economic competition are susceptible to \emph{interest group bidding} that is largely non-systematic and carried out on an individual basis. In contrast, countries with limited political competition and an \emph{elite hegemony} may be susceptible to individuals and groups selling political access to enrich themselves at the cost of the state.'' (Ruth 2002)

`` Individuals in the public sector may find opportunities for personal gain due to the powers that they hold over access to, and transformation and distribution of energy, including procurement for the development, operation and maintenance of energy system components.'' (Ruth 2002)

``...lack of adequate supply, cumbersome paperwork, non-computerized databases, and large backlog make petty corruption more prevalent in developing countries that are overwhelmed with demand and are relatively less regulated. '' (Ruth 2002)

``The main areas of corruption in the distribution of energy – and electricity in particular – include, among others, non-technical system loss (e.g. falsified meter readings, altered invoices and illegal purchases); interference in the flow of funds/barter/offsets within the system and to fuel suppliers; manipulation of the flows of electricity to favored customers; and opaque uneconomic import arrangements. '' (Ruth 2002)

``Corruption occurs in many different forms, depending on features of the supply chain of each specific energy source, the significance of that specific energy source in the local and national economy, the sociopolitical and institutional context within which extraction, transformation and use occur, the number of individuals participating in decision making and the cultural environment within which decisions are made, and the transparency of those decisions and accounting methods, as well as a lack of effectiveness of legal systems to sanction abuse of power by decision makers.'' (Ruth 2002)




`` China activated the official accountability system during the SARS (severe acute respiratory syndrome) crisis in 2003. More than1,000 officials, including then Health Minister Zhang Wenkang and Beijing Mayor Meng Xuenong, were ousted for their attempts to cover up the epidemic situation or incompetence in SARS prevention and control.'' (Jingzhong \& Sulei)


% Don't forget that you have to introduce these two with Xinhua news agency, otherwise it's a ghost citation.
``Experts estimate that around 100 tons of pollutants containing benzene has flown into the Songhua River, caused by a Nov. 13 explosion at an upstream chemical plant of the Jilin Petrochemical Company under the China National Petroleum Corp.'' (``China Pledges'')

``he content of nitrobenzene in the polluted water exceeded the national safe standard by 29.9 times and that of benzene 2.6 times when it passed Zhaoyuan from Nov. 20 to 22. When the polluted water flowed further downstream, the content of nitrobenzene was reduced further to only about 10.7 times of above the national standard and that of benzene 0.08 times of the national standard, he said.'' (``China Pledges'')

% Ghost citation: BBC News
``BBC Beijing correspondent Louisa Lim says residents of Harbin distrust government statements, having originally been told the stoppage was for routine maintenance.'' (``Toxic Leak'')

``The Beijing News showed pictures of dead fish washed up on the banks of the Songhua river near Jilin city, but the authorities said there was no sign that chemicals in the river had contaminated the water supply. '' (`` Toxic Leak'')

`` Police are investigating the cause of death after Wang Wei, 43, was found on Tuesday at his home, officials said.'' (``Chinese toxic blast'')

``Mr Wang had been responsible for dealing with the aftermath of the 13 November explosion at a chemical plant in Jilin.
Two days later he was quoted by the China Business News as saying: ``It will not cause large-scale pollution. We have decided not to have a large-scale evacuation.'' (``Chinese Toxic Blast'')





"The fact that 110 gigawatts of installed capacity is ``illegal'' means neither that the plants are hidden in a closet nor that they lack any governmental oversight. What it does mean is that they are not part of a coherent national policy, that they frequently operate outside of national standards, and that they often evade control even by their ostensible owner at the national corporate level." (Lester \& Steinfield 36)

"Until the late 1970s tailings were indiscrimately dumped near mills and along rivers and were even used for home and building construction. In Durango, Colorado, for example, tailings were piled on the edge of town alongside the Animas River. In Grand Junction, Colorado, several hundred thousand tons of radioactive snd from a uranium mill was placed under the foundations of 4000 new homes. Residents of these homes are exposed to radiation equivalent to ten chest X-rays per week, and the leukemia rate in Grand Junction is two times higher than it is elsewhere int he state." (Chiras 333)


"... approximately one half of total system losses (amounting to an estimated 100 million dollars) of the Bangladesh Power Development Board and the Dhaka Electricity Supply Authority are accounted for by mismanagement and falsified meter reading." (Ruth 5)

\section*{Supporting Topics to be Explored:}
\begin{itemize}
\item The tendency for nuclear power plants to stay in production long past their expiration dates is a consequence of deregulation, which is a purchasable asset. Chernobyl and Fukushima are both strong examples of the effect outdated nuclear power plants have upon the environment. Think irradiated forest, corium burrowing down into the Japanese bedrock, irradiated fish. Think liquidators.
\item Unreasonably low rates and high rates of theft are driving some energy infrastructures into bankruptcy. This often leads to illegal, private generation of power which, as detailed in Zorpette's 2011 article.
\item Military needs and private interests often shape the types of power generation that are built, often using funds intended for sustainable development. Examples are US petrol turbine plants in Afghan/Iraq instead of hydro or geothermal (zorpette) and  gas, coal, nuclear investments in Thailand while largely ignoring renewable resources. (wood) These investments lead societies down a slippery slope of fossil fuels and nuclear plants at the expense of the environment. (CO2 release, all the messiness involved in harvesting/transporting/refining oil.) Can also talk about the research put into these dead-end technologies/resources, and the great deal of effort to make them seem appealing.
\end{itemize}

\section{Regulation in the Nuclear Industry}
\paragraph{Start writing!}
If there were ever a misconception that nuclear energy is clean, that has been dissipated by Fukushima and Chernobyl. In the shadow of these catastrophes, 


\section{Outline}
couple of "case studies" in corruption: Nuclear industry, middle-eastern whatevers going on, and chinese whatevers.

introduction: talk about petty vs grand corruption, differences between socioeconomic and political envionments



major topics:

nuclear industry: talk about its vulnerability to corruption \& deregulation, its history (the mill tailings under houses thing.) Chernobyl vs Fukushima for historical/contemporary basis. Talk about how industry and government at Fukushima took every available opportunity to lie about that situation. Investigate any coverup of chernobyl, whether reactor was outdated, etc.

china's problems: illegal power plants operate outside the jursdiction of government regulation, and therefore pollute considerably more. Look at china setting environmental and social concerns on the shelf in favor of economic growth.
Chinese benzene incident, where initial lies put public safety in jeopardy, as well as delayed any possible cleanup attempt. Public mistrust of the government, combined with the disconnect between head and hand (communist government versus local governments)

middle-east: US building big ass petrol plants instead of geothermal/hydro (what would actually fit in afghanistan/iraq.) Look at distribution of electricity in Sadr city (mostly to the wealthy) and illegal generation. Look at the unsustainably low prices of electricity and how it's destroying the industry




\begin{workscited}

\section*{\small{China}}

\bibent
Lester, Richard; Steinfield, Edward. ``China's Real Energy Crisis'' \underline{Harvard Asia Pacific Review.} Winter 2007. Web. \url{<http://www.hcs.harvard.edu/~hapr/winter07_gov/lester.pdf>}

\bibent
Jingzhong, Wang; Sulei, Tian. ``Environmental chief sacked following major pollution.'' \underline{Xinhua News Agency.} 5 Dec. 2005. 1 Nov. 2011. Web. \url{<http://news.xinhuanet.com/english/2005-12/02/content_3870384.htm>} 

\bibent
``China pledges to minimize impact of river pollution on Russia.'' \underline{Xinhua News Agency} 24 Nov. 2005. 1 Nov. 2011. Web.
\url{<http://news.xinhuanet.com/english/2005-11/24/content_3831641.htm>}

\bibent
``Toxic leak threat to Chinese city.'' \underline{BBC News.} 23 Nov. 2005. 1 Nov. 2011. Web. \url{<http://news.bbc.co.uk/2/hi/asia-pacific/4462760.stm>}

\bibent
``Chinese toxic blast official dead.'' \underline{BBC News.} 7 Dec. 2005. 1 Nov. 2011. Web.
\url{<http://news.bbc.co.uk/2/hi/asia-pacific/4505650.stm>}

\section*{\small{Middle East}}


\bibent
Zorpette, Glenn. ``Reengineering Afghanistan.'' IEEE Spectrum, October 2011. 1 Nov 2011. Web. \url{<http://spectrum.ieee.org/energy/the-smarter-grid/reengineering-afghanistan/0>}

\bibent
Zorpette, Glenn.  ``Reengineering Iraq.'' IEEE Spectrum, February 2006. 1 Nov 2011. Web. \url{<http://spectrum.ieee.org/energy/fossil-fuels/reengineering-iraq/0>}

\section*{\small{Nuclear Industry}}

\bibent
Chiras, Daniel. \emph{Environmental Science: A Framework for Decisionmaking.} Menlo Park, California. Benjamin/Cummings Publishing Company, 1985. Print.

\bibent
``Grand Junction (Climax Uranium) Mill Site'' \underline{U.S. Energy Information Administration.} 10/28/2011. \url{<http://www.eia.gov/cneaf/nuclear/page/umtra/grandjunction_title1.html>}

\section*{\small{General}}

\bibent
Ruth, Matthias. ``Corruption and the Energy Sector." Washington, DC: Management Systems International \& USAID, 2002. 1 November 2011. Web. \url{<http://pdf.usaid.gov/pdf_docs/PNACT875.pdf>}

\bibent
Hooker, John. ``A cross-cultural view of corruption.'' \emph{Ethisphere.} 2 Nov. 2009. 1 Nov. 2011. Web. \url{<http://ethisphere.com/a-cross-cultural-view-of-corruption/>}

\bibent
Wood, Davida. ``Electricity Governance Initiative at the 14th International Anti-Corruption Conference.'' Dec. 2010. 1 Nov. 2011. Web. \url{<http://electricitygovernance.wri.org/news/2010/12/egi-14th-international-anti-corruption-conference>}

\bibent
Mock, Greg. ``Corruption's Corrosive Effect on the Environment.'' April, 2004. 1 Nov. 2011. Web. \url{<http://www.wri.org/publication/content/7813>}

\bibent
Morse, Stephen. ``Is corruption bad for environmental sustainability? A cross-national analysis." 2006. 1 Nov 2011. Web. \url{<http://www.ecologyandsociety.org/vol11/iss1/art22/>}

\bibent
Lovei, Laszlo; McKechnie, Alastair. ``The costs of corruption for the poor -- The Energy Sector.'' Public Policy for the Private Sector, April, 2000. 1 Nov. 2011. Web. \url{<http://cdi.mecon.gov.ar/biblio/docelec/bm/ppps/N207.pdf>}




\bibent
Regenstein, Lewis. \emph{The Politics of Extinction.} New York, New York. Macmillan Publishing Co., Inc, 1975. Print.



\end{workscited}

\section*{Review of Literature / Reading List}
\paragraph{\cite{msi_ruth_02} Corruption and the Energy Sector:} This paper suggests that because energy is fundamental to almost all socio-economic activity, it is subject to a great deal of corruption. Ranging from purported "grand corruption" of the highest level of office down to "petty corruption" between two pawns in the system, this bribery and misguidance stems from the relative lethargy of the government system. Ruth argues that change comes slowly or not at all, so money greases the wheels until it does occur. Matthias Ruth is a professor of economics at the University of Maryland, and has focused his research on economic thermodynamics and the modeling of social/ecologic processes using physical laws.

\paragraph{\cite{lovei_alastair_00} Costs of Corruption for the Poor -- The Energy Sector:}
In their paper, Lovei and McKechnie postulate that corruption in the energy sector, regardless of scale, brings its consequences down squarely upon the poor. They propose that ``the answer to corruption is continuing reform, to reduce the incentive and potential to capture monopoly rents and to increase the transparency of... decisionmaking processes." (1) Laszlo Lovei is the director of the Sustainable Development Department at the World Bank, and has an academic history teaching at the Budapest University of Economics. Alastair McKechnie has had a menagerie of titles including Energy Sector Director (South Asia region) and Division Chief for Energy, Infrastructure and Private-Sector Development in the North Africa/Middle East region. he is currently director for the Fragile and Conflict-Affected Countries Group at the World Bank, where he continues to provide assistance.

\paragraph{\cite{ieeespec_zorpette_11} Reengineering Afghanistan:}

\paragraph{\cite{egi_wood_10} EGI at Anti-Corruption Conference:}

\paragraph{Environmental Science: Framework for Decisionmaking}

\bibliographystyle{plain}
\bibliography{planet_earth_ref}

\end{mla}
\end{document}