\begin{mla}{Cameron}{Carroll}{Prof. Mood}{English 124}{\today}{Utterly Broken}
\vspace{3mm}

\begin{fancyquotes}
 Pity the dispatch operators of distribution substations all over Iraq. A typical substation might have dozens of feeders, each of which might supply power to a large neighborhood. There isn't nearly enough power to keep all the feeders on at once, and it is the dispatch operator's job to turn feeders on and off in a consistent daily pattern, energizing some neighborhoods and blacking out others. By simply doing their jobs, they make countless people suspicious and angry, every day.

 In theory, this daily pattern is set by bureaucrats at the Ministry of Electricity, who telephone or radio instructions to the substation operators. In reality, the situation is murky. ``We've heard stories of guys going into substations with guns and holding them up to distribution engineers' heads and saying, `This is what you're going to do,''' says the embassy source. ``We've heard of distribution engineers' being shot.'' And there are other, less extreme pressures on the dispatch operators: local sheikhs, city council members, and provincial government officials have all been known to exert their influence. And, of course, bribery of operators is not inconceivable. ``Guaranteed, some neighborhoods are getting 20 hours of electricity a day and others are getting 6.'' the embassy source says. \textsc{(Zorpette 2006)}
\end{fancyquotes}
\paragraph{} In this passage from Glenn Zorpette's article, "Reengineering Iraq," he notes that bribery of the operators is not inconceivable; his decision to include monetary incentive as almost an afterthought illustrates the greater forces at play in this unstable region. In contrast to the violent form that influence takes in the Middle East, Asian corruption focuses on money as incentive for everything. For an example: "Korean businessmen who are granted a meeting with an important official may bring an envelope full of cash, as a token of gratitude." (Hooker) In order to establish a working definition of corruption, it is important to make a distinction between cultural differences. Matthias Ruth, in his 2002 paper ``Corruption and the Energy Sector,'' asserts that ``Countries with strong private interests and political and economic competition are susceptible to \emph{interest group bidding} that is largely non-systematic and carried out on an individual basis. In contrast, countries with limited political competition and an \emph{elite hegemony} may be susceptible to individuals and groups selling political access to enrich themselves at the cost of the state.'' He goes on to say that ``Corruption occurs in many different forms depending on features of the supply chain of each specific energy source, the significance of that specific energy source in the local and national economy, [and] the sociopolitical and institutional context...'' Given these variables, providing one definition of corruption to suit all contexts is not possible; Instead, each cultural context will be explored as they are introduced. Corruption will be defined, \emph{within cultural boundaries,} as personal gain at public expense.

 Central to the idea of energy corruption are the notions expressed in Garret Hardin's ``The Tragedy of the Commons.'' Written in 1968, this work postulates that natural resources are finite, whereas demand for those resources can be infinite. This taxation becomes unsustainable and causes the resource to diminish permanently. This idea can be extrapolated both to the pollution of the environment as well as the pollution of the political or economic process protecting that environment.

\vspace{5mm}
%CHINA 
\paragraph{} China has one of the fastest growing economies in the world, and is poised to become a world superpower, but their booming economy requires immense quantities of power for both industry and residential use, especially while foreign deadlines put pressure on manufacturing output. From the outside, the solidarity of communist China seems formidable; Inside, however, regulation is spotty and sometimes corrupt, the state energy system is fragmented, and environmental protection takes a backseat to almost every other concern. 

 A highly visible result of official corruption in China is the amount of illegal power generation in operation; ``...of the 440 gigawatts of generating capacity in place at the beginning of 2005, there were about 110 gigawatts of `illegal' power plants, plants that never received construction approval from the responsible central government agency.'' say Lester \& Steinfield in the Winter 2007 \emph{Harvard Asia Pacific Review.} They go on to note that ``the fact that 110 gigawatts of installed capacity is `illegal' means neither that the plants are hidden in a closet nor that they lack any governmental oversight. What it does mean is that they are not part of a coherent national policy, that they frequently operate outside of national standards, and that they often evade control even by their ostensible owner at the national corporate level.'' A particularly brutal case ``involves Sun Xiaojun, party chief in Hewan village (Jiangsu), who was arrested for hiring a gang of some 200 thugs to attack local farmers to force them off their land in order to build a petrochemical plant. '' (`AsiaNews') These illegal power plants and land seizures are not isolated incidents; ``Illegal small chemical plants, paper and leather mills are still beng set up. Many outdated technologies, which should have been replaced, are still in use.'' (`Corruption to Blame') These older technologies are less efficient, and illegal setups have very little incentive to follow industry regulations for environmental protection. 

Businesses themselves are not always the source of facilitator of corruption. ``If the locality's main goal is to achieve economic growth, and cheap electric power is needed to fuel that growth, then environmental enforcement will play a secondary role, a situation undoubtedly related to the initial Songhua River chemical spill and the subsequent effort to cover up that spill. Local environmental officials who take a different view are likely to run into career difficulties.'' This second incident, at the Songhua River, illustrates the disregard for environmental and often personal safety in favor of growth and energy. Xinhua News reported that  ``experts estimate that around 100 tons of pollutants containing benzene has flown into the Songhua River, caused by a Nov. 13 explosion at an upstream chemical plant of the Jilin Petrochemical Company under the China National Petroleum Corp.'' (`China Pledges') While this incident itself may or may not have had any corrupt influences, the aftermath reveals lies and egregious inaction: When Wang Wei was found dead at his home, BBC News reported that ``Mr. Wang had been responsible for dealing with the aftermath of the 13 November explosion at a chemical plant in Jilin. Two days later he was quoted by the China Business News as saying: `It will not cause large-scale pollution. We have decided not to have a large-scale evacuation.' (`Chinese Toxic Blast') Disgraced Chinese officials tend to wind up either dead or imprisoned, and here is no exception. Given that even downstream of the spill the `content of nitrobenzene was reduced further to only about 10.7 times of above the national standard,' it is clear that there \emph{was} large scale pollution, and hence the death of Wang Wei. In only this single incident, corruption can be observed at the commercial/regulatory level, and at the low-level official level. Considering that ``the Beijing News showed pictures of dead fish washed up on the banks of the Songhua river near Jilin city, but the authorities said there was no sign that chemicals in the river had contaminated the water supply,'' (`Toxic Leak') corruption is observable also at the higher official level. Of course, ``residents of Harbin distrust government statements, having originally been told the stoppage was for routine maintenance.'' (`Toxic Leak')

This small slice of China's history in corruption and environment is sufficient to draw some conclusions about Chinese behavior. First, while there are a great number of scandals and disasters in China which have been attributed to corruption, the overall attitude of the state \emph{and the people} leans toward very heavy punishment: ``Huang Songyou, a former deputy chief justice, was sentenced to life in prison for taking bribes between 2005 and 2008 for a total of 3.9 million yuan (US\$ 574,000)'' (`AsiaNews') Despite life sentences, death sentences, and the ethics classes throughout school, bribery and nepotism remain viable and lucrative options for Chinese officials. Second, there exists a distance betwixt local governments and central Chinese government; Local development is almost completely controlled by local officials, via bank and bureaucracy, enabling phenomena like illegal power plants: Financing, planning, construction are all carried out by local authorities, with no communication with any central authority. These ideas are themselves tragedies of the commons: All of the Chinese officials taking their millions (of Yuan) and running out of country are clearly benefiting themselves at the expense of society. The businesses and local governments that are so starved for power take matters into their own hands, and end up polluting heavily and unapologetically. Succinctly put, ``because of the growing need for energy and limited sources of supply, governments have both natural monopolies in the energy sector and significant interests in protecting their energy supply from disruption... Within this energy scenario, ample opportunities exist for high profits and resource rents, and for individuals to engage in corrupt practices to gain access to, or use, the power associated with access to energy.'' (Ruth 2002)

\vspace{5mm}
%MIDDLEAST

\paragraph{} A second set of cases to study come from Iraq and Afghanistan, where American efforts are rebuilding what they themselves destroyed only a few years before. Afghanistan, a mountainous terrain, has many rivers and a plethora of hydroelectric opportunities. Iraq, on the other hand, boasts impressive natural gas and oil resources... but neither of these countries are truly taking advantage of their ideal energy source. Domestic corruption \& terrorism, combined with foreign aid and terrorism all help to steer these war torn nations toward unsustainable fossil fuels.

 Iraq has extensive treasure in its oil fields, which produce both crude oil and natural gas; One account from Iraq mentions that they ``notice several towering flare stacks across the street from the power plant, at an oil field called East Baghdad. Atop one of the stacks, an enormous orange flame indicates that natural gas pouring out of the oil deposits is being burned off steadily to keep it from exploding... The flaring is notable because if all that gas were captured, pressurized, and distributed rather than being burned off, it could be used to meet more than half of Iraq's demand for electricity. At the moment, Iraq is flaring more than 28 million cubic meters of gas a day. It's enough to fire at least 4000 MW of electricity. '' (Zorpette 2006) The irony of this is striking: The aforementioned power plant runs diesel fuel trucked in from Turkey, or much worse derivatives of crude oil. These are dirty and inefficient compared to the clean-burning natural gas that the combustion turbines were designed to use, requiring more fuel, more maintenance, additive chemicals and greater emissions. The seeming perpetual wars in this region as well as a dysfunctional Iraqi financial earmarking system has shelved the system required to harvest the natural gas across the street, just a fraction of which would be required for this entire plant to run all of its turbines. ``A number of events in Iraqi history have hampered
development of the oil and natural gas industry: the war with Iran from 1980 to 1988, the invasion of Kuwait in 1990 and the 1st Gulf War, the UN sanctions imposed between 1990 and 2003 and, finally, the 2nd Gulf War in 2003 and subsequent political instability.'' (Saniere \& Sabathier) The reason that this seeming unlimited energy source is flared rather than harvested is simply that there is too much turmoil in the region: Any infrastructure leftover from before the wars began is now dilapidated and unusable. Iraq's own petroleum administration only sells the worst derivatives of oil to the internal energy ministry; This crude fuel, combined with great difficulty in construction or repair of infrastructure, leaves Iraq's energy generation crippled. Power generation, as a result of the decisions of Iraqi and American decisions alike, is either pollutive or pollutive and expensive given undesirable fuels and imported diesel respectively. In a multifaceted tragedy, foreign efforts to rebuild have failed, local efforts have failed, and anything leftover from before the wars is certainly dilapidated and nigh unusable.

Looking to the East, to Afghanistan, one finds only more problems with Afghani power infrastructure and the American reconstruction effort. In Zorpette's 2011 ``Reengineering Afghanistan,'' he says that ``the Afghan national electric utility is unable to collect enough revenue to sustain its own operations and it has trouble simply keeping records consistently.'' A refusal to restructure its payment platform, combined with difficulty accounting for and metering electricity usage prevents the national utility from profiting. ``It [the Afghan national utility] charges about 6 to 10 Afghanis (about 12 to 20 U.S. cents) per kilowatt-hour, a third of what it would need simply to break even.'' (Zorpette 2011) 

The administrative woes, however, are not even the extent of the problems... the region's optimal energy solution, hydroelectricity, thrives given the mountainous terrain. Southern Afghanistan's largest hydroelectric power station, Kajaki, was the objective of a 2008 mission involving thousands of soldiers, hundreds of vehicles. The goal was simply to deliver replacement parts to the Kajaki power station, but ``not only did the Chinese engineers contracted flee due to security concerns, but the 500 tons of cement needed have not been brought up'' ('What went wrong?') In addition, the infrastructure is not adequate to handle the extra power, and would have to be upgraded. The Kajaki plant is deep in Taliban territory, and is subject to their influence. ```Its very easy for the Taliban to control electricity because the transmission cables cross the districts where they are in total control.' says Ahlullah Obaidi, the Helmand government's director of electricity and water. `We don't cut power to their areas, and we let them collect all the money there.' '' (`U.S. Rebuilds') Whether the Taliban presence is corruption or lawbreaking is not of much consequence, as they run their own electricity business on the side, supported by the Taliban government. ``The unfortunte reality in Helmand is that there are two governments, the official one and the Taliban one, and both of them have electricity departments.'' (`U.S. Rebuilds')

The fact that Afghanistan has geothermal and hydroelectric resources is of no consequence to those in charge of such matters as building power plants... The U.S. completed a huge diesel power plant near Kabul in 2010, despite diesel's requirement of being shipped in at great expense. ``The plant consists of 18 diesel-generator sets that together can generate 105 megawatts. It is seldom used because its operating costs, around 42 cents per kilowatt-hour, are at least six times the price of electricity available from other sources.'' (Zorpette 2011) This plant was built as part of a political bid, as well as a requirement for short-term energy at any possible cost from the U.S. military. It is apparent that while U.S. have made an effort to build in Iraq, their concerns are political and military as opposed to social and environmental. 

In both Iraq and Afghanistan, U.S. companies, contractors and agencies have continually proven their inability to operate in the region: Strong local resistance, corrupt and/or incompetent contractors, and misguided American agencies have stalled any real energy infrastructure progress that could be made in either region. Individual generation concerns present inefficiency and pollution, while bribery and insurgent influence reallocate power to some favored areas. Continued domestic terrorism occurs, helping to slow any progess: ``Insurgents were blowing up electrical transmission towers at an average rate of two a day this past August, and Iraqi workers and foreign contractors were risking their lives to put them back up.'' (Zorpette 2006) The problems are expanded upon by foreign presence, but it seems that they both have enough internal strife that energy is a problem, with or without foreign aid. Whether or not they would be better on their own is neither in scope, or really answerable at all.

\paragraph{} 
Corruption has a long history, given its moral implications, and does not appear to be diminishing in any form. Coming from the Latin \emph{corruptio,} and given to mean to abuse or destroy, or literally ``utterly broken.'' (Wiki: Corruption) Matthias Ruth wrote that ``the main areas of corruption in the distribution of energy – and electricity in particular – include, among others, non-technical system loss (e.g. falsified meter readings, altered invoices and illegal purchases); interference in the flow of funds/barter/offsets within the system and to fuel suppliers; manipulation of the flows of electricity to favored customers; and opaque uneconomic import arrangements. '' These and many other niches of corruption are easily observable in the information coming out of Iraq, Afghanistan and China. While these countries are perceived to be more corrupt than Western states, their own cultural and socioeconomic issues are important to account for. Desperate for power, environmental issues are simply not important and money tends to `cut through the red tape.'



\begin{workscited}

\bibent
Trofimov, Yaroslav. ``U.S. Rebuilds Power Plant, Taliban Reap a Windfall.'' 13 July 2010. 8 Nov 2011. Web. \url{<http://online.wsj.com/article/SB10001424052748704545004575352994242747012.html>}

\bibent
Urban, Mark. ``What went wrong with Afghanistan Kajaki power project?'' 27 June 2011. 8 Nov. 2011. Web. \url{<http://www.bbc.co.uk/news/13925886>}

\bibent
Saniere, Armelle; Sabathier, Jerome. ``Iraq: making its return to the oil and natural gas markets.'' Nov. 2010. 8 Nov. 2011. Web. 

\bibent
Lester, Richard; Steinfield, Edward. ``China's Real Energy Crisis'' \underline{Harvard Asia Pacific Review.} Winter 2007. Web. \url{<http://www.hcs.harvard.edu/~hapr/winter07_gov/lester.pdf>}

\bibent
Jingzhong, Wang; Sulei, Tian. ``Environmental chief sacked following major pollution.'' \underline{Xinhua News Agency.} 5 Dec. 2005. 1 Nov. 2011. Web. \url{<http://news.xinhuanet.com/english/2005-12/02/content_3870384.htm>} 

\bibent
``China pledges to minimize impact of river pollution on Russia.'' \underline{Xinhua News Agency} 24 Nov. 2005. 1 Nov. 2011. Web.
\url{<http://news.xinhuanet.com/english/2005-11/24/content_3831641.htm>}

\bibent
``Toxic leak threat to Chinese city.'' \underline{BBC News.} 23 Nov. 2005. 1 Nov. 2011. Web. \url{<http://news.bbc.co.uk/2/hi/asia-pacific/4462760.stm>}

\bibent
``Chinese toxic blast official dead.'' \underline{BBC News.} 7 Dec. 2005. 1 Nov. 2011. Web.
\url{<http://news.bbc.co.uk/2/hi/asia-pacific/4505650.stm>}

\bibent
``Corruption in China: Communist official and Supreme Court justice imprisoned.'' \underline{AsiaNews.} 19 Jan. 2010. 4 Nov. 2011. Web. \url{<http://www.asianews.it/news-en/Corruption-in-China:-Communist-official-and-Supreme-Court-justice-imprisoned-17388.html>}

\bibent
``Corruption to Blame for China's Worsening Pollution, Official Says.'' \underline{Associated Press.} 26 Dec 2006. 6 Nov. 2011. Web. \url{<http://www.enn.com/top_stories/article/5711>}

\bibent
Zorpette, Glenn. ``Reengineering Afghanistan.'' IEEE Spectrum, October 2011. 1 Nov 2011. Web. \url{<http://spectrum.ieee.org/energy/the-smarter-grid/reengineering-afghanistan/0>}

\bibent
Zorpette, Glenn.  ``Reengineering Iraq.'' IEEE Spectrum, February 2006. 1 Nov 2011. Web. \url{<http://spectrum.ieee.org/energy/fossil-fuels/reengineering-iraq/0>}	

\bibent
Chiras, Daniel. \emph{Environmental Science: A Framework for Decisionmaking.} Menlo Park, California. Benjamin/Cummings Publishing Company, 1985. Print.

\bibent
``Grand Junction (Climax Uranium) Mill Site'' \underline{U.S. Energy Information Administration.} 10/28/2011. \url{<http://www.eia.gov/cneaf/nuclear/page/umtra/grandjunction_title1.html>}

\bibent
Ruth, Matthias. ``Corruption and the Energy Sector." Washington, DC: Management Systems International \& USAID, 2002. 1 November 2011. Web. \url{<http://pdf.usaid.gov/pdf_docs/PNACT875.pdf>}

\bibent
Hooker, John. ``A cross-cultural view of corruption.'' \emph{Ethisphere.} 2 Nov. 2009. 1 Nov. 2011. Web. \url{<http://ethisphere.com/a-cross-cultural-view-of-corruption/>}

\bibent
Wood, Davida. ``Electricity Governance Initiative at the 14th International Anti-Corruption Conference.'' Dec. 2010. 1 Nov. 2011. Web. \url{<http://electricitygovernance.wri.org/news/2010/12/egi-14th-international-anti-corruption-conference>}

\bibent
Mock, Greg. ``Corruption's Corrosive Effect on the Environment.'' April, 2004. 1 Nov. 2011. Web. \url{<http://www.wri.org/publication/content/7813>}

\bibent
Morse, Stephen. ``Is corruption bad for environmental sustainability? A cross-national analysis." 2006. 1 Nov 2011. Web. \url{<http://www.ecologyandsociety.org/vol11/iss1/art22/>}

\bibent
Lovei, Laszlo; McKechnie, Alastair. ``The costs of corruption for the poor -- The Energy Sector.'' Public Policy for the Private Sector, April, 2000. 1 Nov. 2011. Web. \url{<http://cdi.mecon.gov.ar/biblio/docelec/bm/ppps/N207.pdf>}


\end{workscited}

\end{mla}