\documentclass[12pt,letterpaper]{article}

\usepackage{ifpdf}
\usepackage{mla}
\usepackage{hyperref}

\begin{document}
\begin{mla}{Cameron}{Carroll}{Prof. Mood}{English 124}{\today}{Rituals for a Modern Age}


In the 1997 introduction to ``Blood Rites: Origins and History of the Passion of War," Barbara Ehrenreich postulates that the warrior is a revenant, a shell of the human. This implies that there is a human underneath, but various methods are harnessed to bury it. This introduction, ``The Ecstasy of War," articulates the catalysts for war and their realistic influences: Institutional and societal agendas versus the base, violent instinct in humans. Ehrenreich provides multiple points of view and relies on her tone for rhetorical influence; Psychological and historical basis for a violent disposition hidden inside is introduced, including Freud ironically used as the ethos component of the violent argument. 

The author identifies the dichotomy between instinct and institution fairly, save for a few slips of tone against instinct. After setting aside the question of whether an aggressive instinct exists, she proceeds to dismantle the argument that it drives men to war. She refers to and empathizes with the soldiers repeatedly throughout the text, noting that the men would rather not fight. Throughout history, she claims out, transformations have been required for men to become warriors. The Scandinavian clad in bear skin was not a man, but a warrior. The Greeks and Aztecs and Chinese would undergo intoxicating rituals designed to make them forget their inhibitions: If anything, Ehrenreich implies that the instinct is to survive and must be overcome in order to facilitate war. Numerous examples of transformation are provided; In juxtaposition with the very few examples of base violence, this guides the reader to accept that this inhibition must be overcome.

Knowing the objectives of war, and knowing how warriors are created, Ehrenreich's readers are now set up for her most important postulation: Institutional influence is the factor that drives men (and now, women) to war. These institutions that make up our society rely on their population to do the work. When this work happens to be war, it is ``merely a continuation of policy... by other means." The rituals and the drugs are provided by the society; An individual does not opt to ingest a hallucinogen prior to embarking on a killing spree because that's not in the scope of an individual's needs. The practice of warrior transformation was itself institutionalized: Armies grew from marauding bands and men were then transformed by discipline and exploitation of comradery: These are rituals for a modern age, designed to make a man forget that he is a man and instead a soldier. The author, having established that institutions turn men into warriors, notes that, historically, postwar rites have been used to reintroduce the warrior into society as a man. She writes that ``in war one should kill, should steal, should burn cities and farms..." These atrocities occur in civilian life, but the systematic practice is beyond what men will do soberly... Ehrenreich illustrates that the drive for war pushes the man to become a warrior, and pulls him back out again. This leads to the inference that government and society are greedy and that these wars are not as necessary as we might think, given the epic transformation they must undergo.

\section*{Notes on Author}
Barbara Ehrenreich is a widely-known American writer, known for focusing on issues that are often taboo in discussion; She advocates against war, and for women's rights and social accountability. Born into a union family, Ehrenreich knows better than most the influence of the institution. Her own self-biography can be found at \url{http://www.barbaraehrenreich.com/barbara_ehrenreich.htm}


\end{mla}
\end{document}