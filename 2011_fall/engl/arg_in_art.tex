\documentclass[12pt,letterpaper]{article}

\usepackage{ifpdf}
\usepackage{mla}
\usepackage{hyperref}
\usepackage{setspace}


\begin{document}

\thispagestyle{empty}
\begin{spacing}{1.5}
\begin{center}
{\Large \bfseries
Grossmont College}

\vspace*{10mm}

\includegraphics[width=5cm]{gc_crest}

\vspace*{15mm}
{\large \today
}

\vspace*{10mm}



\LARGE Argument in Art: Until Death do us Part

\vspace*{20mm}

{\large \textsc{ Cameron Carroll}
}

\vspace*{20mm}

\small{Instructor: \textsc{Stephanie Mood}}

\end{center}
\end{spacing}

\newpage
\setcounter{page}{1}


\begin{mla}{Cameron}{Carroll}{Prof. Mood}{English 124}{\today}{Argument in Art}

\begin{abstract}
 Given the sanitation and medicinal shortcomings of the eighteenth century, moral instruction served a greater purpose than simply keeping one's conscience clean; Rather, morals and moral-based stories served and continue to serve as the boundaries of personal health and safety. William Hogarth, in his sextuplet work, argues against myriad moral issues and vices but focusing on lust and gluttony. Throughout the story, each character's vices ultimately bring about their poor fortune, which is Hogarth's vehicle of argument: Suffering.
\end{abstract}


\paragraph{} William Hogarth's `Marriage a la Mode,'  a set of satirical paintings from 1745, is a harsh criticism of the practices of the wealthy and particularly of arranged marriages. It is ``a sequence of six paintings that satirize the marital immoralities of the moneyed classes in England.'' (Kleiner 594) Hogarth explores the coupling of two disinterested young people as ``old Lord Squanderfield contracts his son in marriage to the daughter of a City alderman.'' (Bertelsen 132) 

Hogarth's work, a story in six pieces, begins with the arranged marriage of an uninterested couple amidst the bickering of the two fathers, one wealthy and the other miserly. Almost immediately, their married life is uncouth; They both display evidence of adultery the night before, as the servant walks away with a stack of unpaid bills. The young man, the Viscount, is soon found arguing with a doctor whose mercury pills haven't worked as planned against his syphilis, evident on the prostitute which accompanies him. The Viscount's wife, Lady Squanderfield, also finds herself caught up in an affair, as while she lounges with another man during a party and supposing to attend a masquerade, a child holds a horned figure to symbolize cuckoldry. Finally, the Viscount catches his wife and her lover, the lawyer Silvertongue, together; Before escaping out the window, Silvertongue stabs and thus murders the Viscount. In a coup de grace, Hogarth depicts the Lady Squnderfield's corpse, looked upon by her daughter while father lifts jewelry. The Lady has poisoned herself from grief, not for her husband but for her lover Silvertongue, who has been hanged. Her only legacy is in the child that looks upon her, already doomed from congenital syphilis. The Viscount and his Viscountess... The Countess and her daughter.  All sentenced to death by their own vices.



\paragraph{} The first piece of the tale lies in \emph{The Marriage Settlement.} The bride and groom to be are clearly disinterested, and the lady is already being soothed by another man. The dogs shackled before them represent the loveless, forced marriage ahead. Fathers sit and argue; The banker and architect are absorbed into their respective tasks, reminding the viewer that this marriage is focused on money. One of the fathers can be seen with a bandaged foot, indicating that he suffers from gout. This is an argument against gluttony: While there are so many starving in the world, these few must eat themselves into disability. Hogarth uses the tension between business and distress to argue that arranged marriages serve the interests of the minority: In this case, each father has something to gain from the other, and cares not for the interests of the children involved, or of creating a stable marriage.

\emph{The Breakfast Scene} follows, which gives a glimpse into the already dysfunctional life of this married couple. The disheveled room is immediately striking and the servant's expression of disgust appears to be justified. The Lord and Lady both display evidence of infidelity in their ragged tiredness. Clearly displayed is a black patch upon the Viscount's neck, indicating that he has (and had prior to his marriage) a venereal disease. The man sports a woman's cap in his pocket, while his wife contentedly stretches with her sly look and damp skirt: Both have spent the night producing the stack of bills being carried away, as well as cheating upon one another. Hogarth supports his argument against arranged marriages by this fictional demonstration of a faithless facade of a matrimony: The infidelity and avarice are his prime examples, but the artist took the time to insult their dubious taste as well. Visible just behind the columns is an erotic painting sheathed by curtain, next to a number of religious figures. The fireplace style things (cite that and do it right)

The Viscount thrusts into the third scene smiling, although his business is quite serious; He has returned to demand a refund for the faulty pills to cure his syphilis, all the while with a prostitute on his arm, dabbing her face for syphilis. Hogarth's criticism of this setting is threefold: First, the Viscount's venereal problem is in full bloom, and his utter disregard for anyone else in terms of spreading it is intended. Second, the prostitute he bears is still a child and is an argument against the trade, especially with respect to children. The tertiary argument is of the doctor himself, who is illustrated as a quack whose medications have failed, and whose practice exists only as a trap to unwary fools. The various `medical' equipment scattered around the room ``are numerous icons of death, such as skulls, skeletons, anatomical dissections, and a torture machine complete with French instruction book... this painting/etching is of interest not only for its depiction of quackery but also for the iconography of death which fills this interior scene.'' (`NYU Lit/Art/Med Database')

Hogarth's fourth scene takes place in the Lady's bedroom, where she celebrates becoming Countess after her father-in-law's death. The overbearing theme of this piece is of lust and infidelity... the aforementioned symbol of cuckoldry is visible in the lower right, where the child holds a horned figure. The lawyer Silvertongue reclines near the new Countess, and there is even a painting of him upon the wall. Hogarth is arguing that adultery splits a relationship apart, such that the Earl (formerly Viscount) has not even attended to his wife's bedroom, and that she would have the gall to put up a painting of her lover.

Marriage a la Mode's penultimate painting depicts the murder of the Earl and the escape of Silvertongue, the murderer. Hogarth invokes the vices of envy, wrath and lust and a deduction of his components suggests that he is arguing that each person acts primarily in their own interest: Every character has their own agenda, and while the Countess begs for forgiveness, it is clear that through the Earl's death or not, the marriage is shattered. This drama takes place in a Bagnio, which Hogarth intended to mean a seedy place to rent a room. The term Bagnio has a number of historical meanings, others including ``A prison for slave in Asian countries'' and ``A public bathhouse in Italy or Turkey'' (`The Free Dictionary')

The Lady, ultimately, poisons herself from grief. The last scene bears witness to the dead Countess receiving a final kiss from her daughter, and her father lifts a ring off of her finger to fund his meager existence. The decor of the house, the state of its affairs and the emaciated dog taking the dinner all show the lifestyle of the Countess' father, whose plan to obtain a dowry to live by has failed with her suicide. The utter disregard and betrayal of both the Countess and Earl to each other has culminated in a death spiral for both. Hogarth argues that this is all the outcome of a rotten process and a rotten lifestyle: Arranged marriages between lazy, gluttonous, even cruel people results in tragedy and disappointment. 

Hogarth concludes his tale with a final tragedy: ``the last kiss of a vice-crippled and presumably doomed little girl for her mother, once a naive little girl herself, but now an adulteress and suicide.'' (Bertelsen 133) This refers to the congenital syphilis apparent on the child's face, and the brace upon her leg indicating overconsumption and gluttony. Widespread veneral disease is one of the major topics which Hogarth invokes in his paintings, arguing that infidelity and adultery are immoral and have dire consequences.   Hogarth's work invokes all of Dante's seven virtues, but focuses on lust and gluttony. The child is crippled terminally because of the vices of her parents, but also crippled because the habits she has been brought up into are gluttonous and greedy by nature.

\paragraph{} The arguments sewed into Marriage a la Mode do not apply explicitly to the upper classes: Venereal disease, adultery, substance abuse and the entire spectrum of sins and vices are prevalent in every part of society, regardless of wealth. Hogarth ``debunks the alleged elegance and respectability of the aristocracy by revealing their disreputable conduct and their lack of moral judgment. '' (`English for...') Succinctly put, Hogarth is arguing that the rich and powerful are not exempt from moral and legal bindings, especially when they impose those bindings upon others. This hypocritical action is much like contemporary police brutality \& allegiance questions, or like legislative exemptions for legislators.

\begin{quote}
I have generally found that persons who had studied painting least were the best judges of it. - William Hogarth
\end{quote}

\section*{Works Cited}


Kleiner, Fred. ``Art through the Ages: The Western Perspective'' (2010) Wadsworth, Boston, MA. Print.

Bertelsen, Lance. ``The Interior Structures Of Hogarth's Marriage À La Mode.''  \emph{Art History 6.2} (1983): 131-142. \emph{Academic Search Premier.} Web. 22 Nov. 2011.

`Bagnio - Definition' \emph{The Free Dictionary} Web. 28 Nov. 2011. /url{http://www.thefreedictionary.com/bagnio}

`Marriage a la Mode: The Visit to the Quack Doctor' \emph{NYU School of Medicine Literature, Arts, Medicine Database} Web. 29 Nov. 2011. \url{http://litmed.med.nyu.edu/Annotation?action=view&annid=10348}

``William Hogarth - The Marriage a la Mode - Plate II'' /emph{English for Art Historians (Blog)} Web. 1 Dec. 2011. \url{http://arthistoryinenglish.edublogs.org/painting/xviiith-century-art/william-hogarth-the-marriage-a-la-mode-plate-ii/}

\end{mla}

\begin{mla}{Cameron}{Carroll}{Prof. Mood}{English 124}{\today}{Argument in Art: Rhetorical Precis}
Lance Bertelsen's 1983 article, ``The Interior Structures of Hogarth's Marriage A La Mode,'' asserts that Hogarth separated his six-piece work into three pairs, divided both stylistically and thematically. The author supports his claims using technical art history analysis as well as an insight into Hogarth's character to synthesize an argument from individual elements, or pairs of elements. The author attempts to decode Hogarth's work in order to illuminate any possible meaning regarding the minute details of the work. The author is writing with an audience in mind that is at least somewhat familiar with artistic elements and themes.
\end{mla}

\begin{mla}{Cameron}{Carroll}{Prof. Mood}{English 124}{\today}{Argument in Art: Reflection}
\begin{itemize}
\item It took a long time to select a work of art that I was actually interested in reading and writing about. The most difficult part by far was coming up with ideas... actually exploring them and writing about them isn't hard once it's started.
\item I don't remember and didn't write down my peers' names. The workshop was very helpful this time though, as I gained a deeper understanding of my own paper and how to communicate my ideas by helping my groupmates understand the initial, very confusing, paper.
\item I feel like my paper is weak on sources, technical commentary and expert opinion. I think it has a strong personal commentary though.
\item This essay helped me think about art in ways I simply didn't care to before. It's been years since I've actually sat down with a piece of art and written about its themes and elements.
\end{itemize}
\end{mla}
\end{document}