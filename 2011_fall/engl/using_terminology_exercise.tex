\documentclass[12pt,letterpaper]{article}

\usepackage{ifpdf}
\usepackage{mla}
\usepackage{setspace}

\begin{document}
\begin{mla}{Cameron}{Carroll}{Prof. Mood}{English 124}{\today}{Using Terminology Exercise: Modern Representation in the Electoral College}
\singlespacing
\paragraph{}
The electoral college, established by name in 1845, is an overbearing but often overlooked component of American life; This complicated system represents a number of separate interests which do not always agree with oneanother. Once again, just over a decade ago, the American electoral college yielded a result which did not agree with the popular decision: Some would call this a perversion of democracy. This event has thrust the electoral system back into the spotlight, and raises the questions... ``Whose interests deserve most fully to be represented by the electoral college?" ... ``Which parties suffer as a result of under-representation in the college?" ... succintly: ``Should we eliminate  the Electoral College system?"




\section*{In Favor of Elimination of Electoral College: (Pros)}


\paragraph{Risk of Faithless Electors:}
Due to the indirect nature of the electoral college, an individual vote is not a vote for president. Rather, it is a vote for the person who gets to vote for president: Consistent with the representative democracy, the decision is made for you. This is an appeal to pathos, particularly to the tendency for people to want to be self sufficient. By wording it so that the ``decision is made for you,'' the self-actualization and self-esteem of an individual are insulted. The values of success and security are invoked by this argument. There is also an appeal to the fear that a vote will be cast incorrectly on purpose.                                     


\paragraph{Irrelevancy of Popular Decision:}
There is a great concern about the college yielding a president which was not supported by the majority. and the underrepresentation of more populous areas. This is another appeal to pathos, because it focuses on how a person will be deprived of their right to vote fairly, essentially. It is reminiscient of the 3/5\textsuperscript{ths} rule, and is also concerned with personal attribution, victory and having basic freedoms/individual rights such as voting.

\paragraph{Places emphasis on swing states:}
States which vote tentatively garner 99\% of the attention from candidates, despite not representing even close to 99\% of the population. If you happen to live outside one of these ~17 states, presidential hopefuls often simply couldn't care less.
To an extent, here, there is an appeal to needs, in individual rights, patriotism and a sense that a presidential candidate has a duty to consider the concerns of the states containing the vast majority of the population.

\paragraph{System is undemocratic and complicated:}
There is a great deal of fuss made about democracies and the fairness and freedom of America. But the electoral college is not democratic, and it's not fair. This is an appeal to logos, because logically designed systems are efficient and simple. It's also an appeal to pathos, where the values of order and control are associated with a direct vote and no middleman.

\paragraph{Discourages voter participation:}
For those voters in a high population, non-swing state, there is a sense of futility in showing up to vote; Statistically, a vote in Florida or California can count for as little as (or less than) 1/3 that of a low-population swing state constituent. This is an appeal to pathos, where the individual rights that every American is entitled to include the right to a vote. One assumes that vote is guaranteed to be equal, and it is not so in this electoral system. 


\section*{In Favor of Conservation of Electoral College: (Cons)}

\paragraph{Unique Method of Election:}
Part of the appeal of the electoral college is that it is a unique system; American nationalism has always been concerned with being its 'own.' American infallibility plays some part here, as well. This argument uses ethos in the form of longetivity: Our country has a historical basis upon the electoral college system, and as the propaganda goes, this country is the best and most successful in the world. The foundation of this great nation has been set upon the electoral college, and Madison, Franklin \& Washington all provide a solid base of ethos. This tradition is, of course, a fallacy of pathos.

\paragraph{Supports Presidential Legitimacy:}
The electoral system makes an elected president seem more legitimate and helps him 'come to power' by 'magnifying the winner's margin of victory.' This is an appeal to success and victory, because it invokes the sense of a strong presidency, an orderly government, and a secure executive branch.

\paragraph{States are Significant:}
Under a popular vote system, many of the small states that are today battlegrounds tomorrow would be nothing. The electoral college gives some power to states that do not necessarily have the population base to make a difference otherwise. This is an appeal to logos; Logically, we want equality throughout the union; The congressional deficiency that smaller states suffer is made up for by their senatoral memership, both in the senate and the college. This argument is somewhat fallacious, however, because the ideals that the architects held do not necessarily reflect what is needed today. It is an appeal to tradition, particularly to the will and viewpoints of men who lived hundreds of years ago.

\paragraph{Electoral college was intended to reflect more than the popular will:}
The electoral college was built with states in mind, and gives many advantages to keep small states relevant. Because of its dependency upon senate seats, a number of small states can produce a great number of electors, keeping them competitive politically with the larger (more populous) states. This is an appeal to ethos, given the historical longetivity of the system and the great reverence for its architects. This argument is fa                                 

\paragraph{Popular Vote Places Emphasis on Population Density:}
Rather than a small handful of swing states, candidates would seek out the lowest cost per person reached, which is most likely in heavy urban areas; Essentially, this argument is that the popular vote has its own problems in unfairness, and that switching may push us to the other extreme. This is fallacious first of all because it is a false dilemma: The popular vote is not the only other option, and there are many successful democracies out there. 

\section*{Evaluation:}

\paragraph{Opinion \& Analysis:}
The issue of the electoral college is a question of respresentation. Many people agree with our founding fathers' decision to increase the power of the state over the individuals. Many others believe that the government is an instritution that should work for the people, and when it starts working toward other means, it is time to overthrow it. The division between appeals used becomes clear upon inspection, where arguments often used by conservatives focus on values whereas liberals tend toward logic and evidence. When I picked this issue I was unconditionally for the elimination of the college, but I also hadn't really thought about it or researched the issue. I still believe that it is a flawed system, and I think that it is unfair espesically to Californians, but I found fewer reasons for its elimination than I had expected. In fact, only four were consistent throughout arguments.

\section*{Resources:}

\begin{itemize}
\item http://www.pennumbra.com/debates/debate.php?did=8 \\
\item http://www.congresslink.org/print\_expert\_electoral/college.htm \\
\end{itemize}

\end{mla}
\end{document}