\documentclass[12pt,letterpaper]{article}

\usepackage{ifpdf}
\usepackage{mla}
\usepackage{hyperref}
\usepackage{setspace}


\begin{document}
\begin{mla}{Dealgnaid}{McNamara}{T. Ding}{British Literature \\ Essay Part One: Middle Ages \#2}{\today}{The Miller's Insights}

\paragraph{}
There is a great deal to be noted and learned from the tale recounted by Geoffrey Chaucer for his fictional Miller in The Canterbury Tales, historically and socially alike. The stories told both by him and about him and his fellows have survived the test of time most heartily and remain today to guide us personably to a better understanding of Medieval society and what were considered to be accepted forms of Medieval social networking. All of the characters and even some still surviving character archetypes born of The Canterbury Tales reflect deeply upon the class structure of Great Britain at the time and the consequential personality traits and, often, flaws suffered by the citizens of this time period. Though they are of widely varied lineage, the characters portrayed in the Tales coexist more or less harmoniously over the course of their time spent together, squabbling only briefly and largely indifferently. In one such instance, our Miller makes his grand entrance to the page – or, rather, his adamant interruption.
\paragraph{}
	Though his insolence towards persons of higher rank than himself is socially shocking, it's brushed aside, albeit roughly, in recognition of his intellectual inferiority and general immaturity. He is allowed his say for entertainment's sake alone; the message of his words will not reach past the egos of his comrades.
\paragraph{}
	The Miller's insolence doesn't end with his interruption of the higher ranking Monk who was to precede him; it penetrates the very plot of his Tale and jeers at courtly love, making an essential mockery of the tale told previously by the respectable and valiant Knight. Absolon, a character in The Miller's Tale, has his own unique idea of the courtly love widely popular in Medieval society and clings to it desperately with ardent and unfaltering devotion. Absolon is a courtly lover of his own degree and merit, scrambling almost successfully to emulate true courtly love and prove his heart's worth against his lowly upbringing. His intentions and technique in wooing another main character in the Tale – the carpenter's supple wife, Alisoun – are alluded to by the narrator as a cat and mouse game, or a hunt, as has been often been a popular analogy in poetry of courtly love. In line 237 the narrator observes, “I dar wel sayn, if she hadde been a mous, And he a cat, he wolde hire hente anoon,” meaning the narrator thinks he seems as if he would “pounce on” her given the chance. His persistent yet faltering courtly thoughts and desires are further enforced by his practice of calling at Alisoun's window in what manifests to be a rather uncomely manner. It is known among the townspeople that “Fro day to day this joly Absolon So woweth hire that him is wo-begoon.” (263) Absolon, very much unlike his rival for Alisoun's heart, Nicholas, forces himself through habitual trials and what must be crippling impatience to prove himself to his forbidden love. In juxtaposition to Nicholas's appropriately slapstick approach, his feeble attempt at upholding the standards of courtly love seem a bit laughable and silly. From what we learn of Absolom, in Medieval society courtly love was important and ubiquitous but generally misunderstood, misinterpreted, and even occasionally scoffed at by people of lesser rank and status.
\paragraph{}
	Nicholas, therefore, is similar to a reciprocal of Absolom. Absolom's character is but a lowly townsperson wishing to climb higher in society's ranks and going about it by simply pretending he's already there. Nicholas, on the other hand, is a well educated Oxford student sinking comfortably, albeit lazily, into his minimal potential, as was much easier for higher ranking people to do in Medieval times than it may have been for somebody of Absolom's status, for instance. Thanks to his university education, Nicholas is knowledgeable and savvy enough to manipulate the more common people around him to his will. When he, too, undertakes an interest in the lady Alisoun, he easily thinks of a way to win some time alone with her by convincing her father that God has spoken to him and warned him of a flood, but promised redemption for his household as long as certain steps were taken by its members. People were so incredibly God fearing in Medieval society that this was an unreasonably simple way to manipulate them were they uneducated. The carpenter, Alisoun's father, believed Nicholas's story without question and loaded himself up into a makeshift vessel on the roof of the house. To ensure the carpenter's loyalty and trust, Nicholas offers an impressive display of religious knowledge and invokes the names of several well known saints and other religious figures. In line 375 he exclaims, “Jesu Crist and Sainte Benedight, Blesse this hous from every wikked wight! For nightes nerye the White Pater Noster. Where wentestou, thou Sainte Petres soster?” This technique for blatantly and carelessly manipulating others to one's will by instilling in them the fear of the wrath of God with the front of love, protection, or cautiousness is likely featured in The Miller's Tale to hypothetically  taunt the upcoming Monk, or, the corrupt leaders of the Church at the time. The people of Medieval society were held in fear of God's wrath with a blind faith on a daily basis and commonly made to perform ridiculous feats, as well; often to their own great disadvantage.
\paragraph{}
	There are, of course, myriad additional references, allusions, and even some contextual statements to address in order to fully grasp the workings of Medieval society and its values relayed to us by The Canterbury Tales and its enclosed readings. The Miller's Tale offers us a humorous, almost modern approach to them in an unintimidating and easily followable anecdote. Still today we regard both Chaucer and his characters as clever, relatable people so it is simple to absorb their culture and to pick up clues as to how it came to be, how it functioned, and how it came upon its ultimate downfall. The Medieval time period was a threatening and fearsome but rhythmically and secretively beautiful time period to be alive.

\end{mla}
\end{document}