\documentclass[11pt,letterpaper]{report}
\usepackage[pdftex]{graphicx}

\newcommand{\HRule}{\rule{\linewidth}{0.5mm}}
\setlength{\topmargin}{-.7in}
\setlength{\leftmargin}{-.7in}
\setlength{\textheight}{9in}
\setlength{\oddsidemargin}{0in}
\setlength{\textwidth}{6.25in}


\begin{document}

\begin{titlepage}
\begin{center}

\textsc{\Large Lab 3}\\[1.5cm]
\textsc{\Large Grossmont College - Physics 240}\\[0.5cm]
\includegraphics[width=0.15\textwidth]{./logo.jpg}

\HRule \\[0.4cm]
{ \LARGE \bfseries Mass-Lifter Heat Engine}\\[0.5cm]

\HRule \\[1.5cm]

\begin{minipage}{0.4\textwidth}
\begin{flushleft} \large
\emph{Author:}\\
Cameron \textsc{Carroll}\\[0.2cm]
\emph{Lab Partners:}\\
Stephany  \textsc{Vale}\\
Hani \textsc{Deineh}\\

\end{flushleft}
\end{minipage}
\begin{minipage}{0.4\textwidth}
\begin{flushright} \large
\emph{Instructor \& Class:}\\
Ross \textsc{Cohen} - Phys 240 (4150)
\end{flushright}
\end{minipage}

\vfill

{\large \today}

\end{center}
\end{titlepage}
\pagebreak

\section*{Summary}
\begin{tabular}{ | c | c | c | c | c | c | }
\hline
Trial & $T_{hot}$ & $T_{cold}$ & $\epsilon_{mechanical}$ & $\epsilon_{thermodynamic}$ & $\epsilon_{carnot}$ \\
\hline
1 & 353 K & 291.3 K & n/a & 21\% & 17.5\% \\
\hline
2 & 339 K & 315.5 K & n/a & 1.3\% & 6.9\% \\
\hline
\end{tabular}


\section*{Conclusions}
\paragraph{}
We recorded and operated a rudimentary heat engine consisting of an apparatus whose internal gas is expanded and contracted in a cycle to yield work. The accuracy of the experiment was very much at the whim of our ability to operate the engine, and while some effort was taken to get smooth runs, the human error involved cannot be overlooked. Each corner of the P-V graph has little feet, indicating some lag while switching between stages of the engine.
\paragraph{}
The thermodynamic efficiency for the first trial came out to be greater than the carnot efficiency, suggesting some error (21\% thermodynamic versus 17.5\% carnot.) In addition, we failed to record the height displacement and without that data cannot compute the mechanical efficiency.
\paragraph{}
The trial with a smaller temperature range has much lower percentage efficiency, which is expected because heat engines operated based on a difference in temperature.


\end{document}